\documentclass[a4paper, 11pt]{article}

% Fonts 
\usepackage{opensans}
\usepackage{amsfonts}
\usepackage{montserrat}
\usepackage{amsmath}

\usepackage[mathrm=sym]{unicode-math}

\setmainfont{opensans}
\setmathfont{Fira Math}

\newfontfamily{\montserrateb}{Montserrat SemiBold}
\newfontfamily{\montserratb}{Montserrat Bold}
\newfontfamily{\montserrat}{Montserrat Regular}
\newfontfamily{\montserratl}{Montserrat Light}
\DeclareMathAlphabet{\mathcal}{OMS}{cmbrs}{m}{n}

% \autoref
\usepackage{hyperref}

% Use for [H] option for figures to force in text placement
\usepackage{float}

% Captioning figures
\usepackage{caption}

% Subfigures
\usepackage{subcaption}

% For extending contents beyond margins
\usepackage{scrextend}

% For tables \midrule ect
\usepackage{booktabs}

% Colours
\usepackage[table,xcdraw]{xcolor}
\definecolor{accentcolor}{HTML}{6332a8}

% Change label in enumerate 
\usepackage{enumitem}

% Section settings
\usepackage{titlesec}
\titleformat{\section}
{\LARGE\montserrateb}
{\thesection.}{0.5em}{}

\titleformat{\subsection}
{\large\montserratb}
{\thesubsection.}{0.5em}{}

% Adjust document dimensions
\ExecuteOptions{a4paper}
\addtolength{\oddsidemargin}{-3cm} 
\addtolength{\evensidemargin}{-3cm}
\addtolength{\topmargin}{-3cm} 
\addtolength{\textwidth}{6cm}
\addtolength{\textheight}{4.5cm}
\addtolength{\textheight}{1.5cm}
\addtolength{\headsep}{-0.5cm}
% \addtolength{\footskip}{-1cm}
\parindent0pt
\parskip=4pt



\usepackage{tikz}
\usetikzlibrary{arrows.meta,arrows}
\newcommand*{\TickSize}{0}%

\newcommand*{\AxisMin}{0}%
\newcommand*{\AxisMax}{0}%




\newcommand*{\DrawVerticalPhaseLine}[7][]{%
    \gdef\AxisMin{#5}%
    \gdef\AxisMax{#6}%
    \gdef\TickSize{#7}


    \edef\MyList{#2}% Allows for #1 to be both a macro or not
    \foreach \X in \MyList {
      \draw  (-\TickSize,\X) -- (\TickSize,\X) node [right] {$#1\;=\X$};
      % \node [circle, fill:radius 5pt] at (0,\X) {$y=\X$};
    }

    \edef\MyList{#3}% Allows for #2 to be both a macro or not
    \foreach \X in \MyList {% Up arrows
      \draw [-{Latex[length=2mm, width=2mm]}] (0,\X-0.2) -- (0,\X);
    }

    \edef\MyList{#4}% Allows for #3 to be both a macro or not
    \foreach \X in \MyList {% Down arrows
      \draw [-{Latex[length=2mm, width=2mm]}] (0,\X+0.2) -- (0,\X);
    }

    \draw  (0,\AxisMin) -- (0,\AxisMax) node [above] {#1};
}%

\usepackage{mdframed}

% Creates coloured title box
\newcommand{\thetop}[5]{
    \begin{addmargin}[\oddsidemargin]{\oddsidemargin}
        \colorbox{#5}{\color{white}
          \hbox to \paperwidth{
            \vbox {
              \begin{center}
                {\large\montserratl #1}\\
                \vspace{4pt}
                {\huge\montserratb #2}\\
                {\montserratb #3}\\
                \vspace{-0.5em}
                \rule{20em}{1pt}     
      
                {\large\montserratl 
                    #4
                }
              \end{center}
            }
          }
        }
    \end{addmargin}
}

\newcommand{\NN}{\mathbb{N}}
\newcommand{\ZZ}{\mathbb{Z}}
\newcommand{\RR}{\mathbb{R}}
\newcommand{\CC}{\mathbb{C}}
\newcommand{\dydt}{\frac{dy}{dt}}
\newcommand{\dxdt}{\frac{dx}{dt}}
\def\set#1{\left\{ #1 \right\}}
\def\eval#1#2{\left\ #1\right|_{#2}}
\begin{document}
\thetop{Robert Christie}{MATHS 340}{S2 2023}{Assignment 1\\Due: 1-08-2023}{accentcolor}

\section*{Q1}
\begin{enumerate}[label=(\alph*)]
  \item
        We can see that the surface $S$, is defined implicitly $f(x,y,z)=0$ where $f:\RR^3\to \RR$ is continuously differentiable (as it is a polynomial). Thus, we can see that as $(a,b,c)=(1,2,1)$ is a surface point, the tangent plane is given by:
        \begin{align*}
          0 & = \frac{\partial f}{ \partial x}(a,b,c)(x-a) +
          \frac{\partial f}{ \partial y}(a,b,c)(y-b) +
          \frac{\partial f}{ \partial z}(a,b,c)(z-c)         \\
            & = 2a(x-a) + 4b(y-b) - 10c(z-c)                 \\
            & = 2(x-1) + 8(y-2) -10(z-1)                     \\
            & = 2x - 2 + 8y -16 -10z + 10                    \\
            & = 2x  + 8y  -10z -8                            \\
        \end{align*}

        % We see that $f(1,2,1)=0$, thus the point is indeed on the surface. We can use the formula given in lectures for the tangent plane to an implicit surface $S$ defined by $f(x,y,z)=0$
  \item \textcolor{red}{MATLAB}
  \item We are considering a surface defined by a function $f(x,y,z)=0$,




\end{enumerate}



\section*{Q3}
\begin{enumerate}[label=(\alph*)]
  \item A
  \item The mass of the wire will be given by:
        \begin{align*}
          m & = \int_0^{\frac{\pi}2}\rho(r(t))dt
        \end{align*}
\end{enumerate}

% When $(a,b,c)$ is on the surface $S$, then the tangent plane at $(a,b,c)$ is:
% $$
%   \frac{\partial f}{ \partial x}(a,b,c)(x-a) +
%   \frac{\partial f}{ \partial y}(a,b,c)(y-b) +
%   \frac{\partial f}{ \partial z}(a,b,c)(z-c)
% $$
\end{document}