\documentclass[a4paper, 11pt]{article}

% Fonts 
\usepackage{opensans}
\usepackage{amsfonts}
\usepackage{montserrat}
\usepackage{amsmath}

\usepackage[mathrm=sym]{unicode-math}

\setmainfont{opensans}
\setmathfont{Fira Math}

\newfontfamily{\montserrateb}{Montserrat SemiBold}
\newfontfamily{\montserratb}{Montserrat Bold}
\newfontfamily{\montserrat}{Montserrat Regular}
\newfontfamily{\montserratl}{Montserrat Light}
\DeclareMathAlphabet{\mathcal}{OMS}{cmbrs}{m}{n}

% \autoref
\usepackage{hyperref}

% Use for [H] option for figures to force in text placement
\usepackage{float}

% Captioning figures
\usepackage{caption}

% Subfigures
\usepackage{subcaption}

% For extending contents beyond margins
\usepackage{scrextend}

% For tables \midrule ect
\usepackage{booktabs}

% Colours
\usepackage[table,xcdraw]{xcolor}
\definecolor{accentcolor}{HTML}{6332a8}

% Change label in enumerate 
\usepackage{enumitem}

% Section settings
\usepackage{titlesec}
\titleformat{\section}
{\LARGE\montserrateb}
{\thesection.}{0.5em}{}

\titleformat{\subsection}
{\large\montserratb}
{\thesubsection.}{0.5em}{}

% Adjust document dimensions
\ExecuteOptions{a4paper}
\addtolength{\oddsidemargin}{-3cm} 
\addtolength{\evensidemargin}{-3cm}
\addtolength{\topmargin}{-3cm} 
\addtolength{\textwidth}{6cm}
\addtolength{\textheight}{4.5cm}
\addtolength{\textheight}{1.5cm}
\addtolength{\headsep}{-0.5cm}
% \addtolength{\footskip}{-1cm}
\parindent0pt
\parskip=4pt



\usepackage{tikz}
\usetikzlibrary{arrows.meta,arrows}
\newcommand*{\TickSize}{0}%

\newcommand*{\AxisMin}{0}%
\newcommand*{\AxisMax}{0}%




\newcommand*{\DrawVerticalPhaseLine}[7][]{%
    \gdef\AxisMin{#5}%
    \gdef\AxisMax{#6}%
    \gdef\TickSize{#7}


    \edef\MyList{#2}% Allows for #1 to be both a macro or not
    \foreach \X in \MyList {
      \draw  (-\TickSize,\X) -- (\TickSize,\X) node [right] {$#1\;=\X$};
      % \node [circle, fill:radius 5pt] at (0,\X) {$y=\X$};
    }

    \edef\MyList{#3}% Allows for #2 to be both a macro or not
    \foreach \X in \MyList {% Up arrows
      \draw [-{Latex[length=2mm, width=2mm]}] (0,\X-0.2) -- (0,\X);
    }

    \edef\MyList{#4}% Allows for #3 to be both a macro or not
    \foreach \X in \MyList {% Down arrows
      \draw [-{Latex[length=2mm, width=2mm]}] (0,\X+0.2) -- (0,\X);
    }

    \draw  (0,\AxisMin) -- (0,\AxisMax) node [above] {#1};
}%

\usepackage{mdframed}

% Creates coloured title box
\newcommand{\thetop}[5]{
    \begin{addmargin}[\oddsidemargin]{\oddsidemargin}
        \colorbox{#5}{\color{white}
          \hbox to \paperwidth{
            \vbox {
              \begin{center}
                {\large\montserratl #1}\\
                \vspace{4pt}
                {\huge\montserratb #2}\\
                {\montserratb #3}\\
                \vspace{-0.5em}
                \rule{20em}{1pt}     
      
                {\large\montserratl 
                    #4
                }
              \end{center}
            }
          }
        }
    \end{addmargin}
}

\newcommand{\NN}{\mathbb{N}}
\newcommand{\ZZ}{\mathbb{Z}}
\newcommand{\RR}{\mathbb{R}}
\newcommand{\CC}{\mathbb{C}}
\newcommand{\dydt}{\frac{dy}{dt}}
\newcommand{\dxdt}{\frac{dx}{dt}}
\def\set#1{\left\{ #1 \right\}}
\def\eval#1#2{\left\ #1\right|_{#2}}
\begin{document}
\thetop{Robert Christie}{MATHS 340}{S2 2023}{Assignment 1\\Due: 1-08-2023}{accentcolor}

\section*{Q1}
\begin{enumerate}[label=(\alph*)]
  \item
        We can see that the surface $S$, is defined implicitly $f(x,y,z)=0$ where $f:\RR^3\to \RR$ is continuously differentiable (as it is a polynomial). Thus, we can see that as $(a,b,c)=(1,2,1)$ is a surface point, the tangent plane is given by:
        \begin{align*}
          0 & = \frac{\partial f}{ \partial x}(a,b,c)(x-a) +
          \frac{\partial f}{ \partial y}(a,b,c)(y-b) +
          \frac{\partial f}{ \partial z}(a,b,c)(z-c)         \\
            & = 2a(x-a) + 4b(y-b) - 10c(z-c)                 \\
            & = 2(x-1) + 8(y-2) -10(z-1)                     \\
            & = 2x - 2 + 8y -16 -10z + 10                    \\
            & = 2x  + 8y  -10z -8                            \\
        \end{align*}

        % We see that $f(1,2,1)=0$, thus the point is indeed on the surface. We can use the formula given in lectures for the tangent plane to an implicit surface $S$ defined by $f(x,y,z)=0$
  \item \textcolor{red}{MATLAB}
  \item
        We are considering a surface defined by a function $f(x,y,z)=0$, evaluating $f(0,0,1)=-9\neq 0$, thus the point does not lie on the surface, and it does not make sense to ask for the tangent plane at this point.

        We could still use $f$ to find a plane at this point from the partial derivatives of $f$, however, this plane is not determined by the surface, it is determined by the choice of implicit function $f$, any plane passing through this point could match this definition by choosing a different $f$ that gives the same surface.




\end{enumerate}

\def\pp#1#2{\dfrac{\partial#1}{\partial#2}}
\def\dd#1#2{\dfrac{d#1}{d#2}}


\section*{Q2}
\begin{enumerate}[label=(\alph*)]
  \item Consider the function $\zeta: \begin{bmatrix}
            u \\ v
          \end{bmatrix} \mapsto \begin{bmatrix}
            x \\ y
          \end{bmatrix}$. From the definitions given we know:
        $$
          \zeta: \begin{bmatrix}
            u \\v
          \end{bmatrix} \mapsto \begin{bmatrix}
            ve^u \\
            ve^{-u}
          \end{bmatrix}
        $$
        Thus we find that:
        $$
          D\zeta = \left[ \begin{array}{c|c}\pp\zeta u& \pp\zeta v\end{array} \right] = \begin{bmatrix}
            ve^u     & e^u    \\
            -ve^{-u} & e^{-u} \\
          \end{bmatrix}
        $$
        The columns of $D\zeta$ give us a moving frame of basis vectors, which we normalise to give unit vectors:

        \begin{minipage}[t]{0.5\textwidth}
          \begin{align*}
            e_u & = \frac{     \begin{bmatrix}
                                 ve^u \\
                                 -ve^{-u}
                               \end{bmatrix}}{
            \sqrt{v^2e^{2u} + v^2e^{-2u}}
            }                                           \\
                & = \frac{     \begin{bmatrix}
                                   e^u \\
                                   -e^{-u}
                                 \end{bmatrix}}{
              \sqrt{e^{2u} + e^{-2u}}
            }                                           \\
                & =    \begin{bmatrix}
                         \sqrt{\frac{e^{2u}}{e^{2u} + e^{-2u}}} \\
                         -\sqrt{\frac{e^{-2u}}{e^{2u} + e^{-2u}}}
                       \end{bmatrix}     \\
                & =    \begin{bmatrix}
                         \sqrt{\frac{1}{\frac{e^{2u} + e^{-2u}}{e^{2u}}}} \\
                         -\sqrt{\frac{1}{\frac{e^{2u} + e^{-2u}}{e^{-2u}}}}
                       \end{bmatrix}    \\
                & =    \begin{bmatrix}
                         \frac{1}{\sqrt{1 + e^{-4u}}} \\
                         \frac{-1}{\sqrt{1 + e^{4u}}}
                       \end{bmatrix}
          \end{align*}
        \end{minipage}\begin{minipage}[t]{0.5\textwidth}
          \begin{align*}
            e_v & = \frac{     \begin{bmatrix}
                                 e^u \\
                                 e^{-u}
                               \end{bmatrix}}{
            \sqrt{e^{2u}+e^{-2u}}
            }                                           \\
                & =    \begin{bmatrix}
                         \sqrt{\frac{e^{2u}}{e^{2u} + e^{-2u}}} \\
                         \sqrt{\frac{e^{-2u}}{e^{2u} + e^{-2u}}}
                       \end{bmatrix}    \\
                & =    \begin{bmatrix}
                         \sqrt{\frac{1}{\frac{e^{2u} + e^{-2u}}{e^{2u}}}} \\
                         \sqrt{\frac{1}{\frac{e^{2u} + e^{-2u}}{e^{-2u}}}}
                       \end{bmatrix}    \\
                & =    \begin{bmatrix}
                         \frac{1}{\sqrt{1 + e^{-4u}}} \\
                         \frac{1}{\sqrt{1 + e^{4u}}}
                       \end{bmatrix}
          \end{align*}
        \end{minipage}

  \item See that:
        \begin{align*}
          e_u +e_v & =
          \begin{bmatrix}
            \frac{1}{\sqrt{1 + e^{-4u}}} \\
            \frac{-1}{\sqrt{1 + e^{4u}}}
          \end{bmatrix}
          +\begin{bmatrix}
             \frac{1}{\sqrt{1 + e^{-4u}}} \\
             \frac{1}{\sqrt{1 + e^{4u}}}
           \end{bmatrix}                           \\
                   & =\begin{bmatrix}
                        \frac{2}{\sqrt{1 + e^{-4u}}} \\
                        \frac{1-1}{\sqrt{1 + e^{4u}}}
                      \end{bmatrix}               \\
                   & =\begin{bmatrix}
                        \frac{2}{\sqrt{1 + e^{-4u}}} \\
                        0
                      \end{bmatrix} = \frac{2}{\sqrt{1 + e^{-4u}}}e_x
        \end{align*}
        Now for $e_u-e_v$ we see that:
        \begin{align*}
          e_u -e_v & =
          \begin{bmatrix}
            \frac{1}{\sqrt{1 + e^{-4u}}} \\
            \frac{-1}{\sqrt{1 + e^{4u}}}
          \end{bmatrix}
          -\begin{bmatrix}
             \frac{1}{\sqrt{1 + e^{-4u}}} \\
             \frac{1}{\sqrt{1 + e^{4u}}}
           \end{bmatrix}                           \\
                   & =\begin{bmatrix}
                        \frac{0}{\sqrt{1 + e^{-4u}}} \\
                        \frac{-1-1}{\sqrt{1 + e^{4u}}}
                      \end{bmatrix}              \\
                   & =\begin{bmatrix}
                        0 \\\frac{-2}{\sqrt{1 + e^{4u}}}
                      \end{bmatrix} = \frac{-2}{\sqrt{1 + e^{4u}}}e_y
        \end{align*}
  \item We find the partial derivatives of the unit vectors:
        % \begin{align*}
        %   \pp{e_u}u & = 
        % \end{align*}


        \begin{mdframed}
          \begin{minipage}[t]{0.5\textwidth}
            \begin{align*}
              \pp{e_u}u & = \pp{}u\begin{bmatrix}
                                    \frac{1}{\sqrt{1 + e^{-4u}}} \\
                                    \frac{-1}{\sqrt{1 + e^{4u}}}
                                  \end{bmatrix}                       \\
                        & = \begin{bmatrix}
                              \pp{}u\left( 1 + e^{-4u} \right)^{-\frac 12} \\
                              -\pp{}u\left(1 + e^{4u} \right)^{-\frac 12}
                            \end{bmatrix}                 \\
                        & =  \begin{bmatrix}
                               \frac{-1}2\left( 1+e^{-4u} \right)^{\frac{-3}2}(0-4e^{-4u}) \\
                               -\frac{-1}2\left( 1+e^{4u} \right)^{\frac{-3}2}(0+4e^{4u})
                             \end{bmatrix} \\
                        & =  2\begin{bmatrix}
                                e^{-4u}\left( 1+e^{-4u} \right)^{\frac{-3}2} \\
                                e^{4u}\left( 1+e^{4u} \right)^{\frac{-3}2}
                              \end{bmatrix}               \\
                        & =
              \frac{e^{-4u}}{
              1+e^{-4u}
              }(e_u+e_v)
              -\frac{e^{4u}}{1+e^{4u}}(e_u-e_v)                                          \\
                        & = \frac{e_u+e_v}{1+e^{4u}}-\frac{e_u-e_v}{1+e^{-4u}}
            \end{align*}
          \end{minipage}\begin{minipage}[t]{0.5\textwidth}
            \begin{align*}
              \pp{e_u}v & = \pp{}v\begin{bmatrix}
                                    \frac{1}{\sqrt{1 + e^{-4u}}} \\
                                    \frac{-1}{\sqrt{1 + e^{4u}}}
                                  \end{bmatrix} \\
                        & = 0
            \end{align*}
          \end{minipage}

          \begin{minipage}[t]{0.5\textwidth}
            \begin{align*}
              \pp{e_v}u & = \pp{}u\begin{bmatrix}
                                    \frac{1}{\sqrt{1 + e^{-4u}}} \\
                                    \frac{1}{\sqrt{1 + e^{4u}}}
                                  \end{bmatrix}               \\
                        & =  2\begin{bmatrix}
                                e^{-4u}\left( 1+e^{-4u} \right)^{\frac{-3}2} \\
                                -e^{4u}\left( 1+e^{4u} \right)^{\frac{-3}2}
                              \end{bmatrix}       \\
                        & =
              \frac{e^{-4u}}{1+e^{-4u}}(e_u+e_v)
              +\frac{e^{4u}}{1+e^{4u}}(e_u-e_v)                                  \\
                        & = \frac{e_u+e_v}{1+e^{4u}} + \frac{e_u-e_v}{1+e^{-4u}}
            \end{align*}
          \end{minipage}\begin{minipage}[t]{0.5\textwidth}
            \begin{align*}
              \pp{e_v}v & = \pp{}v\begin{bmatrix}
                                    \frac{1}{\sqrt{1 + e^{-4u}}} \\
                                    \frac{1}{\sqrt{1 + e^{4u}}}
                                  \end{bmatrix} \\
                        & = 0\end{align*}
          \end{minipage}
        \end{mdframed}


        Finding the velocity in terms of $e_u,e_v$:
        \begin{align*}
          R'(t) & = \dd{R(t)}t                                                                                                                                                                                    \\
                & = \dd{}t\Big[u(t)\cdot e_u(u(t),v(t))  \Big] + \dd{}t\Big[ v(t)\cdot e_v(u(t),v(t)) \Big]                                                                                                       \\
                & = \left[ u'(t)e_u  +u(t)\cdot\dd{}t\Big[e_u(u(t),v(t))\Big] \right] + \left[   v'(t)e_v +v(t)\cdot\dd{}t\Big[e_v(u(t),v(t))\Big]\right]                                                         \\
                & = \left[ u'(t)e_u  +u(t)\cdot\Big[u'(t)\cdot \pp {e_u}{u} + v'(t)\cdot \pp{e_u}{v}\Big] \right] + \left[   v'(t)e_v +v(t)\cdot\Big[u'(t)\cdot \pp {e_v}{u} + v'(t)\cdot \pp{e_v}{v}\Big]\right] \\
          %       & = \left[ u'e_u  +uu'\left(
          %   \frac{e_u+e_v}{1+e^{4u}}
          %   -\frac{e_u-e_v}{1+e^{-4u}}
          %   \right)\right]
          % + \left[   v'e_v +vu'\left(
          %   \frac{e_u+e_v}{1+e^{4u}}
          %   +\frac{e_u-e_v}{1+e^{-4u}}
          % \right)\right]                                                                                                                                                                                          \\
                & =  u'e_u  +uu'\left(
          \frac{e_u+e_v}{1+e^{4u}}
          -\frac{e_u-e_v}{1+e^{-4u}}
          \right)
          +    v'e_v +vu'\left(
          \frac{e_u+e_v}{1+e^{4u}}
          +\frac{e_u-e_v}{1+e^{-4u}}
          \right)
          % & = fe_u + ge_u
        \end{align*}

        % \begin{mdframed}
        %   \begin{align*}
        %      & \hphantom{=} \dd{}t\left(\frac{e_u+e_v}{1+e^{4u}} \right)                                                                \\
        %      & =  \frac{\left(u'\pp{e_u}u + v' \pp{e_u}v+u'\pp{e_v}u + v'\pp{e_v}v\right)(1+e^{4u}) - (e_u+e_v)(4e^{4u})}{(1+e^{4u})^2} \\
        %      & =  \frac{
        %     \left(
        %     u'\left[ \frac{e_u+e_v}{1+e^{4u}}-\frac{e_u-e_v}{1+e^{-4u}} \right]
        %     + 0+u'\left[\frac{e_u+e_v}{1+e^{4u}} + \frac{e_u-e_v}{1+e^{-4u}}\right]+0
        %     \right)(1+e^{4u}) - (e_u+e_v)(4e^{4u})
        %     }{(1+e^{4u})^2}                                                                                                             \\
        %      & =  \frac{
        %     2u'
        %     (e_u+e_v)
        %     - (e_u+e_v)(4e^{4u})
        %     }{(1+e^{4u})^2}                                                                                                             \\
        %      & =  2\frac{(e_u+e_v)\left(u'- 2e^{4u}\right)}{(1+e^{4u})^2}                                                               \\
        %   \end{align*}
        %   \begin{align*}
        %      & \hphantom{=} \dd{}t\left( \frac{e_u-e_v}{1+e^{-4u}} \right)                                                                                                                                                               \\
        %      & =  \frac{\left(u'\pp{e_u}u  + v'\pp{e_v}v - u'\pp{e_v}u - v'\pp{e_v}v\right)(1+e^{-4u}) -(e_u-e_v)(-4e^{-4u})}{(1+e^{-4u})^2}                                                                                             \\
        %      & =  \frac{\left( u'\left[ \frac{e_u+e_v}{1+e^{4u}}-\frac{e_u-e_v}{1+e^{-4u}} \right]  +0 - u'\left[\frac{e_u+e_v}{1+e^{4u}} + \frac{e_u-e_v}{1+e^{-4u}}\right] - 0\right)(1+e^{-4u}) -(e_u-e_v)(-4e^{-4u})}{(1+e^{-4u})^2} \\
        %      & =  \frac{-2u'(e_u-e_v) -(e_u-e_v)(-4e^{-4u})}{(1+e^{-4u})^2}                                                                                                                                                              \\
        %      & =  \frac{(e_u-e_v) \left( -2u' + 4e^{-4u} \right)}{(1+e^{-4u})^2}                                                                                                                                                         \\
        %   \end{align*}
        % \end{mdframed}

        % Where:
        % \begin{align*}
        %   f & = u'
        %   + uu'\left( \frac{1}{1+e^{4u}} - \frac{1}{1+e^{-4u}} \right)
        %   + vu'\left(  \frac{1}{1+e^{4u}} + \frac{1}{1+e^{-4u}} \right)
        %   \\
        %     & = u' + \frac{uu' + vu'}{1+e^{4u}} + \frac{vu' - uu'}{1+e^{-4u}}
        %   \\
        %   g & = v'
        %   +  uu'\left( \frac{1}{1+e^{4u}} + \frac{1}{1+e^{-4u}} \right)
        %   + vu'\left(  \frac{1}{1+e^{4u}} - \frac{1}{1+e^{-4u}} \right)       \\
        %     & = v'+ \frac{uu' + vu'}{1+e^{4u}}+ \frac{ uu'-vu'}{1+e^{-4u}}
        %   \\
        % \end{align*}

        % \begin{mdframed}

        %   Finding $f'$ and $g'$:
        %   \begin{align*}
        %     f' & = \dd{}t \left(
        %     u' + \frac{uu' + vu'}{1+e^{4u}} + \frac{vu' - uu'}{1+e^{-4u}}
        %     \right)                                                                                                                                                                      \\
        %        & =  u'' + \frac{(u'u' + u'' + v'u' + vu'')(1+e^{4u})-4(uu' + vu')(e^{4u})}{(1+e^{4u})^2} + \frac{(v'u' + vu'' - u'u' - u'')(1+e^{-4u})+4(vu' - uu')(e^{-4u})}{1+e^{-4u}}
        %   \end{align*}

        %   \begin{align*}
        %     g' & = \dd{}t \left(
        %     v'+ \frac{uu' + vu'}{1+e^{4u}}+ \frac{ uu'-vu'}{1+e^{-4u}}
        %     \right)
        %   \end{align*}

        % \end{mdframed}



        % \[
        %   \begin{array}{cc}
        %     \begin{align*}
        %       \pp{e_u}u
        %     \end{align*} & \pp{e_v}u   \\
        %     \pp{e_u}v      & \pp{e_v}v
        %   \end{array}
        % \]


\end{enumerate}



\section*{Q3}
\begin{enumerate}[label=(\alph*)]
  \item First we choose $\alpha,\beta$ such for $t\in[0,\frac\pi2]$:
        \begin{align*}
          4 & = (\alpha\cos t)^2 + 2(\beta\sin t)^2  \\
            & = \alpha^2\cos^2 t  + 2\beta^2\sin^2 t \\
        \end{align*}
        See that by choosing $\alpha=2$ and $\beta=\sqrt 2$, the equation is satisfied for all $t\in[0,\frac\pi 2]$:
        \begin{align*}
          4 & = 2^2 \cos^2t + 2(\sqrt 2)^2\sin^2 t \\
            & = 4(\cos^2 t+\sin^2 t)               \\
            & = 4
        \end{align*}
        Thus $\set{r(t): t\in [0,\frac \pi2] }\subseteq \set{(x,y)\in\RR^2 : x^2 + 2y^2 = 4 \land x,y\geq0}$.

        % Now for any $(x,y)$ satisfying $x^2+2y^2=4$, 



        % $\set{x^2+2y^2=4 : \begin{bmatrix}
        %             x \\y
        %           \end{bmatrix}=\begin{bmatrix}
        %             \alpha\cos t \\
        %             \beta \sin t
        %           \end{bmatrix}}$
  \item The mass of the wire will be given by:
        $$m = \int_0^{\frac{\pi}2}\rho(r(t))\|r'(t)\|dt$$
        \begin{align*}
          r'(t)     & = \begin{bmatrix}
                          -\alpha\sin t \\
                          \beta \cos t
                        \end{bmatrix}                           \\
          \|r'(t)\| & = \sqrt{\alpha^2\sin^2t + \beta^2\cos^2 t} \\
                    & =\sqrt{4\sin^2t + 2\cos^2 t}
        \end{align*}
        Thus:

        \begin{align*}
          m & = \int_0^{\frac{\pi}2} \left[ 2\sqrt 2\sin t \cos t  \right]\sqrt{4\sin^2t + 2\cos^2 t}\;dt \\
            & =4 \int_0^{\frac{\pi}2} \left[ \sin t \cos t  \right]\sqrt{\sin^2t + 1 t}\;dt
        \end{align*}
        Now let $u(t)=\sin^2t + 1 t$, hence $\frac{du}{dt} = 2\sin t \cos t$, therefore $dt= \frac{du}{2\sin t \cos t}$. Thus:
        \begin{align*}
          m & = 4 \int_{u(0)}^{u(\frac{\pi}2)} \left[ \sin t \cos t  \right]\sqrt{u}\frac{du}{2\sin t\cos t } \\
            & = 2\int_{u(0)}^{u(\frac{\pi}2)}\sqrt{u}\;du                                                     \\
            & = 2\int_1^2 \sqrt u \; du                                                                       \\
            & = 2\left[  \frac 23 2^{\frac 32} - \frac 23 1^{\frac 32}             \right]                    \\
            & = \frac 43 \left( \sqrt{8} - 1 \right)                                                          \\
        \end{align*}

\end{enumerate}

% When $(a,b,c)$ is on the surface $S$, then the tangent plane at $(a,b,c)$ is:
% $$
%   \frac{\partial f}{ \partial x}(a,b,c)(x-a) +
%   \frac{\partial f}{ \partial y}(a,b,c)(y-b) +
%   \frac{\partial f}{ \partial z}(a,b,c)(z-c)
% $$
\end{document}