
\documentclass[a4paper, 11pt]{article}

% Fonts 
\usepackage{opensans}
\usepackage{amsfonts}
\usepackage{montserrat}
\usepackage{amsmath}

\setmainfont{opensans}
% \usepackage[mathrm=sym]{unicode-math}
\usepackage{lato-math}
% \setmathfont[Path=font/,mathrm=sym]{FiraMath-Regular}
% \setmathfont[Path=font/,mathrm=sym]{LatoMath}

\newfontfamily{\montserrateb}{Montserrat SemiBold}
\newfontfamily{\montserratb}{Montserrat Bold}
\newfontfamily{\montserrat}{Montserrat Regular}
\newfontfamily{\montserratl}{Montserrat Light}
% \DeclareMathAlphabet{\mathcal}{OMS}{cmbrs}{m}{n}

% \usepackage[mathrm=sym]{unicode-math}
% \setmainfont{opensans}
% \setmathfont{Fira Math}

% \newfontfamily{\montserrateb}{Montserrat SemiBold}
% \newfontfamily{\montserratb}{Montserrat Bold}
% \newfontfamily{\montserrat}{Montserrat Regular}
% \newfontfamily{\montserratl}{Montserrat Light}
% \DeclareMathAlphabet{\mathcal}{OMS}{cmbrs}{m}{n}
% \setmathfont{Latin Modern Math}[range={\vdots}]

% \autoref
\usepackage{hyperref}

% Use for [H] option for figures to force in text placement
\usepackage{float}

% Captioning figures
\usepackage{caption}

% Subfigures
\usepackage{subcaption}

% For extending contents beyond margins
\usepackage{scrextend}

% For tables \midrule ect
\usepackage{booktabs}

% Colours
\usepackage[table,xcdraw]{xcolor}
\definecolor{accentcolor}{HTML}{a13640}

% Change label in enumerate 
\usepackage{enumitem}

% Section settings
\usepackage{titlesec}
\titleformat{\section}
{\LARGE\montserrateb}
{\thesection.}{0.5em}{}

\titleformat{\subsection}
{\large\montserratb}
{\thesubsection.}{0.5em}{}

% Adjust document dimensions
\ExecuteOptions{a4paper}
\addtolength{\oddsidemargin}{-3cm}
\addtolength{\evensidemargin}{-3cm}
\addtolength{\topmargin}{-3cm}
\addtolength{\textwidth}{6cm}
\addtolength{\textheight}{4.5cm}
\addtolength{\textheight}{1.5cm}
\addtolength{\headsep}{-0.5cm}
% \addtolength{\footskip}{-1cm}
\parindent0pt
\parskip=4pt



\usepackage{matlab-prettifier}
\usepackage{graphicx}
\usepackage{mdframed}

% Creates coloured title box
\newcommand{\thetop}[5]{
	\begin{addmargin}[\oddsidemargin]{\oddsidemargin}
		\colorbox{#5}{\color{white}
			\hbox to \paperwidth{
				\vbox {
					\begin{center}
						{\large\montserratl #1}\\
						\vspace{4pt}
						{\huge\montserratb #2}\\
						{\montserratb #3}\\
						\vspace{-0.5em}
						\rule{20em}{1pt}

						{\large\montserratl
							#4
						}
					\end{center}
				}
			}
		}
	\end{addmargin}
}

\newcommand{\NN}{\mathbb{N}}
\newcommand{\ZZ}{\mathbb{Z}}
\newcommand{\QQ}{\mathbb{Q}}
\newcommand{\RR}{\mathbb{R}}
\newcommand{\CC}{\mathbb{C}}

\newcommand{\dydt}{\frac{dy}{dt}}
\newcommand{\dxdt}{\frac{dx}{dt}}
\def\set#1{\left\{ #1 \right\}}
\def\gen#1{\left\langle#1 \right\rangle}
\def\eval#1#2{\left\ #1\right|_{#2}}

\def\pp#1#2{\frac{\partial #1}{\partial #2}}
\def\dd#1#2{\frac{\,d#1}{\,d#2}}
\def\abs#1{\left|#1\right|}
\def\conj#1{\overline{#1}}


\usepackage{multicol}
\usepackage{tikz}
\usepackage{pgfplots}
\usetikzlibrary {graphs,graphdrawing} \usegdlibrary {force} 
\usetikzlibrary{graphs.standard}
\usetikzlibrary{positioning, 
                quotes}
\usegdlibrary {circular}

\usepackage{pst-platon}

\usepackage{comment}
\usepackage{cancel}

\begin{comment}
	Q1a: Answered 
	Q1b: Answered (recheck)
	Q1c: Answered
	Q1d: Asnwered
	Q1e: 
	Q1f: 
	Q1g: Answered (recheck)
	Q1h: Answered

	Q2a: Answered
	Q2b: Answered
	Q2c: Answered (recheck)
	
	Q3a: Answered
	Q3b: Answered
	Q3c: Answered

	Q4a: Answered
	Q4b: 
	Q4c: Answered
	Q4d: Answered

	Q5a: Answered
	Q5b: Answered (recheck)

	Q6a: Answered
	Q6b: Answered
	Q6c: Answered
	Q6d: Answered
\end{comment}

\begin{document}
\thetop{Robert Christie}{MATHS 320}{S2 2024}{Assignment 1\\Due: 2-08-2024}{accentcolor}

% \begin{multicols*}{2}
\section*{Q1}
Let $G:=\RR^*$, and define the binary operation $*:G\times G\to G$ by: 
\[
	a*b = \begin{cases}
		ab  & a > 0 \\
		a/b & a < 0
	\end{cases}	
\]
Which is well-defined as $a,b\neq0$ so $ab,a/b\in G$ and all cases are covered. 

\begin{enumerate}[label=(\alph*)]
	\item Checking the group axioms: 
		\begin{itemize}
			\item \textit{Associativity:} For all $a,b,c\in G$: 
			\begin{center}
				\begin{tabular}{cc|ccccccc}
					$a$ & $b$ \\
					\midrule
					$+$ & $+$ & $(a*b)*c$ &$=$& $(a\cdot b)\cdot c$ &$=$& $a\cdot (b\cdot c$) &$=$& $a*(b*c)$ \\ 
					$-$ & $+$ & $(a*b)*c$ &$=$& $(a/b)     /c     $ &$=$& $a/     (b\cdot c$) &$=$& $a*(b*c)$ \\ 
					$+$ & $-$ & $(a*b)*c$ &$=$& $(a\cdot b)/c     $ &$=$& $a\cdot (b/c)    $  &$=$& $a*(b*c)$ \\
					$-$ & $-$ & $(a*b)*c$ &$=$& $(a/b)     \cdot c$ &$=$& $a/     (b/c)    $  &$=$& $a*(b*c)$  
				\end{tabular}
			\end{center}
			Thus, all cases are covered and $(G,*)$ satisfies the associative property.
			
			\item \textit{Identity:} Claim $e=1$ is an identity element. For any $a\in G$: 
			Since $1>0$, we have $1*a=1a=a$. Now consider: 
			\[
				a*e 
				= \begin{cases}
					a\cdot 1  & a > 0 \\
					a/1       & a < 0 
				\end{cases}	\quad \implies a*e = a
			\]  
			Thus $a*e = a = e*a$ so $e=1$ is an identity.

			\item \textit{Inverse:} Claim: For any $a\in G$, the inverse is given by: 
			\[
				b = \begin{cases}
					1/a & a>0 \\
					a   & a<0
				\end{cases}	
			\]
			Note that $a>0\iff b>0$ therefore: 
			\begin{itemize}
				\item 
				Case: $a,b>0$, then:
				\begin{alignat*}{2}
					a*b &= a\cdot (1/a) & = 1 = e\\
					b*a &= (1/a)\cdot a & = 1 = e
				\end{alignat*}
				
				\item
				Case: $a,b<0$, then $a=b$ so:
				$$
					b*a = a*b = a*a = a/a = 1 = e 
				$$
			\end{itemize}

			Thus, we have show $b$ is an inverse of $a$ in $(G,*)$.
		\end{itemize}
	\item 
	Checking the \textit{Two-step Test}: First note $\QQ^*\neq \emptyset$ and $\QQ\subset\RR^*$. Thus, $H=(\QQ^*,*)$ is a subgroup of $G$ if and only if for all $a,b\in H$ have $ab\in H$ and $b^{-1}\in H$.  
	
	For any $x\in \QQ^+$, we can write $x=\frac pq$ where $p,q\in\ZZ^+$. Thus, for any $a,b\in\QQ^*$ for some $\alpha,\beta,\gamma,\delta \in \ZZ^+$: 

	\begin{align*}
		a > 0 \implies & a*b 
		= \frac \alpha\beta * \frac \gamma\delta 
		= \frac \alpha\beta \cdot \frac \gamma\delta
		= \frac {\alpha\gamma}{\beta\delta}\in\QQ^* \\
		a < 0 \implies & a*b 
		= \frac \alpha\beta * \frac \gamma\delta 
		= \frac \alpha\beta / \frac \gamma\delta  
		= \frac {\alpha\delta}{\beta\gamma} \in\QQ^*  
	\end{align*}		
	
	Thus, $a*b\in H$. Take any $a\in H$, for some $\alpha,\beta\in\ZZ^*$ we have $a=\alpha/\beta$ so:
	\[
		 a^{-1}\in\set{
			\frac\alpha\beta,
			\frac\beta\alpha
		}\subset H 
	\]

	Thus, the \textit{Two-Step Test} is satisfied so $H=(\QQ^*,*)$ is a subgroup of $G$. 

	\item 
	Consider the element $2\in\ZZ^*$, if $(\ZZ^*, *)$ were a subgroup (and thus a group), then $2$ should have an inverse $b\in\ZZ^*$ such that $1=2*b=2b$, no such element exists in $\ZZ^*$, therefore $\ZZ^*$ is not a subgroup.  

	\item 
	Consider the elements $-1,2 \in G$, then: 
	\[
		(-1) * 2 = -1/2  \quad \land \quad 2 * (-1) = 2 \cdot (-1) = -2 	
	\]
	Yet $-1/2 \neq -2$ so $*$ is not commutative, hence $(G,*)$ is not an abelian group. 

	\item 
	Let $x=5\in G$ \textit{Theorem 6 of Week 2's Notes}, $\gen x=\set{x^i: i\in\ZZ}$. For clarity, let $g\uparrow n$ denote the product of $n\in\NN$ copies of $g\in G$, let $g\uparrow -n$ denote $n$ copies of $g^{-1}$ and let $g\uparrow 0=1$. This is done to avoid confusion with the operation $g^n$ exponentiation in $\RR$. 
	
	
	We claim that $5\uparrow n = 5^{n}$. 
	Base case $n=0$, $5\uparrow 0 = 1 = 5^0$. Now: 
	
	Positive induction case: Assume that $5\uparrow k=5^k$, then: 
	\[
		5\uparrow (k+1)=(5\uparrow k)*5 = (5\uparrow k)\cdot 5 = 5^{k+1}	
	\]
	
	Negative induction case: Assume that $5\uparrow -k=5^{-k}$, then: 
	\[
		5\uparrow (-k-1)=(5\uparrow -k)*5^{-1} = (5\uparrow -k)\cdot \frac 15 = 5^{-k-1}	
	\]
	
	Thus, for all $k\in\ZZ$: 
	\[
		5\uparrow k = 5^k 	
	\]
	Hence: 
	\[
		\gen 5 = \set{5^k : k\in\ZZ}	
	\]

	\item 
	Since $-5<0$, $(-5)^2=-5/-5=1$ so for any $k\in\ZZ$:  
	\[
		-5\uparrow 2k = (-5\uparrow2)\uparrow k = 1\uparrow k = 1 
	\]
	For the odd case: 
	\[
		-5\uparrow (2k+1) = (-5\uparrow 2k)*-5 = -5
	\]
	Hence: 
	\[
		-5\uparrow k = \begin{cases}
			 1 & k \equiv_2 0 \\ 
			-5 & k \equiv_2 1
		\end{cases}	
	\]
	Therefore: 
	\[
		\gen{-5}=\set{1,-5}	
	\]
	
	
	\item 
	We want elements where $x^1\neq 1$ but $x^2=1$, for any $x>0$ where $x\neq 1$ we have $x\star x = xx \neq 1$.

	For $x<0$, we have $x^2=x/x=1$ so all $x\in\RR^*$ where $x<0$ are of order $2$. 
	
	\item 
	Take any $g\in\RR^*$, we check for potential elements of $Z(G)$. 

	\begin{itemize}
		\item For $g>0$ let $z<0$, then:
		\begin{alignat*}{2}
			&          & z*g &= g*z   \\
			& \iff     & z/g &= gz    \\ 
			& \iff     & 1   &= g^2  \\ 
		\end{alignat*}
		Which holds only for $g=1$, since the only $g>0$ with $|g|\leq 2$ is $g=1$. 

		\item For $g<0$ let $z>0$, then:
		\begin{alignat*}{2}
			&          & z*g &= g*z \\
			& \iff     & zg  &= g/z \\ 
			& \iff     & z^2 &= 1   \\ 
		\end{alignat*}
		Which does not hold for all $z$, therefore, this does not hold for any $g<0$. 
	\end{itemize}
	Verifying the $g=1$ case, take any $z\in\RR^*$, then $zg = z = gz$, so $g=1\in Z(G)$. 

\end{enumerate}

\pagebreak
\section*{Q2}
\begin{enumerate}[label=(\alph*)]
	\item 
	We check the conditions for the \textit{Two-step Test}:

	First $1_G=x^0y^0\in H$, so $H$ is non-empty.

	To show $\forall_{\alpha,\beta\in H}\alpha\beta\in H$:
	\begin{mdframed}
		For any $\alpha,\beta\in H$ there are $n,m\in\NN$ with $a,b\in\ZZ^n$ and $a',b'\in\ZZ^m$ such that for some $x,y\in G$ we have: 
		\[
			\alpha = x^{a_1}y^{b_1}\dots x^{a_n}y^{b_n},\quad
			\beta  = x^{a'_1}y^{b'_1}\dots x^{a'_m}y^{b'_m}	
		\]
		Therefore: 
		\[
			\alpha\beta = x^{a_1}y^{b_1}\dots x^{a_n}y^{b_n}x^{a'_1}y^{b'_1}\dots x^{a'_m}y^{b'_m}\in H	
		\]
		Since $n+m\in\NN$ and $(a_1,\dots,a_n,a'_1,\dots,a'_m),(b_1,\dots,b_n,b'_1,\dots,b'_m)\in\ZZ^{n+m}$.
	\end{mdframed}

	To show $\forall_{\alpha\in H}\alpha^{-1}\in H$:
	\begin{mdframed}
		For any $\alpha\in H$ there are $n\in\NN$ and $a,b\in\ZZ^n$ such that for some $x,y\in G$ we have: 
		\[
			\alpha = x^{a_1}y^{b_1}\dots x^{a_n}y^{b_n}
		\]
		Now define $a',b'\in\ZZ^{n+1}$ where: 
		\[
			a'_i = \begin{cases}
				0          & i=1  \\ 
				-a_{n+2-i} & i > 1
			\end{cases}
			\qquad b'_i = \begin{cases}
				-b_{n+1-i} & i < n + 1 \\
				0          & i = n + 1   
			\end{cases}
		\]
		So there is a $\beta\in H$ where: 
		\[
			\beta = x^{a'_1}y^{b'_1}\dots x^{a'_{n+1}}y^{b'_{n+1}}	
		\]
		And so we have:


		\begin{minipage}{0.5\linewidth}
			\begin{align*}
				\alpha \beta 
				&= x^{a_1}y^{b_1}\dots x^{a_n}y^{b_n}
				x^{a'_1}y^{b'_1}\dots x^{a'_{n+1}}y^{b'_{n+1}}	\\
				&= x^{a_1}y^{b_1}\dots x^{a_n}y^{b_n}
				ey^{-b_n}x^{-a_n}\dots y^{-b_1}x^{-a_1}e	\\
				&= x^{a_1}y^{b_1}\dots x^{a_n}y^{b_n}
				y^{-b_n}x^{-a_n}\dots y^{-b_1}x^{-a_1}	\\
				&= e 
			\end{align*}
		\end{minipage}
		\begin{minipage}{0.5\linewidth}
			\begin{align*}
				\beta\alpha 
				&= x^{a'_1}y^{b'_1}\dots x^{a'_{n+1}}y^{b'_{n+1}}	
				x^{a_1}y^{b_1}\dots x^{a_n}y^{b_n} \\
				&= ey^{-b_n}x^{-a_n}\dots y^{-b_1}x^{-a_1}e	
				x^{a_1}y^{b_1}\dots x^{a_n}y^{b_n} \\
				&= y^{-b_n}x^{-a_n}\dots y^{-b_1}x^{-a_1}	
				x^{a_1}y^{b_1}\dots x^{a_n}y^{b_n} \\
				&= e 
			\end{align*}
		\end{minipage}
		As consecutive terms cancel from the middle, hence $a^{_1}=\beta \in H$.
	\end{mdframed}		
	So by the \textit{Two-step Test}, we have show $H$ is a subgroup of $G$. 

	\item Using the same construction from \textit{Part a}, construct $H'$ using group $(L,\cdot)$ as $x,y\in L$:
	\[
		H' = \set{x^{a_1}y^{b_1} \dots x^{a_n}y^{a_n},a_i,b_j\in\ZZ, n\in\NN}	\leq L 
	\]
	
	But $H'=H$ so $H\leq L$. Now by definition notice that: 
	\[
		\gen{x,y} = \bigcap_{\set{x,y}\subset L \leq G}L	
	\]
	Since $\gen{x,y}\leq G$, we have that $H\leq \gen{x,y}$, however, $\gen{x,y}$ is the smallest subgroup of $G$ containing $\set{x,y}$ so we must have $H=\gen{x,y}$. 

	\item 
	First let $n\in\ZZ$ with $n\geq 0$ for $a,b\in{-1,1}$, $a/b=ab$ so $x^n$ in $G$ is $(-1)^k$ in $\ZZ$. IE: 
	\[
		x^n = \begin{cases}
			\hphantom{-}
			 1 & n \equiv_2 0  \\
			-1 & n \equiv_2 1 
		\end{cases}	
	\]
	Now consider:
	\[
		y^{2n} = (y^2)^n = (-2/-2)^n = 1	
	\]  
	Therefore 
	\[
		y^{2n+1}=y^{2n}y = 1y = -2
	\]
	\[
		y^n = \begin{cases}
			\hphantom{-}
			 1 & n \equiv_2 0  \\
			-2 & n \equiv_2 1 
		\end{cases}	
	\]
	Clearly for any $a\in\RR^*$ we have $a^{-n}=(a^n)^{-1}$ since $a^{-n}a^{n}=1$. Using the inverse found in \textit{Part 1}: 
	\[
		x^{-n}=x^n	
		,\qquad
		y^{-n}=y^n 	
	\]
	Therefore we have:
	\[
		\set{x^n: n\in\ZZ}=\set{1,-1},\qquad
		\set{y^n: n\in\ZZ}=\set{1,-2}
	\]
	Hence: 
	\begin{align*}
		\gen{x,y} &= \set{x^{a_1}y^{b_1} \dots x^{a_n}y^{a_n},a_i,b_j\in\ZZ, n\in\NN} \\
				  &= \set{g_1\dots g_n :n\in\NN, g_i\in \set{1,-1,-2}} \\
		          &= \set{2^n : n\in\NN}
		\cup\set{-2^n:n\in\NN}\cup\set{1,-1}
	\end{align*}


\end{enumerate}

\pagebreak 
\section*{Q3}
\begin{enumerate}[label=(\alph*)]
	\item We can compute $U(20)=\set{1, 3, 7, 9 11, 13, 17, 19}$. The Caley Table can be constructed by computed for the upper right entries and mirrored as $(U(20),\times_{20})$ is an abelian group. 
	
	\begin{center}
		\begin{tabular}{c|cccccccc}
			$\times_{20}$ 
			& $ 1$ & $ 3$ & $ 7$ & $ 9$ & $11$ & $13$ & $17$ & $19$ \\
			\midrule
			$1 $ & $ 1$ & $ 3$ & $ 7$ & $ 9$ & $11$ & $13$ & $17$ & $19$ \\
			$3 $ & $ 3$ & $ 9$ & $ 1$ & $ 7$ & $13$ & $19$ & $11$ & $17$ \\
			$7 $ & $ 7$ & $ 1$ & $ 9$ & $ 3$ & $17$ & $11$ & $19$ & $13$ \\
			$9 $ & $ 9$ & $ 7$ & $ 3$ & $ 1$ & $19$ & $17$ & $13$ & $11$ \\
			$11$ & $11$ & $13$ & $17$ & $19$ & $ 1$ & $ 3$ & $ 7$ & $ 9$ \\
			$13$ & $13$ & $19$ & $11$ & $17$ & $ 3$ & $ 9$ & $ 1$ & $ 7$ \\
			$17$ & $17$ & $11$ & $19$ & $13$ & $ 7$ & $ 1$ & $ 9$ & $ 3$ \\
			$19$ & $19$ & $17$ & $13$ & $11$ & $ 9$ & $ 7$ & $ 3$ & $ 1$ \\
		\end{tabular}
	\end{center}
	
	\item Let $g$ correspond to the label of a row in the Caley table, then the inverse of $g$ is given by the column $h$ where the cell corresponding to $g\times_{20}h$ contains a $1$. Thus, we can read off the inverses from the Caley table: 
	\begin{center}
		\begin{tabular}{c|cccccccc}
			$g\in U(20)$ & 
			$ 1$ & $ 3$ & $ 7$ & $ 9$ & $11$ & $13$ & $17$ & $19$ \\
			\midrule
			$g^{-1}$ & 
			$ 1$ & $ 7$ & $ 3$ & $ 9$ & $11$ & $17$ & $13$ & $19$ \\
		\end{tabular}
	\end{center}

	\item Since all elements have finite order, we use the Caley table to compute repeated multiplication for each element, and find the lowest exponent where $g^i=1$, thus determining the order of each element:
	\begin{center}
		\begin{tabular}{c|cccccccc}
			$g\in U(20)$ & 
			$ 1$ & $ 3$ & $ 7$ & $ 9$ & $11$ & $13$ & $17$ & $19$ \\
			\midrule
			$|g|$ & 
			$ 1$ & $4$ & $4$ & $2$ & $2$ & $4$ & $4$ & $2$ \\
		\end{tabular}
	\end{center}

	For any $g\in U(20)$, we have seen that $|g|\leq 4$ so $\abs{\gen{g}}\leq 4$, since $\abs{U(20)}=8$, there is no $g\in U(20)$ where $\set{g}$ generates $U(20)$. Therefore $U(20)$ is not a cyclic group. 
\end{enumerate}

\pagebreak
\section*{Q4}
\begin{enumerate}[label=(\alph*)]
	\item Computing the powers $g^n$ for each $g\in\set{a,b,ba}$:
	\begin{alignat*}{7}
	a^1&=\begin{bmatrix}
		1 & 1 \\
		1 & 2 \\
	\end{bmatrix},\quad
	&a^2&=\begin{bmatrix}
		2 & 0 \\
		0 & 2 \\
	\end{bmatrix},\quad
	&a^3&=\begin{bmatrix}
		2 & 2 \\
		2 & 1 \\
	\end{bmatrix},\quad
	&a^4&=\begin{bmatrix}
		1 & 0 \\
		0 & 1 \\
	\end{bmatrix}
	\\
	b^1&=\begin{bmatrix}
		0 & 2 \\
		1 & 0 \\
	\end{bmatrix},\quad
	&b^2&=\begin{bmatrix}
		2 & 0 \\
		0 & 2 \\
	\end{bmatrix},\quad
	&b^3&=\begin{bmatrix}
		0 & 1 \\
		2 & 0 \\
	\end{bmatrix},\quad
	&b^4&=\begin{bmatrix}
		1 & 0 \\
		0 & 1 \\
	\end{bmatrix}
	\\
	(ba)^1&=\begin{bmatrix}
		2 & 1 \\
		1 & 1 \\
	\end{bmatrix},\quad
	&(ba)^2&=\begin{bmatrix}
		2 & 0 \\
		0 & 2 \\
	\end{bmatrix},\quad
	&(ba)^3&=\begin{bmatrix}
		1 & 2 \\
		2 & 2 \\
	\end{bmatrix},\quad
	&(ba)^4&=\begin{bmatrix}
		1 & 0 \\
		0 & 1 \\
	\end{bmatrix}
	\end{alignat*}
	For each $g$, we find that the smallest $n$ with $g^n=1_G$ is $n=4$ so $a,b,ba$ are all order $4$.

	\item From part (a), we have already found $8$ distinct $g$ which must be present in any group containing both $a$ and $b$.
	
	\begin{center}
		\begin{tabular}{|c|c|c|c|c|c|c|c|}
			\toprule
			$1$ & $2$ & $3$ & $4$ & $5$ & $6$ & $7$ & $8$ \\
			\midrule 
			$a$ & 
			$b$ & 
			$(ba)$ & 
			$a^2=b^2=(ba)^2$ & 
			$a^3$ &
			$b^3$ &
			$(ba)^3$ &
			$e=a^4=b^4=(ba)^4$ \\
			\midrule 
			$\begin{bmatrix}
				1 & 1 \\ 
				1 & 2 
			\end{bmatrix}$&
			$\begin{bmatrix}
				0 & 2 \\ 
				1 & 0 
			\end{bmatrix}$&
			$\begin{bmatrix}
				2 & 1 \\ 
				1 & 1 
			\end{bmatrix}$&
			$\begin{bmatrix}
				2 & 0 \\ 
				0 & 2 
			\end{bmatrix}$&
			$\begin{bmatrix}
				2 & 2 \\ 
				2 & 1 
			\end{bmatrix}$&
			$\begin{bmatrix}
				0 & 1 \\ 
				2 & 0 
			\end{bmatrix}$&
			$\begin{bmatrix}
				1 & 2 \\ 
				2 & 2 
			\end{bmatrix}$&
			$\begin{bmatrix}
				1 & 0 \\ 
				0 & 1 
			\end{bmatrix}$
			\\
			\bottomrule
		\end{tabular}
	\end{center}

	Thus, $8\leq|Q|$. First notice that $a^2=(ba)^2=baba$ so $a=bab$. Now observe that: 
	\[
	 	ab = \begin{bmatrix}
			1 & 2 \\ 
			2 & 2
		\end{bmatrix} =(ba)^3	
	\]
	Hence $ab=bababa=ba(bab)a=ba^3$. From \textit{Question 2}, we can write: 
	\[
		\gen{a,b} = \set{
			a^{\alpha_1}b^{\beta_1}\dots 
			a^{\alpha_k}b^{\beta_k} : \alpha_i,\beta_i\in\ZZ, k\in\NN
		}	
	\]
	So using our observation we can replace all $ab$ to $ba^3$ until all $b$ terms come before $a$ terms. Thus, we can transform the generated subgroup: 
	\[
		\gen{a,b} =\set{b^ia^j: i,j\in\ZZ} 	
	\]
	Since $a^{k+4}=a^k$ and $b^{k+4}=b^k$:
	\[
		\gen{a,b}=\set{b^ia^j: i,j\in\ZZ_4} 	
	\]
	Notice that if we have $i,j\in\ZZ_4$ and $j\geq 2$, then since $a^2=b^2$
	\[
		b^ia^j = b^i a^2a^{j-2}	=b^{i+_42}=a^{j-2}
	\]
	\[
		\gen{a,b}=\set{b^ia^j: i\in\ZZ_4,j\in\ZZ_2} 	
	\]
	Which means $|\set{b^ia^j: i\in\ZZ_4,j\in\ZZ_2}|\leq 4\cdot 2 = 8$. Thus, $|Q|=8$. 





	% aa = bb = bba 

	% ab = bababa 
	% ab = aaba 
	% b = aba 
	% 

	% aa = bb = baba 

	% a = bab 
	% b = aba 

	% a^4 = b^4 

	% a 
	% b 
	% ba 
	% baa 



	\item From \textit{Example 3 of Week 2's Notes}, we see that for a field $F$: 
	\[
		Z(\operatorname{GL}_2(F)) = 
		\set{
			\alpha I : \alpha \in F^*
		}
	\] 
	Since we have $F=\ZZ_3$, we have $F^*=\set{1,2}$ and so: 
	\[
		Z(\operatorname{GL}_2(F))=\set{
			\begin{bmatrix}
				1 & 0 \\ 
				0 & 1
			\end{bmatrix},
			\begin{bmatrix}
				2 & 0 \\
				0 & 2 
			\end{bmatrix}
		}
	\]

	\item 
	Consider any $g=\begin{bmatrix}
		a & b\\
		c & d
	\end{bmatrix}\in Q$, then: 
	
	\begin{alignat*}{3}
		&& g\in C_Q(ba) \\
		&\iff&\begin{bmatrix}
			2 & 1 \\
			1 & 1 
		\end{bmatrix}\cdot g 
		&= g\cdot\begin{bmatrix}
			2 & 1 \\
			1 & 1 
		\end{bmatrix}\\
		&\iff & 
		\begin{bmatrix}
			2a + 1c & 2b + 1d \\
			1a + 1c & 1b + 1d 
		\end{bmatrix} &= 
		\begin{bmatrix}
			2a + 1b & 1a+1b \\ 
			2c + 1d & 1c+1d 
		\end{bmatrix} & \quad & \text{Multiplying Matrices}\\
		&\iff & 
		\begin{bmatrix}
			c & b + d \\
			a & b 
		\end{bmatrix} &= 
		\begin{bmatrix}
			b & a \\ 
			c + d & c
		\end{bmatrix} &\quad & \text{Cancelling}\\
		&\iff & 
		\begin{bmatrix}
			0 & b + d \\
			a & 0 
		\end{bmatrix} &= 
		\begin{bmatrix}
			0 & a \\ 
			b + d & 0
		\end{bmatrix} & \quad & \text{As $b=c$}
	\end{alignat*}
	Thus, the elements in $C_Q(ba)$ are exactly those where $a=b+d$ and $c=b$. Since we already know the elements of $Q$, by checking these conditions we find: 

	\[
		C_Q(ba) = \set{
			\begin{bmatrix}
				2 & 1 \\
				1 & 1
			\end{bmatrix},			
			\begin{bmatrix}
				2 & 0 \\
				0 & 2
			\end{bmatrix},
			\begin{bmatrix}
				1 & 2 \\
				2 & 2
			\end{bmatrix},
			\begin{bmatrix}
				1 & 0 \\
				0 & 1
			\end{bmatrix}
		}
	\]

\end{enumerate}

\pagebreak
\section*{Q5}
Let $H=\gen{h}$ be cyclic group of infinite order, $a=h^n\in H$ and $b=h^m\in H$ where $n,m\in\ZZ$. 
\begin{enumerate}[label=(\alph*)]
	\item 
	Consider $\gen{a,b}$, from \textit{Question 2}, we know that:
	\begin{align*}
		\gen{a,b} &= \set{
			a^{\alpha_1}b^{\beta_1}\dots 
			a^{\alpha_k}b^{\beta_k} : \alpha_i,\beta_i\in\ZZ, k\in\NN
		}\\
		&= \set{
			(h^n)^{\alpha_1}(h^m)^{\beta_1}\dots 
			(h^n)^{\alpha_k}(h^m)^{\beta_k} : \alpha_i,\beta_i\in\ZZ, k\in\NN
		}\\
		&= \set{
			(h^{\alpha_1n})(h^{\beta_1m})\dots 
			(h^{\alpha_kn})(h^{\beta_km})
			: \alpha_i,\beta_i\in\ZZ, k\in\NN
		}\\
		&= \set{
			(h^{
				\alpha_1n + \beta_1m 
				+ \dots + 
				\alpha_kn+\beta_km
			})
			: \alpha_i,\beta_i\in\ZZ, k\in\NN
		}\\
		&= \set{
			(h^{
				\alpha n + \beta m 
			})
			: \alpha,\beta\in\ZZ
		}
	\end{align*}
	To show $\alpha n+\beta m$ is equivalent to $k\cdot\gcd(n,m)$ for some $k\in\ZZ$. 
	\begin{mdframed}
		For any $\alpha,\beta\in\ZZ$, let $p=\frac{n}{\gcd(n,m)}\in\ZZ$ and $q=\frac{m}{\gcd(n,m)}\in\ZZ$, then: 
		\[
			\alpha n + \beta m = k\cdot\gcd(n,m) \iff \alpha p + \beta q = k 
		\]
		Hence $(\alpha p + \beta q)\in\ZZ$ so there must exist $k\in\ZZ$ satisfying the equation. 

		Now to show that for any $k\in\ZZ$ we can write $k\cdot\gcd(n,m)$ in the form $\alpha n + \beta m$, for some $\alpha,\beta\in\ZZ$. 

		For any $k\in\ZZ$, let $p=\frac{n}{\gcd(n,m)}$ and $q=\frac{m}{\gcd(n,m)}$, then $p,q$ must be coprime, so there are $a,b\in\ZZ$ such that: 
		\begin{align*}
					& ap+bq=1 \\ 
		\implies 	& k\gcd(n,m)ap + k\gcd(n,m) bq = k\gcd(n,m)\\  
		\implies 	& akn + bkm = k\gcd(n,m)  
		\end{align*}
		So let $\alpha = ak$ and $\beta = bk$ to obtain a solution. 
	\end{mdframed}
	Thus, we have:  
	\[
		\gen{a,b}=
		\set{
			(h^{
				\alpha n + \beta m 
			})
			: \alpha,\beta\in\ZZ
		} = \set { 
			h^{k\cdot\gcd(n,m)} : k \in\ZZ 
		} = \gen{h^{\gcd(n,m)}}
	\]

	\item First see that: 
	\begin{align*}
		\gen{a}\cap\gen{b} 
		&= \set{a^i:i\in\ZZ}\cap\set{b^i:i\in\ZZ}                       & 
		\text{By \textit{Theorem 6 of Week 2's Notes}}   \\ 
		&= \set{h^{in}:i\in\ZZ}\cap\set{h^{im}:i\in\ZZ}                 &   \\ 
		&= \set{h^k:k\in\ZZ, \exists_{i,j\in\ZZ}\left[ in=k=jm \right]} & 
	\end{align*}
	We want to show that all such $k$ are of the form $t\cdot\operatorname{lcm}(n,m)$ for some $t\in\ZZ$. Consider Euclidean Division: 
	
	% First note that by definition $\operatorname{lcm}(n,m)\leq k$ as all $k$ are common multiples of $n,m$. Clearly this holds for $k=\operatorname{lcm}(n,m)$ with $t=1$. 

	% Now consider $\operatorname{lcm}(n,m)<k$, 
	\[
		k = t\cdot\operatorname{lcm}(n,m) + r 	
	\]
	Where $0\leq r< \operatorname{lcm}(n,m)$. Now note that both $n,m$ divide $k$ and $t\cdot\operatorname{lcm}(n,m)$, so they must also divide $r$, since $0\leq r<\operatorname{lcm}(n,m)$, we must have $r=0$.  Hence, for some $t\in\ZZ$:
	\[
		k=t\cdot\operatorname{lcm}(n,m)
	\] 
	Now notice that $\operatorname{lcm}(n,m)$ is divisible by $n,m$ so $t\cdot \operatorname{lcm}(n,m)=in=jm$ for some $i,j\in\ZZ$. Thus: 

	\[
		\gen{a}\cap\gen{b} 
		= \set{h^k:k\in\ZZ, \exists_{i,j\in\ZZ}\left[ in=k=jm \right] }
		= \set{h^{t\cdot\operatorname{lcm}(n,m)}:t\in\ZZ}
		= \gen{h^{\operatorname{lcm}(n,m)}}
	\]

\end{enumerate}

\pagebreak
\section*{Q6}
Note that by \textit{Theorem 2} of the \textit{Week 2 Lecture Notes}, if $G=\gen{a}$ is finite then $|a|=|G|$. 

\begin{enumerate}[label=(\alph*)]
	\item Since $|G|=28$, we have that $|a|=28$. Thus, by \textit{Theorem 4 of Week 3's Notes}: 
	$$|a^{10}|=\frac{28}{\gcd(10,28)}=\frac{28}{2}=14$$
	Hence $H=\gen{a^{10}}$ has $|H|=|a^{10}|=14$. 

	\item Since $G=\gen{a}$ is a finite cyclic group of order $n=28$ generated by $a$, by \textit{Theorem 8 of Week 2's Notes}, the set of all generators of $G$ is:
	\[
		\set{a^k:k\in U(28)}=\set{
			a^{1},
			a^{3},
			a^{5},
			a^{9},
			a^{11},
			a^{13},
			a^{15},
			a^{17},
			a^{19},
			a^{23},
			a^{25},
			a^{27}
		}
	\]

	\item By the \textit{Fundamental Theorem of Cyclic Groups}, since $G=\gen{a}$ and $|G|=28$, there is a unique subgroup of order $k=14$, $K=\gen{a^{\frac{28}{14}}}=\gen{a^2}$. The generators of this subgroup are the elements of order $14$. Applying \textit{Theorem 8 of Week 2's Notes}, these are: 
	\[
		\set{\big(a^2\big)^i : i \in U(14)}
		=\set{
			a^{2}, a^{6}, a^{10}, a^{18}, a^{22}, a^{26}
		}
	\]
	
	\item By \textit{Fundamental Theorem of Cyclic Groups}, each $H$ subgroup of $G=\gen{a}$ must have order $k=|H|$ where $k$ divides $n$, and there is a unique subgroup for each $k\in\NN$ where $k$ divides $n$. Thus, there are subgroups of orders: 
	\[
		k\in\set{1,2,4,7,14}	
	\]
	The fundamental theorem also gives us the generators $a^{\frac nk}$ for a subgroup of order $k$: 

	\begin{center}
		\begin{tabular}{r|ccccc}
			Subgroup& 
			$\gen{a^{28}}$&
			$\gen{a^{14}}$&
			$\gen{a^{7}}$&
			$\gen{a^{4}}$&
			$\gen{a^{2}}$ \\ 
			\midrule
			Order&
			$1$&
			$2$&
			$4$&
			$7$&
			$14$
		\end{tabular}
	\end{center}
\end{enumerate}
\end{document}

