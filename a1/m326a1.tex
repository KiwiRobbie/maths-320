
\documentclass[a4paper, 11pt]{article}

% Fonts 
\usepackage{opensans}
\usepackage{amsfonts}
\usepackage{montserrat}
\usepackage{amsmath}

\usepackage[mathrm=sym]{unicode-math}

\setmainfont{opensans}
\setmathfont{Fira Math}

\newfontfamily{\montserrateb}{Montserrat SemiBold}
\newfontfamily{\montserratb}{Montserrat Bold}
\newfontfamily{\montserrat}{Montserrat Regular}
\newfontfamily{\montserratl}{Montserrat Light}
\DeclareMathAlphabet{\mathcal}{OMS}{cmbrs}{m}{n}

% \autoref
\usepackage{hyperref}

% Use for [H] option for figures to force in text placement
\usepackage{float}

% Captioning figures
\usepackage{caption}

% Subfigures
\usepackage{subcaption}

% For extending contents beyond margins
\usepackage{scrextend}

% For tables \midrule ect
\usepackage{booktabs}

% Colours
\usepackage[table,xcdraw]{xcolor}
\definecolor{accentcolor}{HTML}{a13640}

% Change label in enumerate 
\usepackage{enumitem}

% Section settings
\usepackage{titlesec}
\titleformat{\section}
{\LARGE\montserrateb}
{\thesection.}{0.5em}{}

\titleformat{\subsection}
{\large\montserratb}
{\thesubsection.}{0.5em}{}

% Adjust document dimensions
\ExecuteOptions{a4paper}
\addtolength{\oddsidemargin}{-3cm}
\addtolength{\evensidemargin}{-3cm}
\addtolength{\topmargin}{-3cm}
\addtolength{\textwidth}{6cm}
\addtolength{\textheight}{4.5cm}
\addtolength{\textheight}{1.5cm}
\addtolength{\headsep}{-0.5cm}
% \addtolength{\footskip}{-1cm}
\parindent0pt
\parskip=4pt



\usepackage{matlab-prettifier}
\usepackage{graphicx}
\usepackage{mdframed}

% Creates coloured title box
\newcommand{\thetop}[5]{
	\begin{addmargin}[\oddsidemargin]{\oddsidemargin}
		\colorbox{#5}{\color{white}
			\hbox to \paperwidth{
				\vbox {
					\begin{center}
						{\large\montserratl #1}\\
						\vspace{4pt}
						{\huge\montserratb #2}\\
						{\montserratb #3}\\
						\vspace{-0.5em}
						\rule{20em}{1pt}

						{\large\montserratl
							#4
						}
					\end{center}
				}
			}
		}
	\end{addmargin}
}

\newcommand{\NN}{\mathbb{N}}
\newcommand{\ZZ}{\mathbb{Z}}
\newcommand{\RR}{\mathbb{R}}
\newcommand{\CC}{\mathbb{C}}
\newcommand{\dydt}{\frac{dy}{dt}}
\newcommand{\dxdt}{\frac{dx}{dt}}
\def\set#1{\left\{ #1 \right\}}
\def\eval#1#2{\left\ #1\right|_{#2}}

\def\pp#1#2{\frac{\partial #1}{\partial #2}}
\def\dd#1#2{\frac{\,d#1}{\,d#2}}
\def\abs#1{\left|#1\right|}
\def\conj#1{\overline{#1}}

\usepackage{multicol}
\usepackage{tikz}
\usepackage{pgfplots}
\usetikzlibrary {graphs,graphdrawing} \usegdlibrary {force} 
\usetikzlibrary{graphs.standard}
\usegdlibrary {circular}

\begin{document}
\thetop{Robert Christie}{MATHS 326}{S2 2023}{Assignment 4\\Due: 20-03-2024}{accentcolor}

\begin{multicols*}{2}
	\section*{Q1}
	\begin{enumerate}[label=(\alph*)]
		\item Call the graph $G$, and apply deletion contraction theorem to the edge $e$ in red:

		      \begin{center}
			      \tikz \graph [empty nodes, nodes={circle, draw}, spring layout] { a --[red] b -- c -- a -- d -- c -- e -- f -- d };
		      \end{center}

		      Thus, $P_G(x) = P_{G-e}(x) - P_{G_e}(x)$  This produces two graphs, $G-e$ and $_e$:
		      \begin{center}
			      \begin{tabular}{cc}
				      \tikz \graph [empty nodes, nodes={circle, draw}, spring layout] {  b -- c -- a -- d -- c -- e -- f -- d };
				              & \tikz \graph [empty nodes, nodes={circle, draw}, spring layout] {  b -- c -- b -- d -- c -- e -- f -- d };
				      \\
				      $G - e$ & $G_e$
			      \end{tabular}
		      \end{center}

		      Notice that $G-e$ can be constructed by adding a vertex to $H=G_e$. Applying deletion contraction to $H$:
		      \begin{center}
			      \begin{tabular}{cc}
				      \tikz \graph [empty nodes, nodes={circle, draw}, spring layout] {  a -- b -- c -- d -- a -- e};
				            & \tikz \graph [empty nodes, nodes={circle, draw}, spring layout] {  a -- b -- c -- d -- a};
				      \\
				      $H-e$ & $H_e$
			      \end{tabular}
		      \end{center}
		      We see that $H_e = C_4$, and $H-e$ is $C_4$ with an additional vertex $v$ added, which can take any color except that of its neighbor. Thus:
		      \begin{align*}
			      P_{H-e}(x) & = (x-1)C_4(x) \\
			      P_{H_e}(x) & = C_4(x)
		      \end{align*}
		      Using this with the deletion contraction gives the chromatic polynomial of $H$:
		      \begin{align*}
			      P_H(x) & = P_{H-e}(x) - P_{H_e}(x) \\
			             & = (x-2)C_4(x)             \\
			             & = (x-2)((x-1)^4+x-1)
		      \end{align*}

		      Since $G-e$, is just $H$ with an additional vertex $v$ with a single neighbor, we find:
		      \begin{align*}
			      P_{G-e}(x)=(x-1)H
		      \end{align*}

		      From deletion contraction we can use our results to determine :
		      \begin{align*}
			      P_G(x) & = P_{G-e}(x) - P_{G_e}(x) \\
			             & = (x-1)H - H              \\
			             & = (x-2)H                  \\
			             & = (x-2)^2((x-1)^4+x-1)
		      \end{align*}
		\item We apply deletion contraction theorem to $G$ on the edge in red:
		      \begin{center}
			      \tikz \graph [empty nodes, nodes={circle, draw}, spring layout] { a -- z -- b -- c -- a -- d -- c -- e --[red] f -- d };
		      \end{center}
		      This produces graphs:
		      \begin{center}
			      \begin{tabular}{cc}
				      \tikz \graph [empty nodes, nodes={circle, draw}, spring layout] { z -- b -- c -- a -- d -- c -- e -- f -- d };
				              & \tikz \graph [empty nodes, nodes={circle, draw}, spring layout] { a -- b -- c -- a -- d -- c -- e -- f -- d };
				      \\
				      $G - e$ & $G_e$
			      \end{tabular}
		      \end{center}
		      Notice that $G_e$ is the graph from part a, while $G-e$ can be constructed by adding two vertices to the graph $H$ from part a. Notice that the added vertices have only one neighbor each meaning they can take on $(x-1)$ colors for each existing coloring of $H$. Thus, the chromatic polynomials are:
		      \begin{align*}
			      P_{G-e}    & = (x-1)^2 P_H(x)            \\
			                 & = (x-1)^2(x-2)((x-1)^4+x-1) \\
			      P_{G_e}(x) & = P_{G_a}(x)                \\
			                 & = (x-2)^2((x-1)^4+x-1)
		      \end{align*}
		      Finally we use these results in the deletion contraction theorem applied to $G$:
		      \begin{align*}
			      P_G(x) & = (x-1)^2(x-2)((x-1)^4+x-1)
			      \\
			             & - (x-2)^2 ( (x-1)^4 + (x-1))                            \\
			             & = \left((x-1)^2-x+2\right)(x-2)\left((x-1)^4+x-1\right)
		      \end{align*}
	\end{enumerate}

	\subsection*{Q2}

	\begin{enumerate}[label=(\alph*)]
		\item

		      First we show that $G$ is connected implies $P_G(x)$ contains a non-zero $x$ term. Apply induction on the number of vertices $n$ of a connected graph $G$.
		      \begin{itemize}
			      \item Base case: $n=1$, so $P_G(x)=x$.
			      \item Induction step: Apply induction on the number of edges $m$:
			            \begin{itemize}
				            \item
				                  Base case: Spanning tree of $G$, so $P_G(x)=x(x-1)^{n-1}$.

				            \item
				                  Induction step: Consider some $G$ with $n+1$ vertices and $m+1$ edges, apply deletion contraction to some edge $e$. As $G-e$ has $m$ vertices it contains a non-zero $x$ term by the induction hypothesis for the edge induction. The graph $G_e$ will have $n$ vertices and has a non-zero $x$ term by the induction hypothesis for induction on the vertices.

				                  Since $P_{G-e}(x), P_{G_e}(x)$ have degree $n+1,n$ and signs alternate as they are chromatic polynomials, the $x$ terms must have different signs. Therefore:
				                  $$P_G(x)=P_{G-e}(x) - P_{G_e}(x)$$
				                  Will also have a non-zero $x$ term as the $x$ term in $P_{G-e}(x)$ and  $-P_{G_e}(x)$ now have the same sign.
			            \end{itemize}
		      \end{itemize}


		      Now we prove the other direction, that is, if $P_G(x)$ contains a non-zero $x$ term, then $G$ is connected:

		      A disconnected $G$ consists of multiple connected subgraphs $C_1,\dots,C_k$, each subgraph can be colored independently, thus:
		      $$P_G(x)=\prod_{i=1}^{k}P_{C_i}(x)$$
		      Since each $C_i$ is connected, by the first implication, $P_{C_i}(x)$ has a non-zero $x$. As $P_{C_i}(x)$ is a chromatic polynomial, the constant term must be zero. Thus, when we take the product of $P_{C_i}(x)$ the lowest degree term is $x^k$ and so $P_G(x)$ has a coefficient of zero for the $x$ term.

		\item
	\end{enumerate}

	\subsection*{Q3}
	\begin{enumerate}[label=(\alph*)]
		\item Expanding $(x-1)^4$ would result in a polynomial with constant term of $(-1)^4=1$, thus the polynomial is not the chromatic polynomial for any graph as it has a non-zero constant term.

		\item Notice that the $x^3$ term is positive but the $x$ term is negative, thus the signs do not alternate otherwise the terms would have the same sign.

		\item Assume the polynomial is a chromatic polynomial $P_G(x)$. We can deduce the following:
		      \begin{itemize}
			      \item Since it has degree $4$, it corresponds to a graph of $4$ vertices.
			      \item The $x^{n-1}=x^3$ term has a coefficient of $-3$, so $G$ must have $3$ edges.
			      \item The $x$ term has a non coefficient, thus $G$ is connected.
		      \end{itemize}
		      The only connected graph with $3$ edges and $4$ vertices is $P_4$, which has chromatic polynomial:
		      \[
			      P_{P_4}(x)=x(x-1)^3= x^4 - 3x^3 + 3x^2 - x \neq P_G(x)
		      \]
		      Thus $P_G(x)$ is not a chromatic polynomial of any graph.


	\end{enumerate}

	\subsection*{Q5}
	\begin{enumerate}
		\item We can create a $3$-coloring of the Petersen graph:
		      \begin{center}
			      \begin{tikzpicture}[every node/.style={draw,circle}]
				      \begin{scope}[shift={(0,-1cm)}]
					      \graph[empty nodes, simple necklace layout, clockwise, radius=1cm] {
						      A[green], B[red], C[red], D[blue], E[green];
						      A -- C -- E -- B -- D -- A;
					      };
				      \end{scope}
				      \graph[empty nodes, simple necklace layout, clockwise, radius=2cm] {
					      1[red] -- 2[blue] -- 3[green] -- 4[red] -- 5[blue] -- 1;
				      };
				      \draw (1) -- (A);
				      \draw (2) -- (B);
				      \draw (3) -- (C);
				      \draw (4) -- (D);
				      \draw (5) -- (E);
			      \end{tikzpicture}
		      \end{center}
		      Therefore, $\chi(P)\leq 3$, however, since $P$ contains an odd cycle of length $5$, we also have $\chi(P)\geq3$, hence $\chi(P)=3$.

		\item A $4$-edge-coloring of the Petersen graph,
		      \begin{center}
			      \begin{tikzpicture}[every node/.style={draw,circle}]
				      \begin{scope}[shift={(0,-1cm)}]
					      \graph[empty nodes, simple necklace layout, clockwise, radius=1cm] {
					      A, B, C, D, E;
					      A --[red] C --[green] E --[red] B --[green] D --[blue] A;
					      };
				      \end{scope}
				      \graph[empty nodes, simple necklace layout, clockwise, radius=2cm] {
				      1--[red]2--[green]3--[red]4--[green]5--[yellow]1;
				      };
				      \draw[green] (1) -- (A);
				      \draw[blue]  (2) -- (B);
				      \draw[blue]  (3) -- (C);
				      \draw[blue]  (4) -- (D);
				      \draw[blue]  (5) -- (E);
			      \end{tikzpicture}
		      \end{center}
	\end{enumerate}
\end{multicols*}
\end{document}
