
\documentclass[a4paper, 11pt]{article}

% Fonts 
\usepackage{opensans}
\usepackage{amsfonts}
\usepackage{montserrat}
\usepackage{amsmath}

\usepackage[mathrm=sym]{unicode-math}

\setmainfont{opensans}
\setmathfont{Fira Math}

\newfontfamily{\montserrateb}{Montserrat SemiBold}
\newfontfamily{\montserratb}{Montserrat Bold}
\newfontfamily{\montserrat}{Montserrat Regular}
\newfontfamily{\montserratl}{Montserrat Light}
\DeclareMathAlphabet{\mathcal}{OMS}{cmbrs}{m}{n}
\setmathfont{Latin Modern Math}[range={\vdots}]

% \autoref
\usepackage{hyperref}

% Use for [H] option for figures to force in text placement
\usepackage{float}

% Captioning figures
\usepackage{caption}

% Subfigures
\usepackage{subcaption}

% For extending contents beyond margins
\usepackage{scrextend}

% For tables \midrule ect
\usepackage{booktabs}

% Colours
\usepackage[table,xcdraw]{xcolor}
\definecolor{accentcolor}{HTML}{a13640}

% Change label in enumerate 
\usepackage{enumitem}

% Section settings
\usepackage{titlesec}
\titleformat{\section}
{\LARGE\montserrateb}
{\thesection.}{0.5em}{}

\titleformat{\subsection}
{\large\montserratb}
{\thesubsection.}{0.5em}{}

% Adjust document dimensions
\ExecuteOptions{a4paper}
\addtolength{\oddsidemargin}{-3cm}
\addtolength{\evensidemargin}{-3cm}
\addtolength{\topmargin}{-3cm}
\addtolength{\textwidth}{6cm}
\addtolength{\textheight}{4.5cm}
\addtolength{\textheight}{1.5cm}
\addtolength{\headsep}{-0.5cm}
% \addtolength{\footskip}{-1cm}
\parindent0pt
\parskip=4pt



\usepackage{matlab-prettifier}
\usepackage{graphicx}
\usepackage{mdframed}

% Creates coloured title box
\newcommand{\thetop}[5]{
	\begin{addmargin}[\oddsidemargin]{\oddsidemargin}
		\colorbox{#5}{\color{white}
			\hbox to \paperwidth{
				\vbox {
					\begin{center}
						{\large\montserratl #1}\\
						\vspace{4pt}
						{\huge\montserratb #2}\\
						{\montserratb #3}\\
						\vspace{-0.5em}
						\rule{20em}{1pt}

						{\large\montserratl
							#4
						}
					\end{center}
				}
			}
		}
	\end{addmargin}
}

\newcommand{\NN}{\mathbb{N}}
\newcommand{\ZZ}{\mathbb{Z}}
\newcommand{\RR}{\mathbb{R}}
\newcommand{\CC}{\mathbb{C}}
\newcommand{\dydt}{\frac{dy}{dt}}
\newcommand{\dxdt}{\frac{dx}{dt}}
\def\set#1{\left\{ #1 \right\}}
\def\eval#1#2{\left\ #1\right|_{#2}}

\def\pp#1#2{\frac{\partial #1}{\partial #2}}
\def\dd#1#2{\frac{\,d#1}{\,d#2}}
\def\abs#1{\left|#1\right|}
\def\conj#1{\overline{#1}}

\usepackage{multicol}
\usepackage{tikz}
\usepackage{pgfplots}
\usetikzlibrary {graphs,graphdrawing} \usegdlibrary {force} 
\usetikzlibrary{graphs.standard}
\usegdlibrary {circular}

\usepackage{pst-platon}

\begin{document}
\thetop{Robert Christie}{MATHS 326}{S1 2024}{Assignment 1\\Due: 20-03-2024}{accentcolor}

\begin{multicols*}{2}
	\section*{Q1}
	\begin{enumerate}[label=(\alph*)]
		\item Call the graph $G$, and apply deletion contraction theorem to the edge $e$ in red:

		      \begin{center}
			      \tikz \graph [empty nodes, nodes={circle, draw}, spring layout] { a --[red] b -- c -- a -- d -- c -- e -- f -- d };
		      \end{center}

		      Thus, $P_G(x) = P_{G-e}(x) - P_{G_e}(x)$  This produces two graphs, $G-e$ and $_e$:
		      \begin{center}
			      \begin{tabular}{cc}
				      \tikz \graph [empty nodes, nodes={circle, draw}, spring layout] {  b -- c -- a -- d -- c -- e -- f -- d };
				              & \tikz \graph [empty nodes, nodes={circle, draw}, spring layout] {  b -- c -- b -- d -- c -- e -- f -- d };
				      \\
				      $G - e$ & $G_e$
			      \end{tabular}
		      \end{center}

		      Notice that $G-e$ can be constructed by adding a vertex to $H=G_e$. Applying deletion contraction to $H$:
		      \begin{center}
			      \begin{tabular}{cc}
				      \tikz \graph [empty nodes, nodes={circle, draw}, spring layout] {  a -- b -- c -- d -- a -- e};
				            & \tikz \graph [empty nodes, nodes={circle, draw}, spring layout] {  a -- b -- c -- d -- a};
				      \\
				      $H-e$ & $H_e$
			      \end{tabular}
		      \end{center}
		      We see that $H_e = C_4$, and $H-e$ is $C_4$ with an additional vertex $v$ added, which can take any color except that of its neighbor. Thus:
		      \begin{align*}
			      P_{H-e}(x) & = (x-1)C_4(x) \\
			      P_{H_e}(x) & = C_4(x)
		      \end{align*}
		      Using this with the deletion contraction gives the chromatic polynomial of $H$:
		      \begin{align*}
			      P_H(x) & = P_{H-e}(x) - P_{H_e}(x) \\
			             & = (x-2)C_4(x)             \\
			             & = (x-2)((x-1)^4+x-1)
		      \end{align*}

		      Since $G-e$, is just $H$ with an additional vertex $v$ with a single neighbor, we find:
		      \begin{align*}
			      P_{G-e}(x)=(x-1)H
		      \end{align*}

		      From deletion contraction we can use our results to determine :
		      \begin{align*}
			      P_G(x) & = P_{G-e}(x) - P_{G_e}(x) \\
			             & = (x-1)H - H              \\
			             & = (x-2)H                  \\
			             & = (x-2)^2((x-1)^4+x-1)
		      \end{align*}
		\item We apply deletion contraction theorem to $G$ on the edge in red:
		      \begin{center}
			      \tikz \graph [empty nodes, nodes={circle, draw}, spring layout] { a -- z -- b -- c -- a -- d -- c -- e --[red] f -- d };
		      \end{center}
		      This produces graphs:
		      \begin{center}
			      \begin{tabular}{cc}
				      \tikz \graph [empty nodes, nodes={circle, draw}, spring layout] { z -- b -- c -- a -- d -- c -- e -- f -- d };
				              & \tikz \graph [empty nodes, nodes={circle, draw}, spring layout] { a -- b -- c -- a -- d -- c -- e -- f -- d };
				      \\
				      $G - e$ & $G_e$
			      \end{tabular}
		      \end{center}
		      Notice that $G_e$ is the graph from part a, while $G-e$ can be constructed by adding two vertices to the graph $H$ from part a. Notice that the added vertices have only one neighbor each meaning they can take on $(x-1)$ colors for each existing coloring of $H$. Thus, the chromatic polynomials are:
		      \begin{align*}
			      P_{G-e}    & = (x-1)^2 P_H(x)            \\
			                 & = (x-1)^2(x-2)((x-1)^4+x-1) \\
			      P_{G_e}(x) & = P_{G_a}(x)                \\
			                 & = (x-2)^2((x-1)^4+x-1)
		      \end{align*}
		      Finally we use these results in the deletion contraction theorem applied to $G$:
		      \begin{align*}
			      P_G(x) & = (x-1)^2(x-2)((x-1)^4+x-1)
			      \\
			             & - (x-2)^2 ( (x-1)^4 + (x-1))                            \\
			             & = \left((x-1)^2-x+2\right)(x-2)\left((x-1)^4+x-1\right)
		      \end{align*}
	\end{enumerate}

	\subsection*{Q2}

	\begin{enumerate}[label=(\alph*)]
		\item
		      First we show that if $G$ is connected, then $P_G(x)$ contains a non-zero $x$ term:

		      Apply strong induction on $m=\abs{E}$ for connected graphs $G=(V,E)$.
		      \begin{itemize}
			      \item Base case:

			            For $\abs{E}=0$, the only connected graph is a single vertex with $P_G(x)=x$.

			      \item Induction Step (``$m\implies m+1$''):

			            Consider a graph with $\abs{E}=m+1$, apply deletion contraction, thus:
			            $$P_G(x)= P_{G-e}(x) - P_{G_e}(x)$$
			            Note that both $G-e$ and $G_e$ have $m$ or fewer edges and therefore contain an $x$ term in their chromatic polynomials by the induction hypothesis.

			            Note that if $G$ has $n$ vertices, so does $G-e$ while $G_e$ has $n-1$ vertices. Hence, $P_{G-e}(x)$ is of degree $n$ and $P_{G_e}(x)$ degree $n-1$. Since the $x^n$ term has positive coefficient and the signs alternate, the two polynomials have opposite signs on the $x$ term. Applying this observation to the deletion contraction equation:
			            \begin{align*}
				            P_G(x) & = P_{G-e}(x) - P_{G_e}(x)      \\
				                   & = (x^n + \dots \pm c_a x )     \\
				                   & -  (x^{n-1} + \dots \mp c_b x)
				            \quad\text{For $c_a,c_b\in\RR^+$}       \\
				                   & = x^n + \dots \pm (c_a + c_b)x
			            \end{align*}
			            Note that $c_a+c_b > 0$ therefore $P_G(x)$ contains a non-zero $x$ term.

		      \end{itemize}

		\item
		      Now we show that if $P_G(x)$ contains an $x$ term, then $G$ is connected. We prove the contrapositive $G$ is disconnected implies that $P_G(x)$ does not contain a nonzero $x$ term.

		      Consider any disconnected graph $G$ which can be expressed as the disjoint union of $k\geq2$ subgraphs $H_1,\dots,H_k$.

		      By \textit{Lemma 1.7}:
		      $$P_G(x)= \prod_{i=1}^k P_{H_i}(x)$$
		      Since $P_{H_i}(x)$ are chromatic polynomials, they do not contain a constant term. Thus, the product is of terms:

		      \begin{align*}
			       & \begin{matrix}
				          & \hphantom{=} P_G(x) \\
			         \end{matrix}                         \\
			       & \begin{matrix}
				         = & (x^{n_1} + \dots + c_{2,1}x^2 + c_{1,1}x ) \\
				           & \times                                     \\
				           & \vdots                                     \\
				           & \times                                     \\
				           & (x^{n_k} + \dots + c_{2,k}x^2 + c_{1,k}x )
			         \end{matrix} \\
			       & \begin{matrix}
				         = & \text{H.O.T.} + (c_{1,1}\dots c_{1,k})x^k \\
			         \end{matrix}  \\
		      \end{align*}
		      Thus, there is no $x$ term in $P_G(x)$.

		\item Consider the two cases:

		      \begin{itemize}
			      \item
			            $G$ is connected:

			            As $G$ is connected, \textit{Corollary 1.15} still applies and so $G$ is a tree if and only if $P_G(x)=x(x-1)^{n-1}$. Thus, the new statement holds in this case.
			      \item
			            $G$ is disconnected and \textit{Corollary 1.15} cannot be applied:

			            LHS of the equivalence is false as $G$ does not satisfy the definition of a tree.

			            Since $G$ is disconnected, it does not contain an non-zero $x$ term in its chromatic polynomial. However:
			            $$x(x-1)^{n-1}=\dots\pm x$$
			            Does contain a non-zero $x$ term meaning it is not the chromatic polynomial of $G$. Hence the RHS of the equivalence is also false.

			            Since $F\iff F$, the equivalence also holds in this case.

		      \end{itemize}
		      These cases cover all possibilities and so ``connected'' can be omitted from the Corollary.



	\end{enumerate}

	\subsection*{Q3}
	\begin{enumerate}[label=(\alph*)]
		\item Expanding $(x-1)^4$ would result in a polynomial with constant term of $(-1)^4=1$, thus the polynomial is not the chromatic polynomial for any graph as it has a non-zero constant term.

		\item Notice that the $x^3$ term is positive but the $x$ term is negative, thus the signs do not alternate otherwise the terms would have the same sign.

		\item Assume the polynomial is a chromatic polynomial $P_G(x)$. We can deduce the following:
		      \begin{itemize}
			      \item Since it has degree $4$, it corresponds to a graph of $4$ vertices.
			      \item The $x^{n-1}=x^3$ term has a coefficient of $-3$, so $G$ must have $3$ edges.
			      \item The $x$ term has a non coefficient, thus $G$ is connected.
		      \end{itemize}
		      The only connected graph with $3$ edges and $4$ vertices is $P_4$, which has chromatic polynomial:
		      \[
			      P_{P_4}(x)=x(x-1)^3= x^4 - 3x^3 + 3x^2 - x \neq P_G(x)
		      \]
		      Thus $P_G(x)$ is not a chromatic polynomial of any graph.


	\end{enumerate}

	\section*{Q4}
	Consider $n=11$. Thus, $m\leq27$ by theorem 1.19.

	Consider a graph of $n\leq 11$ vertices and $m$ edges. By \textit{Theorem 1.19}, we know that $m\leq3n-6$. Assume the statement does not hold, then all vertices have degree at least $5$. By the handshaking lemma:
	\[
		5n=	5\abs{V} \leq	\sum_{v\in V}\operatorname{deg}(v) = 2\abs{E} = 2m
	\]
	Thus, $2.5n \leq m \leq 3n -6 $. This is only possible for:
	\begin{align*}
		     & 2.5n \leq 3n- 6 \\
		\iff & 6 \leq 0.5n     \\
		\iff & 12 \leq n
	\end{align*}
	Thus, we have a contraction when $n<12$, thus the assumption was false and the original statement was true.

	To show the bound is sharp we look for a graph on $12$ vertices such that $\forall_{v\in V} \deg{v} \geq 5$. Since $2.5n = 18 \leq m \leq 3n-6=18$, we have exactly the number of edges of the lower bound which corresponds to all vertices having degree $5$. Thus, we are looking for a planar $5$ regular graph.

	The regular icosahedron is a Platonic solid where each vertex is adjacent to exactly $5$ edges, this is similar to our graph requirements.
	\begin{center}
		\psIcosahedron[Frame=false,faceName=\hphantom]
	\end{center}
	In the 2D projection above, $3$ vertices corresponding to the back most face, by scaling this face to surround the rest of the projection, we remove all line-line intersections. If we consider the graph of vertices and edges of the icosahedron, this new projection produces a planar embedding:

	\begin{center}
		\begin{tikzpicture}[every node/.style={draw,circle}]
			\begin{scope}[shift={(0,-1cm)}]
				\graph[empty nodes, simple necklace layout, clockwise, radius=3cm] {
					O1 -- O2 -- O3 -- O1;
				};
				\begin{scope}[shift={(0,-1.5cm)}]
					\graph[empty nodes, simple necklace layout, clockwise, radius=1cm] {
						1 -- 2 -- 3 -- 4 -- 5 -- 6 -- 1;
					};
				\end{scope}
				\begin{scope}[shift={(0,-2.5)}, rotate=60]
					\begin{scope}[shift={(0,0.5)}]
						\graph[empty nodes, simple necklace layout, clockwise, radius=0.5cm] {
							I1 -- I2 -- I3 -- I1;

						};
					\end{scope}
				\end{scope}
			\end{scope}
			\draw (O1) -- (6) -- (I3);
			\draw (O1) -- (2) -- (I1);
			\draw (O1) -- (1) -- (I1);
			\draw 				(1) -- (I3);

			\draw (O2) -- (2) -- (I1);
			\draw (O2) -- (4) -- (I2);
			\draw (O2) -- (3) -- (I2);
			\draw 				(3) -- (I1);

			\draw (O3) -- (4) -- (I2);
			\draw (O3) -- (6) -- (I3);
			\draw (O3) -- (5) -- (I3);
			\draw 				(5) -- (I2);
		\end{tikzpicture}
	\end{center}

	\subsection*{Q5}
	\begin{enumerate}[label=(\alph*)]
		\item We can create a $3$-coloring of the Petersen graph:
		      \begin{center}
			      \begin{tikzpicture}[every node/.style={draw,circle}]
				      \begin{scope}[shift={(0,-1cm)}]
					      \graph[empty nodes, simple necklace layout, clockwise, radius=1cm] {
						      A[green], B[red], C[red], D[blue], E[green];
						      A -- C -- E -- B -- D -- A;
					      };
				      \end{scope}
				      \graph[empty nodes, simple necklace layout, clockwise, radius=2cm] {
					      1[red] -- 2[blue] -- 3[green] -- 4[red] -- 5[blue] -- 1;
				      };
				      \draw (1) -- (A);
				      \draw (2) -- (B);
				      \draw (3) -- (C);
				      \draw (4) -- (D);
				      \draw (5) -- (E);
			      \end{tikzpicture}
		      \end{center}
		      Therefore, $\chi(P)\leq 3$, however, since $P$ contains an odd cycle of length $5$, we also have $\chi(P)\geq3$, hence $\chi(P)=3$.


		      We can construct the following proper edge coloring of $P$:
		      \begin{center}
			      \begin{tikzpicture}[every node/.style={draw,circle}]
				      \begin{scope}[shift={(0,-1cm)}]
					      \graph[empty nodes, simple necklace layout, clockwise, radius=1cm] {
					      A, B, C, D, E;
					      A --[red] C --[green] E --[red] B --[green] D --[blue] A;
					      };
				      \end{scope}
				      \graph[empty nodes, simple necklace layout, clockwise, radius=2cm] {
				      1--[red]2--[green]3--[red]4--[green]5--[yellow]1;
				      };
				      \draw[green] (1) -- (A);
				      \draw[blue]  (2) -- (B);
				      \draw[blue]  (3) -- (C);
				      \draw[blue]  (4) -- (D);
				      \draw[blue]  (5) -- (E);
			      \end{tikzpicture}
		      \end{center}
		      Thus $\chi_e(P)\leq 4$. Now consider a three coloring of $P$. We make the following observations:
		      \begin{itemize}
			      \item The cycle $C_5$ has chromatic index $3$.
			      \item If we take a subgraph of three edges where no pair is adjacent, then there a $6$ vertices in the subgraph by definition of edge adjacency.
			      \item Since $C_5$ has only $5$ vertices, we cannot find $3$ adjacent edges such that no pair is adjacent. Thus, a coloring of $C_5$ can use each color at most twice. A $3$ edge-coloring of $C_5$ must use each color at least once as no $2$ edge-coloring exists.
			      \item We must use two colors twice and one color once in order to color all $5$ edges if each color is used either once or twice.
			      \item WLOG, let \textcolor{red}{$r$} be the color used once. The remaining colors $\textcolor{green}{g},\textcolor{blue}{b}$ must alternate on the remaining vertices.

			            \begin{center}
				            \begin{tikzpicture}[every node/.style={draw,circle}]
					            \graph[empty nodes, simple necklace layout, clockwise, radius=1cm] {
					            A --[red] B --[green] C --[blue] D --[green] E --[blue] A;
					            };
				            \end{tikzpicture}
			            \end{center}

			      \item As only three colors are available, when edges are added to produce the following graph, their coloring is determined by the existing edges in the cycle.
			            \begin{center}
				            \begin{tikzpicture}[every node/.style={draw,circle}]
					            \begin{scope}[shift={(0,-1cm)}]
						            \graph[empty nodes, simple necklace layout, clockwise, radius=1cm] {
							            A, B, C, D, E;
						            };
					            \end{scope}
					            \graph[empty nodes, simple necklace layout, clockwise, radius=2cm] {
					            1--[red]2--[blue]3--[green]4--[blue]5--[green]1;
					            };
					            \draw[blue] (1) -- (A);
					            \draw[green]  (2) -- (B);
					            \draw[red]  (3) -- (C);
					            \draw[red]  (4) -- (D);
					            \draw[red]  (5) -- (E);
				            \end{tikzpicture}
			            \end{center}

			      \item Consider the subgraph of $P$ created by adding the two dashed edges from the Petersen graph to the previous subgraph: 			            \begin{center}
				            \begin{tikzpicture}[every node/.style={draw,circle}]
					            \begin{scope}[shift={(0,-1cm)}]
						            \graph[empty nodes, simple necklace layout, clockwise, radius=1cm] {
						            A, B, C, D, E;

						            D --[dashed] A --[dashed] C;
						            };
					            \end{scope}
					            \graph[empty nodes, simple necklace layout, clockwise, radius=2cm] {
					            1--[red]2--[blue]3--[green]4--[blue]5--[green]1;
					            };
					            \draw[blue] (1) -- (A);
					            \draw[green]  (2) -- (B);
					            \draw[red]  (3) -- (C);
					            \draw[red]  (4) -- (D);
					            \draw[red]  (5) -- (E);
				            \end{tikzpicture}
			            \end{center}
			            Both edges are adjacent to $\textcolor{red}{r}$ and $\textcolor{blue}{b}$ edges already, so they must both be $\textcolor{green}{g}$. However, since they are adjacent they cannot receive the same colors resulting in a contraction. Thus $P$ is not $3$ edge-colorable and $\chi_e(P)>3$.
		      \end{itemize}
		      Since $3<\chi_e(P)\leq 4$, the chromatic index of the Petersen graph $\chi_e(P)=4$.

		\item Contract the edges of $P$ show in red below:
		      \begin{center}
			      \begin{tikzpicture}[every node/.style={draw,circle}]
				      \begin{scope}[shift={(0,-1cm)}]
					      \graph[empty nodes, simple necklace layout, clockwise, radius=1cm] {
						      A, B, C, D, E;
						      A -- C -- E -- B -- D -- A;
					      };
				      \end{scope}
				      \graph[empty nodes, simple necklace layout, clockwise, radius=2cm] {
					      1--2--3--4--5--1
				      };
				      \draw[red] (1) -- (A);
				      \draw[red]  (2) -- (B);
				      \draw[red]  (3) -- (C);
				      \draw[red]  (4) -- (D);
				      \draw[red]  (5) -- (E);
			      \end{tikzpicture}
		      \end{center}
		      This produces $K_5$ so by Wagner's theorem, $P$ must be non-planar.
	\end{enumerate}

	\section*{Q6}


	\textcolor{red}{Claim: A subgraph of a bipartite graph is bipartite.}

	Claim: If $H$ is bipartite, then $\alpha(G)\geq \frac 12 m_H$.

	% Proof: Proof by contraction, if $\alpha(H) < m_H$ then the largest independent set in $H$ has less than half the vertices. If we take any bipartite graph $G$ , 


\end{multicols*}
\end{document}
