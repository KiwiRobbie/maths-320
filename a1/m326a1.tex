
\documentclass[a4paper, 11pt]{article}

% Fonts 
\usepackage{opensans}
\usepackage{amsfonts}
\usepackage{montserrat}
\usepackage{amsmath}

\usepackage[mathrm=sym]{unicode-math}

\setmainfont{opensans}
\setmathfont{Fira Math}

\newfontfamily{\montserrateb}{Montserrat SemiBold}
\newfontfamily{\montserratb}{Montserrat Bold}
\newfontfamily{\montserrat}{Montserrat Regular}
\newfontfamily{\montserratl}{Montserrat Light}
\DeclareMathAlphabet{\mathcal}{OMS}{cmbrs}{m}{n}

% \autoref
\usepackage{hyperref}

% Use for [H] option for figures to force in text placement
\usepackage{float}

% Captioning figures
\usepackage{caption}

% Subfigures
\usepackage{subcaption}

% For extending contents beyond margins
\usepackage{scrextend}

% For tables \midrule ect
\usepackage{booktabs}

% Colours
\usepackage[table,xcdraw]{xcolor}
\definecolor{accentcolor}{HTML}{6332a8}

% Change label in enumerate 
\usepackage{enumitem}

% Section settings
\usepackage{titlesec}
\titleformat{\section}
{\LARGE\montserrateb}
{\thesection.}{0.5em}{}

\titleformat{\subsection}
{\large\montserratb}
{\thesubsection.}{0.5em}{}

% Adjust document dimensions
\ExecuteOptions{a4paper}
\addtolength{\oddsidemargin}{-3cm}
\addtolength{\evensidemargin}{-3cm}
\addtolength{\topmargin}{-3cm}
\addtolength{\textwidth}{6cm}
\addtolength{\textheight}{4.5cm}
\addtolength{\textheight}{1.5cm}
\addtolength{\headsep}{-0.5cm}
% \addtolength{\footskip}{-1cm}
\parindent0pt
\parskip=4pt



\usepackage{matlab-prettifier}
\usepackage{graphicx}
\usepackage{mdframed}

% Creates coloured title box
\newcommand{\thetop}[5]{
	\begin{addmargin}[\oddsidemargin]{\oddsidemargin}
		\colorbox{#5}{\color{white}
			\hbox to \paperwidth{
				\vbox {
					\begin{center}
						{\large\montserratl #1}\\
						\vspace{4pt}
						{\huge\montserratb #2}\\
						{\montserratb #3}\\
						\vspace{-0.5em}
						\rule{20em}{1pt}

						{\large\montserratl
							#4
						}
					\end{center}
				}
			}
		}
	\end{addmargin}
}

\newcommand{\NN}{\mathbb{N}}
\newcommand{\ZZ}{\mathbb{Z}}
\newcommand{\RR}{\mathbb{R}}
\newcommand{\CC}{\mathbb{C}}
\newcommand{\dydt}{\frac{dy}{dt}}
\newcommand{\dxdt}{\frac{dx}{dt}}
\def\set#1{\left\{ #1 \right\}}
\def\eval#1#2{\left\ #1\right|_{#2}}

\def\pp#1#2{\frac{\partial #1}{\partial #2}}
\def\dd#1#2{\frac{\,d#1}{\,d#2}}
\def\abs#1{\left|#1\right|}
\def\conj#1{\overline{#1}}

\usepackage{multicol}
\usepackage{pgfplots}

\begin{document}
\thetop{Robert Christie}{MATHS 340}{S2 2023}{Assignment 4\\Due: 17-10-2023}{accentcolor}

\begin{multicols}{2}
	\section*{Q1}
	The function $f$ is differentiable at $z$ the following limit:
	$$ \lim_{\Delta z\to 0}\frac{f(z+\Delta z)-f(z)}{\Delta z} $$
	Exists, consider $\Delta z$ approaching along a straight line with angle $\theta$.
	\begin{align*}
		  & \lim_{r\to 0}\frac{f(z+re^{i\theta})-f(z)}{re^{i\theta}}                                                                                                                                                                                          \\
		= & \lim_{r\to 0}\frac{\frac 1{\overline z+re^{-i\theta}}-\frac1{\overline z}}{re^{i\theta}}                                                                                                                                                          \\
		= & \lim_{r\to 0}\frac{\frac {\overline z}{\left( \overline z+re^{-i\theta} \right)\overline z}-\frac{\left( \overline z+re^{-i\theta} \right)}{\left( \overline z+re^{-i\theta} \right)\overline z}}{re^{i\theta}}\quad\text{as $\overline z\neq 0$} \\
		= & \lim_{r\to 0}\frac {\overline z-\left( \overline z+re^{-i\theta} \right)}{re^{i\theta}\left( \overline z+re^{-i\theta} \right)\overline z}                                                                                                        \\
		= & \lim_{r\to 0}\frac {-re^{-i\theta}}{re^{i\theta}\overline z^2+r^2\overline z}                                                                                                                                                                     \\
		= & \lim_{r\to 0}\frac {-e^{-2i\theta}}{\overline z^2+re^{-i\theta}\overline z}
	\end{align*}
	Notice that this limit will depend on $\theta$ as the denominator:
	$$\overline{z^2}+re^{-i\theta}\overline z\to \overline z ^2$$
	While the numerator depends on $\theta$. Hence, the limit does not exist for any $z\in \CC \backslash \set 0$ meaning the function is differentiable nowhere on $\CC$.


	To verify this, we can write

	\begin{align*}
		f(x+iy) = & \frac{1}{x-iy}
		=\frac{x+iy}{x^2+y^2}                           \\
		=         & \frac x{x^2+y^2}+\frac{iy}{x^2+y^2} \\
		=         & u(x,y)+iv(x,y)
	\end{align*}
	Using the product rule:

	\begin{alignat*}{6}
		\pp{u}{x}  & = \frac{1-2x}{(x^2+y^2)^2}      & \;\;\neq\;\;           &  &
		\pp{v}{y}  & = \frac{1-2y}{(x^2+y^2)^2}\quad & \text{When $x\neq y$}       \\
		\pp{u}{y}  & = \frac{-2y}{(x^2+y^2)^2}       & \;\;\neq\;\;           &  &
		-\pp{v}{x} & = -\frac{-2x}{(x^2+y^2)^2}\quad & \text{When $x\neq -y$}
	\end{alignat*}
	So $f(z)$ does not satisfy the Cauchy-Riemann equations for any $z\in\CC\backslash\set 0$ and is therefore not differentiable anywhere on $z\in\CC\backslash\set 0$.




	\section*{Q2}
	Note that $f$ is analytic except when $z=0$ or $z^3=1$. Of the three solutions to $z^3$ only $z=1$ lies within $C$.

	Since $C$ is a piecewise smooth closed curve oriented anticlockwise, and $f(z)$ is analytic on and inside $C$ except at $2$ isolated points $z=0$ and $z=1$, we can apply the residue theorem:
	$$\int_C f(z)dz=2\pi i (\text{Res}_0f + \text{Res}_1 f)$$

	As $z=0$ is a pole of order $m=3$:
	\begin{align*}
		\text{Res}_0 f & = \frac 1{(m-1)!}\lim_{z\to 0} \left( \dd{^{m-1}}{z^{m-1}} \left[ (z-0)^m f(z) \right] \right)      \\
		               & = \frac 1{2}\lim_{z\to 0} \left( \dd{^2}{z^2} \left[ z^3 \frac{1}{z^3(1-z^3)} \right] \right)       \\
		               & = \frac 1{2}\lim_{z\to 0} \left( \dd{}{z} \left[ 3x^2(1-x^3) \right]\right)                         \\
		               & = \frac 1{2}\lim_{z\to 0} \left( \frac{6z}{(1-z^3)^2} + \frac{18z^4}{\left( 1-z^3 \right)^3}\right) \\
		               & = 0
	\end{align*}

	We have a first order pole at $z=1$ as we can write $(1-z^3)=(1-z)(z^2+z+1)$. Thus, the residue at $z=1$ is:
	\begin{align*}
		\text{Res}_1 f & = \frac 1{(m-1)!}\lim_{z\to 1} \left( \dd{^{m-1}}{z^{m-1}} \left[ (z-1)^m f(z) \right] \right) \\
		               & \lim_{z\to1} \left( (z-1) \frac{1}{z^3(1-z^3)} \right)                                         \\
		               & \lim_{z\to1} \left( (z-1) \frac{1}{z^3(1-z)(z^2+z+1)} \right)                                  \\
		               & \lim_{z\to1} \left(  \frac{-1}{z^3(z^2+z+1)} \right)                                           \\
		               & = \frac{-1}{1(1+1+1)}                                                                          \\
		               & = -\frac 13
	\end{align*}
	Thus:
	$$\int_C f(z)dz =-\frac{2\pi i}3$$

	\section*{Q3}
	\begin{enumerate}[label=(\alph*)]
		\item We know the Taylor series of $e^z$ about $z=0$ which converges to $e^z$ is:
		      $$\sum_{n=0}^\infty \frac {z^n}{n!}=1+z+\frac{z^2}2+\frac{z^3}6+\dots$$
		      From lectures we also know that the Laurent series for $\frac{1}{\sin z}$ in the region $0<|z|<\pi$ is:
		      $$z^{-1}+z\left( \frac 1{3!} \right) + z^3 \left(  - \frac 1{5!} + \left(\frac 1{3!}\right)^2 \right)+\dots$$
		      Multiplying these series to the third term gives:
		      \begin{align*}
			        & z^{-1}\cdot 1 \;\;+\;\; z^{-1}\cdot z  \;\;+\;\; z^{-1}\cdot \frac{z^2}2  \;\;+\;\;  \frac z{3!} \cdot \frac{z^0}1 \\
			      = & \frac 1z + 1 + \frac{2z}3
		      \end{align*}



		\item We use the series
		      $$(1-a)^{-1}=1+a+a^2+a^3+\dots \quad\text{For $|a| < 1$}$$
		      Let $\omega = z-1$ so $z=\omega+1$, where $|\omega| >1\implies |\omega^{-1}|<1$. Thus:


		      \begin{align*}
			      g(z) & = \frac{z-2}{z(z-1)}                                                                                                                     \\
			           & = \frac{\omega-1}{(\omega + 1)\omega}                                                                                                    \\
			           & = \frac{\omega-1}{(1 + \frac{1}\omega)\omega^2}                                                                                          \\
			           & = \frac{\omega-1}{\omega^2} \left(1 - \left(-\frac1\omega\right) \right)^{-1}                                                            \\
			           & = \left[ \frac1\omega - \frac 1{\omega^2}  \right] \left[ 1-\frac1\omega+\frac1{\omega^2}-\frac1{\omega^3}+\dots \right]                 \\
			           & = \left[ \omega^{-1}-1 \right] + \left[ -1 +\omega  \right] +\left[ \omega - \omega^2 \right] + \left[ -\omega^2 +\omega^3 \right]+\dots \\
			           & = \frac 1\omega -2 +2\omega  -2 \omega^2 +2\omega^3 +\dots                                                                               \\
		      \end{align*}
		      Thus, the first three terms in the Laurent series of $g(z)$ about $z=1$ in the region $|z-1>1$ are:

		      $$1 (z-1)^{-1}  - 2(z-1)^{-2} + 2(z-1)^{-3} $$



	\end{enumerate}

	\section*{Q4}
	\begin{align*}
		  & \int_{0}^{\pi}\frac{1}{6+\cos\left(t\right)}dt                                \\
		= & - \int_{2\pi}^{\pi}\frac{1}{6+\cos\left(u \right)}du\quad u=2\pi-t, dt = - du \\
		= & \int_{\pi}^{2\pi}\frac{1}{6+\cos\left(t\right)}dt                             \\
	\end{align*}
	Therefore:
	\begin{align*}
		  & \int_{0}^{\pi}\frac{1}{6+\cos\left(t\right)}dt           \\
		= & \frac 12 \int_{0}^{2\pi}\frac{1}{6+\cos\left(t\right)}dt
	\end{align*}
	Now let be $C$ an anticlockwise unit circle in the complex plane. So $z=e^{it}$ is on the circle. Hence:
	$$\cos t = \frac{z+\frac 1z}2$$
	And:
	$$dz= ie^{it}dt=izdt \implies dt = \frac {dz}{iz}$$
	Thus, the integral can be written as:
	\begin{align*}
		  & \int_{0}^{\pi}\frac{1}{6+\cos\left(t\right)}dt                 \\
		= & \frac 12 \int_{0}^{2\pi}\frac{1}{6+\cos\left(t\right)}dt       \\
		= & \frac 12 \int_C\frac{1}{iz\left( 6+\frac{z+z^{-1}}2 \right)}dz \\
		= & \int_C\frac{1}{i\left( 12+z^2+1 \right)}dz                     \\
		= & -i\int_C\frac{1}{ z^2+12z+1 }dz                                \\
	\end{align*}
	The integrand is not analytic at the roots of $z^2+12z+1$, using the quadratic formula:
	\begin{align*}
		  & \frac{-12\pm \sqrt{12^2-4}}{2} \\
		= & \frac{-12\pm 2\sqrt{35}}{2}    \\
		= & -6\pm \sqrt{35}
	\end{align*}
	Thus $f$ is not analytic at the single isolated point in $C$, $z=\sqrt{35}-6=a$. Which will be a simple pole as we can factorise the denominator into linear factors:
	$$(z+6-\sqrt{35})(z+6+\sqrt{35})$$

	Thus, the residue at the pole in $C$ is:
	\begin{align*}
		\text{Res}_a f & = \frac 1{(m-1)!}\lim_{z\to a} \left( \dd{^{m-1}}{z^{m-1}} \left[ (z-a)^m f(z) \right] \right) \\
		               & =\lim_{z\to a} \left( (z-a)f(z) \right)                                                        \\
		               & =\lim_{z\to a} \left(  \frac{z+6-\sqrt{35}}{(z+6-\sqrt{35})(z+6+\sqrt{35})}\right)             \\
		               & =\lim_{z\to a} \left(  \frac{1}{(z+6+\sqrt{35})}\right)                                        \\
		               & =  \frac{1}{\sqrt{35}-6+6+\sqrt{35}}                                                           \\
		               & =  \frac{1}{2\sqrt{35}}                                                                        \\
	\end{align*}




	Since $C$ is a piecewise smooth closed curve oriented anticlockwise, and $f(z)$ is analytic on and inside $C$ except at $1$ isolated points $z=a$, we can apply the residue theorem:

	\[
		-i \left[ \int_C\frac{1}{ z^2+12z+1 }dz  \right]  = -i \left[ 2\pi i \frac{1}{2\sqrt {35}} \right] = \frac{\pi}{\sqrt 35}
	\]
	Thus we have found:
	\[
		\int_0^{\pi} \frac 1{1+6\cos t}dt = \frac{\pi}{\sqrt 35}
	\]


\end{multicols}
\end{document}
