
\documentclass[a4paper, 11pt]{article}

% Fonts 
\usepackage{opensans}
\usepackage{amsfonts}
\usepackage{montserrat}
\usepackage{amsmath}

\setmainfont{opensans}
\usepackage[mathrm=sym]{unicode-math}
% \setmathfont[Path=font/,mathrm=sym]{FiraMath-Regular}
\setmathfont[Path=font/,mathrm=sym]{LatoMath}

\newfontfamily{\montserrateb}{Montserrat SemiBold}
\newfontfamily{\montserratb}{Montserrat Bold}
\newfontfamily{\montserrat}{Montserrat Regular}
\newfontfamily{\montserratl}{Montserrat Light}
% \DeclareMathAlphabet{\mathcal}{OMS}{cmbrs}{m}{n}

% \usepackage[mathrm=sym]{unicode-math}
% \setmainfont{opensans}
% \setmathfont{Fira Math}

% \newfontfamily{\montserrateb}{Montserrat SemiBold}
% \newfontfamily{\montserratb}{Montserrat Bold}
% \newfontfamily{\montserrat}{Montserrat Regular}
% \newfontfamily{\montserratl}{Montserrat Light}
% \DeclareMathAlphabet{\mathcal}{OMS}{cmbrs}{m}{n}
% \setmathfont{Latin Modern Math}[range={\vdots}]

% \autoref
\usepackage{hyperref}

% Use for [H] option for figures to force in text placement
\usepackage{float}

% Captioning figures
\usepackage{caption}

% Subfigures
\usepackage{subcaption}

% For extending contents beyond margins
\usepackage{scrextend}

% For tables \midrule ect
\usepackage{booktabs}

% Colours
\usepackage[table,xcdraw]{xcolor}
\definecolor{accentcolor}{HTML}{a13640}

% Change label in enumerate 
\usepackage{enumitem}

% Section settings
\usepackage{titlesec}
\titleformat{\section}
{\LARGE\montserrateb}
{\thesection.}{0.5em}{}

\titleformat{\subsection}
{\large\montserratb}
{\thesubsection.}{0.5em}{}

% Adjust document dimensions
\ExecuteOptions{a4paper}
\addtolength{\oddsidemargin}{-3cm}
\addtolength{\evensidemargin}{-3cm}
\addtolength{\topmargin}{-3cm}
\addtolength{\textwidth}{6cm}
\addtolength{\textheight}{4.5cm}
\addtolength{\textheight}{1.5cm}
\addtolength{\headsep}{-0.5cm}
% \addtolength{\footskip}{-1cm}
\parindent0pt
\parskip=4pt



\usepackage{matlab-prettifier}
\usepackage{graphicx}
\usepackage{mdframed}

% Creates coloured title box
\newcommand{\thetop}[5]{
	\begin{addmargin}[\oddsidemargin]{\oddsidemargin}
		\colorbox{#5}{\color{white}
			\hbox to \paperwidth{
				\vbox {
					\begin{center}
						{\large\montserratl #1}\\
						\vspace{4pt}
						{\huge\montserratb #2}\\
						{\montserratb #3}\\
						\vspace{-0.5em}
						\rule{20em}{1pt}

						{\large\montserratl
							#4
						}
					\end{center}
				}
			}
		}
	\end{addmargin}
}

\newcommand{\NN}{\mathbb{N}}
\newcommand{\ZZ}{\mathbb{Z}}
\newcommand{\RR}{\mathbb{R}}
\newcommand{\CC}{\mathbb{C}}
\newcommand{\dydt}{\frac{dy}{dt}}
\newcommand{\dxdt}{\frac{dx}{dt}}
\def\set#1{\left\{ #1 \right\}}
\def\eval#1#2{\left\ #1\right|_{#2}}

\def\pp#1#2{\frac{\partial #1}{\partial #2}}
\def\dd#1#2{\frac{\,d#1}{\,d#2}}
\def\abs#1{\left|#1\right|}
\def\conj#1{\overline{#1}}

\usepackage{multicol}
\usepackage{tikz}
\usepackage{pgfplots}
\usetikzlibrary {graphs,graphdrawing} \usegdlibrary {force} 
\usetikzlibrary{graphs.standard}
\usegdlibrary {circular}

\usepackage{pst-platon}

\begin{document}
\thetop{Robert Christie}{MATHS 326}{S1 2024}{Assignment 1\\Due: 22-03-2024}{accentcolor}

\begin{multicols*}{2}
	\section*{Q1}
	\begin{enumerate}[label=(\alph*)]
		\item Call the graph $G$, and apply deletion contraction theorem to the edge $e$ in red:

		      \begin{center}
			      \tikz \graph [empty nodes, nodes={circle, draw}, spring layout] { a --[red] b -- c -- a -- d -- c -- e -- f -- d };
		      \end{center}

		      Thus, $P_G(x) = P_{G-e}(x) - P_{G_e}(x)$  This produces two graphs, $G-e$ and $G_e$:
		      \begin{center}
			      \begin{tabular}{cc}
				      \tikz \graph [empty nodes, nodes={circle, draw}, spring layout] {  b -- c -- a -- d -- c -- e -- f -- d };
				              & \tikz \graph [empty nodes, nodes={circle, draw}, spring layout] {  b -- c -- b -- d -- c -- e -- f -- d };
				      \\
				      $G - e$ & $G_e$
			      \end{tabular}
		      \end{center}

		      Notice that $G-e$ can be constructed by adding a vertex to $H=G_e$. Applying deletion contraction to $H$:
		      \begin{center}
			      \begin{tabular}{cc}
				      \tikz \graph [empty nodes, nodes={circle, draw}, spring layout] {  a -- b -- c -- d -- a -- e};
				            & \tikz \graph [empty nodes, nodes={circle, draw}, spring layout] {  a -- b -- c -- d -- a};
				      \\
				      $H-e$ & $H_e$
			      \end{tabular}
		      \end{center}
		      We see that $H_e = C_4$, and $H-e$ is $C_4$ with an additional vertex $v$ added, which can take any colour except that of its neighbour. Thus:
		      \begin{align*}
			      P_{H-e}(x) & = (x-1)C_4(x) \\
			      P_{H_e}(x) & = C_4(x)
		      \end{align*}
		      Using this with the deletion contraction gives the chromatic polynomial of $H$:
		      \begin{align*}
			      P_H(x) & = P_{H-e}(x) - P_{H_e}(x) \\
			             & = (x-2)C_4(x)             \\
			             & = (x-2)((x-1)^4+x-1)
		      \end{align*}

		      Since $G-e$, is just $H$ with an additional vertex $v$ with a single neighbor, we find:
		      \begin{align*}
			      P_{G-e}(x)=(x-1)H
		      \end{align*}

		      From these results, we finish the deletion contraction of $G$:
		      \begin{align*}
			      P_G(x) & = P_{G-e}(x) - P_{G_e}(x) \\
			             & = (x-1)H - H              \\
			             & = (x-2)H                  \\
			             & = (x-2)^2((x-1)^4+x-1)
		      \end{align*}
		\item
		      Let $G$ be the given graph, apply deletion contraction theorem on the edge in red:
		      \begin{center}
			      \tikz \graph [empty nodes, nodes={circle, draw}, spring layout] { a -- z -- b -- c -- a -- d -- c -- e --[red] f -- d };
		      \end{center}
		      This produces graphs:
		      \begin{center}
			      \begin{tabular}{cc}
				      \tikz \graph [empty nodes, nodes={circle, draw}, spring layout] { z -- b -- c -- a -- d -- c -- e -- f -- d };
				              & \tikz \graph [empty nodes, nodes={circle, draw}, spring layout] { a -- b -- c -- a -- d -- c -- e -- f -- d };
				      \\
				      $G - e$ & $G_e$
			      \end{tabular}
		      \end{center}
		      Notice that $G_e$ is the graph from part (a), while $G-e$ can be constructed by adding two vertices to the graph $H$ from the working for part (a). Notice that both the added vertices have only one neighbour each, so they can have any of $(x-1)$ colours for each existing colouring of $H$. Thus, the chromatic polynomials are:
		      \begin{align*}
			      P_{G-e}    & = (x-1)^2 P_H(x)            \\
			                 & = (x-1)^2(x-2)((x-1)^4+x-1) \\
			      P_{G_e}(x) & = P_{G_a}(x)                \\
			                 & = (x-2)^2((x-1)^4+x-1)
		      \end{align*}
		      Finally we use these results in the deletion contraction theorem applied to $G$:
		      \begin{align*}
			      P_G(x) & = (x-1)^2(x-2)((x-1)^4+x-1)
			      \\
			             & - (x-2)^2 ( (x-1)^4 + (x-1))                            \\
			             & = \left((x-1)^2-x+2\right)(x-2)\left((x-1)^4+x-1\right)
		      \end{align*}
	\end{enumerate}

	\section*{Q2}

	\begin{enumerate}[label=(\alph*)]
		\item

		      First we prove that: $P_G(x)$ contains an $x$ term, implies $G$ is connected.

		      \begin{mdframed}
			      We prove the contrapositive: $G$ is disconnected implies that $P_G(x)$ does not contain a non-zero $x$ term.

			      Consider any disconnected graph $G$ which can be expressed as the disjoint union of $k\geq2$ subgraphs $H_1,\dots,H_k$.

			      By \textit{Lemma 1.7}:
			      $$P_G(x)= \prod_{i=1}^k P_{H_i}(x)$$
			      Since $P_{H_i}(x)$ are chromatic polynomials, they do not contain a constant term. Thus:

			      \begin{align*}
				       & \begin{matrix}
					          & \hphantom{=} P_G(x) \\
				         \end{matrix}                         \\
				       & \begin{matrix}
					         = & (x^{n_1} + \dots + c_{2,1}x^2 + c_{1,1}x ) \\
					           & \times                                     \\
					           & \vdots                                     \\
					           & \times                                     \\
					           & (x^{n_k} + \dots + c_{2,k}x^2 + c_{1,k}x )
				         \end{matrix} \\
				       & \begin{matrix}
					         = & \text{H.O.T.} + (c_{1,1}\dots c_{1,k})x^k \\
				         \end{matrix}  \\
			      \end{align*}
			      Which does not contain a non-zero $x$.
		      \end{mdframed}


		      Now we prove the inverse: $G$ is connected, implies $P_G(x)$ contains a non-zero $x$ term.

		      \begin{mdframed}
			      Apply strong induction on $m=\abs{E}$ for connected graphs $G=(V,E)$.

			      \textbf{Base case:}

			      For $\abs{E}=0$, the only connected graph is a single vertex with $P_G(x)=x$.

			      \vspace{1em}

			      \textbf{Induction Step ``$1,...,m\implies m+1$'':}

			      Consider a graph with $\abs{E}=m+1$, apply deletion contraction, thus:
			      $$P_G(x)= P_{G-e}(x) - P_{G_e}(x)$$

			      Where $G_e$ has $m$ or fewer edges. Clearly $G_e$ is connected, as any path from $u$ to $v$ in $G_e$ can be constructed from an existing path in $G$.

			      Since $G_e$ is connected with $\leq m$ edges, the induction hypothesis applies and $P_{G_e}$ contains a non-zero $x$ term.

			      Since $G$ has $n$ vertices, so does $G-e$ so $P_{G-e}(x)$ is of degree $n$ while $G_e$ has $n-1$ vertices so $P_{G_e}(x)$ has degree $n-1$.

			      For a chromatic polynomial, the $x^n$ has coefficient $1$ and the signs alternate, since $P_{G-e}$ and $P_{G_e}$ differ in degree by $1$ they have opposite signs for the $x$ term.

			      So by deletion contraction for some $c_a\geq0,c_b>0$:
			      \begin{align*}
				      P_G(x) & = P_{G-e}(x) - P_{G_e}(x)      \\
				             & = (x^n + \dots \pm c_a x )     \\
				             & -  (x^{n-1} + \dots \mp c_b x) \\
				             & = x^n + \dots \pm (c_a + c_b)x
			      \end{align*}
			      As $c_a+c_b > 0$, $P_G(x)$ must also contain a non-zero $x$ term.
		      \end{mdframed}

		      Thus, we have shown that $G$ is connected if and only if $P_G(x)$ contains a non-zero $x$ term.

		\item
		      When $G$ is connected, \textit{Corollary 1.15} still applies and so $G$ is a tree if and only if $P_G(x)=x(x-1)^{n-1}$. So the new statement holds in this case.

		      When $G$ is disconnected, $G$ is not a tree as trees are connected by definition. However, $P_G(x)$ also cannot contain a non-zero $x$ term by (a). So the statement holds.

		      Since all simple graphs are either connected or disconnected, the new statement is true.
	\end{enumerate}

	\section*{Q3}
	\begin{enumerate}[label=(\alph*)]
		\item Expanding $P(x)=(x-1)^4$ results in a polynomial with constant term $(-1)^4=1$, so by \textit{Theorem 1.14}, $P(x)$ is not the chromatic polynomial of any graph as it has a non-zero constant term.

		\item Let $P(x)=x^6+6x^5+7x^3-2x$. As $x^3$ and $-2x$ have different signs but are $2$ terms apart the signs of coefficients do not alternate. Thus, by \textit{Theorem 1.14}, $P(x)$ is not the chromatic polynomial of any graph.

		\item Assume $P(x)=x^4-3x^3+4x^2-2x$ is a chromatic polynomial for some $G$. We can deduce the following using \textit{Theorem 1.14}:
		      \begin{itemize}
			      \item Since $P$ has degree $4$, $G$ has $4$ vertices.
			      \item The $x^{n-1}=x^3$ term has a coefficient of $-3$, so $G$ must have $3$ edges.
			      \item The $x$ term has a non-zero coefficient, thus $G$ is connected.
		      \end{itemize}
		      We can't have a cycle in $G$ as the smallest cycle requires all $3$ edges, and so we can't connect all vertices. As $G$ is connected and acyclic, it is a tree.

		      By part (2b), $G$ is a tree if and only if it has chromatic polynomial:
		      $$P_G(x)=x(x-1)^{n-1}=x(x-1)^3$$.
		      Now see that:
		      $$P_G(2)=2 \neq P(2)= 2^4 - 3\cdot 2^3 + 4 \cdot 2^2 - 2\cdot 2 = 4$$
		      Thus, $P(x)$ is not $P_G(x)$.

		      As this resulted in a contraction, $P$ is not the chromatic polynomial for any graph $G$.
	\end{enumerate}

	\section*{Q4}

	Let $G=(V,E)$ be any planar simple graph.

	When $|V|\leq 2$, $G$ has at most $1$ edge so the statement holds.

	For the case with $3\leq n\leq 11$ vertices and $m$ edges. We use proof by contraction:
	\begin{mdframed}
		By \textit{Theorem 1.19}, we know that $m\leq3n-6$.

		Assume all vertices of $G$ have degree at least $5$. By the handshaking lemma:
		\[
			5n\leq	\sum_{v\in V}\operatorname{deg}(v) = 2\abs{E} = 2m
		\]
		Thus, $2.5n \leq m \leq 3n -6 $. We can only find such $m$ when:
		\begin{align*}
			     & 2.5n \leq 3n- 6 \\
			\iff & 6 \leq 0.5n     \\
			\iff & 12 \leq n
		\end{align*}
		This is a contraction for $n<12$, so the assumption must be false and for graphs with $n<12$, there exists a vertex with degree at most $4$.
	\end{mdframed}

	To show the bound is sharp we look for a graph on $12$ vertices such that $\forall_{v\in V} \deg{v} \geq 5$. We derived the bound:
	$$2.5n = 30 \leq m \leq 3n-6=30$$
	So we have exactly $m=30$ edges. Since $5\cdot 12 = 2 \cdot 30$ each vertex has degree exactly $5$.

	The regular icosahedron is a Platonic solid with $12$ vertices and $30$ edges where each vertex is adjacent to $5$ edges. This is similar to our graph requirements.
	\begin{center}
		\psIcosahedron[Frame=false,faceName=\hphantom]
	\end{center}
	In the 2D projection above, the $3$ vertices of the back most hidden face can be scaled up to remove all edge-edge intersections in the protection. We can define a graph from the edges and vertices of the icosahedron and consider the modified projection a planar embedding of this graph:
	\begin{center}
		\begin{tikzpicture}[every node/.style={draw,circle}]
			\begin{scope}[shift={(0,3cm)}]
				\graph[empty nodes, simple necklace layout, clockwise, radius=3cm] {
					O1 -- O2 -- O3 -- O1;
				};
			\end{scope}

			\begin{scope}[shift={(0,1cm)}]
				\graph[empty nodes, simple necklace layout, clockwise, radius=1cm] {
					1 -- 2 -- 3 -- 4 -- 5 -- 6 -- 1;
				};
			\end{scope}

			\begin{scope}[shift={(0,0.0cm)}, rotate=60]
				\begin{scope}[shift={(0,0.5cm)}]
					\graph[empty nodes, simple necklace layout, clockwise, radius=0.5cm] {
						I1 -- I2 -- I3 -- I1;

					};
				\end{scope}
			\end{scope}
			\draw (O1) -- (6) -- (I3);
			\draw (O1) -- (2) -- (I1);
			\draw (O1) -- (1) -- (I1);
			\draw 				(1) -- (I3);

			\draw (O2) -- (2) -- (I1);
			\draw (O2) -- (4) -- (I2);
			\draw (O2) -- (3) -- (I2);
			\draw 				(3) -- (I1);

			\draw (O3) -- (4) -- (I2);
			\draw (O3) -- (6) -- (I3);
			\draw (O3) -- (5) -- (I3);
			\draw 				(5) -- (I2);
		\end{tikzpicture}
	\end{center}
	This is a planar simple $5$-regular graph on $12$ vertices demonstrating the bound is sharp as each vertex has degree $5>4$.

	\section*{Q5}
	\begin{enumerate}[label=(\alph*)]
		\item We can create a $3$-colouring of the Petersen graph:
		      \begin{center}
			      \begin{tikzpicture}[every node/.style={draw,circle}]
				      \begin{scope}[shift={(0,-1cm)}]
					      \graph[empty nodes, simple necklace layout, clockwise, radius=1cm] {
						      A[green], B[red], C[red], D[blue], E[green];
						      A -- C -- E -- B -- D -- A;
					      };
				      \end{scope}
				      \graph[empty nodes, simple necklace layout, clockwise, radius=2cm] {
					      1[red] -- 2[blue] -- 3[green] -- 4[red] -- 5[blue] -- 1;
				      };
				      \draw (1) -- (A);
				      \draw (2) -- (B);
				      \draw (3) -- (C);
				      \draw (4) -- (D);
				      \draw (5) -- (E);
			      \end{tikzpicture}
		      \end{center}
		      Therefore, $\chi(P)\leq 3$, however, since $P$ contains an odd cycle of length $5$, we also have $\chi(P)\geq3$, hence $\chi(P)=3$.


		      The following proper edge $4$-colouring of $P$ exists:
		      \begin{center}
			      \begin{tikzpicture}[every node/.style={draw,circle}]
				      \begin{scope}[shift={(0,-1cm)}]
					      \graph[empty nodes, simple necklace layout, clockwise, radius=1cm] {
					      A, B, C, D, E;
					      A --[red] C --[green] E --[red] B --[green] D --[blue] A;
					      };
				      \end{scope}
				      \graph[empty nodes, simple necklace layout, clockwise, radius=2cm] {
				      1--[red]2--[green]3--[red]4--[green]5--[yellow]1;
				      };
				      \draw[green] (1) -- (A);
				      \draw[blue]  (2) -- (B);
				      \draw[blue]  (3) -- (C);
				      \draw[blue]  (4) -- (D);
				      \draw[blue]  (5) -- (E);
			      \end{tikzpicture}
		      \end{center}
		      Thus $\chi_e(P)\leq 4$.

		      Next we show that no edge $3$-colouring exists:
		      \begin{itemize}
			      \item
			            The cycle $C_5$ (a subgraph of $P$) has chromatic index $3$. So no edge $2$-colouring exists.

			      \item
			            If we take any graph with $3$ edges where no pair is adjacent, then there a $6$ vertices in the graph.

			            Since $C_5$ has only $5$ vertices, we cannot find a subgraph with $3$ edges such that no pair is adjacent.

			            Since we can't find $3$ non-adjacent edges to give the same colour, each colour is used at most twice in an edge-colouring of $C_5$.

			      \item
			            We must use $2$ colours twice and $1$ colour once in order to colour all $5$ edges of $C_5$ as each colour is used either once or twice.

			      \item
			            WLOG, let \textcolor{red}{$r$} be the colour used once. The remaining colours $\textcolor{green}{g},\textcolor{blue}{b}$ must alternate on the remaining edges.

			            \begin{center}
				            \begin{tikzpicture}[every node/.style={draw,circle}]
					            \graph[empty nodes, simple necklace layout, clockwise, radius=1cm] {
					            A --[red] B --[green] C --[blue] D --[green] E --[blue] A;
					            };
				            \end{tikzpicture}
			            \end{center}

			      \item
			            When edges are added to $C_5$ to produce the following subgraph of $P$, their colouring is fully determined by the existing edges in the cycle.
			            \begin{center}
				            \begin{tikzpicture}[every node/.style={draw,circle}]
					            \begin{scope}[shift={(0,-1cm)}]
						            \graph[empty nodes, simple necklace layout, clockwise, radius=1cm] {
							            A, B, C, D, E;
						            };
					            \end{scope}
					            \graph[empty nodes, simple necklace layout, clockwise, radius=2cm] {
					            1--[red]2--[blue]3--[green]4--[blue]5--[green]1;
					            };
					            \draw[blue] (1) -- (A);
					            \draw[green]  (2) -- (B);
					            \draw[red]  (3) -- (C);
					            \draw[red]  (4) -- (D);
					            \draw[red]  (5) -- (E);
				            \end{tikzpicture}
			            \end{center}

			      \item Consider the subgraph of $P$ created by adding the two dashed edges to the previous subgraph:
			            \begin{center}
				            \begin{tikzpicture}[every node/.style={draw,circle}]
					            \begin{scope}[shift={(0,-1cm)}]
						            \graph[empty nodes, simple necklace layout, clockwise, radius=1cm] {
						            A, B, C, D, E;

						            D --[dashed] A --[dashed] C;
						            };
					            \end{scope}
					            \graph[empty nodes, simple necklace layout, clockwise, radius=2cm] {
					            1--[red]2--[blue]3--[green]4--[blue]5--[green]1;
					            };
					            \draw[blue] (1) -- (A);
					            \draw[green]  (2) -- (B);
					            \draw[red]  (3) -- (C);
					            \draw[red]  (4) -- (D);
					            \draw[red]  (5) -- (E);
				            \end{tikzpicture}
			            \end{center}
			            Both edges are adjacent to $\textcolor{red}{r}$ and $\textcolor{blue}{b}$ edges already, so they must both be $\textcolor{green}{g}$. However, since they are adjacent they cannot receive the same colours.  Thus, we have a contraction, so $P$ is not $3$ edge-colourable and $\chi_e(P)>3$.
		      \end{itemize}
		      Since $3<\chi_e(P)\leq 4$, the chromatic index of the Petersen graph $\chi_e(P)=4$.

		\item Contract the edges of $P$ show in red below:
		      \begin{center}
			      \begin{tikzpicture}[every node/.style={draw,circle}]
				      \begin{scope}[shift={(0,-1cm)}]
					      \graph[empty nodes, simple necklace layout, clockwise, radius=1cm] {
						      A, B, C, D, E;
						      A -- C -- E -- B -- D -- A;
					      };
				      \end{scope}
				      \graph[empty nodes, simple necklace layout, clockwise, radius=2cm] {
					      1--2--3--4--5--1
				      };
				      \draw[red] (1) -- (A);
				      \draw[red]  (2) -- (B);
				      \draw[red]  (3) -- (C);
				      \draw[red]  (4) -- (D);
				      \draw[red]  (5) -- (E);
			      \end{tikzpicture}
		      \end{center}
		      This produces $K_5$ so by Wagner's theorem, $P$ must be non-planar.
	\end{enumerate}

	\section*{Q6}
	First we prove that for a simple graph $G$: $G$ is bipartite implies every subgraph $H$ of $G$ satisfies $\alpha(H)\geq m_H/2$:
	\begin{mdframed}
		Let $G$ be a bipartite graph. By \textit{Theorem 1.6}, $G$ contains no odd cycle. So any subgraph $H$ of $G$ also contains no odd cycle, so by \textit{Theorem 1.6}, $H=(V,E)$ is bipartite. Therefore, its vertices can be written as the disjoint union $V=V_1\uplus V_2$ so that every edge in $H$ has one endpoint in each of $V_1,V_2$.

		Let $V_i,V_j$ be the largest and smallest of the two vertex sets respectively (or $V_1,V_2$ if the sets have the same size). Clearly $V_i$ must contain at least $m_H/2$ vertices, otherwise:
		$$|V_j|\leq|V_i|<m_H/2\implies |V_1|+|V_2|<m_H=|V|$$
		By the definition of a bipartite graph, no two vertices in $V_i$ are adjacent, thus $V_i$ is a set of independent vertices and so $\alpha(H)\geq V_i\geq m_H/2$.
	\end{mdframed}

	\vfill\null
	\columnbreak
	Now we prove the inverse to show equivalence:

	\begin{mdframed}
		Consider any odd cycle graph $C_{2k+1}$ with vertex set $V$. Assume that $\alpha(C_{2k+1})\geq |V|/2$. There must be an independent set $U$ containing at least $|V|/2$ vertices. Since $|V|$ is odd $|U|>|V|/2$.

		Every $v\in V$ is adjacent to exactly two other vertices. Therefore, every $u\in U$ is adjacent to two vertices $w_1,w_2\in V\setminus U$. Let $W$ be the count of how many distinct $\set{u,w_i}$ edges exist, $W=2|U|$. Since each $w_i$ vertex appears in at most $2$ edges, there at least $W/2$ such vertices and so:
		$$|V\setminus U|\geq \frac W2 = |U|$$
		Therefore:
		$$|U|+|V\setminus U|=2|U|>|V|$$
		Which is a contraction as $U$ and $V\setminus U$ are disjoint subsets of $V$, so we should have:
		$$|U|+|V\setminus U|=|V|$$
		As our assumption was false:
		$$\alpha(C_{2k+1})<|V|/2$$
		For any odd cycle graph $C_{2k+1}$.

		Now let $G$ be a graph where every subgraph $H$ of $G$ satisfies $\alpha(H)\geq m_H/2$. Thus, $G$ contains no subgraph $H$ with $\alpha(H)< m_H/2$. By the previous working, $G$ contains no odd cycles and so by \textit{Theorem 1.6}, $G$ is bipartite.
	\end{mdframed}

	Thus, we have shown that $G$ is bipartite if and only if every subgraph $H$ of $G$ satisfies $\alpha(H)\geq m_H/2$.

	\vfill
	\pagebreak
\end{multicols*}
\end{document}
