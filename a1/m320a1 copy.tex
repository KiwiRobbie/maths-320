
\documentclass[a4paper, 11pt]{article}

% Fonts 
\usepackage{opensans}
\usepackage{amsfonts}
\usepackage{montserrat}
\usepackage{amsmath}

\setmainfont{opensans}
% \usepackage[mathrm=sym]{unicode-math}
\usepackage{lato-math}
% \setmathfont[Path=font/,mathrm=sym]{FiraMath-Regular}
% \setmathfont[Path=font/,mathrm=sym]{LatoMath}

\newfontfamily{\montserrateb}{Montserrat SemiBold}
\newfontfamily{\montserratb}{Montserrat Bold}
\newfontfamily{\montserrat}{Montserrat Regular}
\newfontfamily{\montserratl}{Montserrat Light}
% \DeclareMathAlphabet{\mathcal}{OMS}{cmbrs}{m}{n}

% \usepackage[mathrm=sym]{unicode-math}
% \setmainfont{opensans}
% \setmathfont{Fira Math}

% \newfontfamily{\montserrateb}{Montserrat SemiBold}
% \newfontfamily{\montserratb}{Montserrat Bold}
% \newfontfamily{\montserrat}{Montserrat Regular}
% \newfontfamily{\montserratl}{Montserrat Light}
% \DeclareMathAlphabet{\mathcal}{OMS}{cmbrs}{m}{n}
% \setmathfont{Latin Modern Math}[range={\vdots}]

% \autoref
\usepackage{hyperref}

% Use for [H] option for figures to force in text placement
\usepackage{float}

% Captioning figures
\usepackage{caption}

% Subfigures
\usepackage{subcaption}

% For extending contents beyond margins
\usepackage{scrextend}

% For tables \midrule ect
\usepackage{booktabs}

% Colours
\usepackage[table,xcdraw]{xcolor}
\definecolor{accentcolor}{HTML}{a13640}

% Change label in enumerate 
\usepackage{enumitem}

% Section settings
\usepackage{titlesec}
\titleformat{\section}
{\LARGE\montserrateb}
{\thesection.}{0.5em}{}

\titleformat{\subsection}
{\large\montserratb}
{\thesubsection.}{0.5em}{}

% Adjust document dimensions
\ExecuteOptions{a4paper}
\addtolength{\oddsidemargin}{-3cm}
\addtolength{\evensidemargin}{-3cm}
\addtolength{\topmargin}{-3cm}
\addtolength{\textwidth}{6cm}
\addtolength{\textheight}{4.5cm}
\addtolength{\textheight}{1.5cm}
\addtolength{\headsep}{-0.5cm}
% \addtolength{\footskip}{-1cm}
\parindent0pt
\parskip=4pt



\usepackage{matlab-prettifier}
\usepackage{graphicx}
\usepackage{mdframed}

% Creates coloured title box
\newcommand{\thetop}[5]{
	\begin{addmargin}[\oddsidemargin]{\oddsidemargin}
		\colorbox{#5}{\color{white}
			\hbox to \paperwidth{
				\vbox {
					\begin{center}
						{\large\montserratl #1}\\
						\vspace{4pt}
						{\huge\montserratb #2}\\
						{\montserratb #3}\\
						\vspace{-0.5em}
						\rule{20em}{1pt}

						{\large\montserratl
							#4
						}
					\end{center}
				}
			}
		}
	\end{addmargin}
}

\newcommand{\NN}{\mathbb{N}}
\newcommand{\ZZ}{\mathbb{Z}}
\newcommand{\QQ}{\mathbb{Q}}
\newcommand{\RR}{\mathbb{R}}
\newcommand{\CC}{\mathbb{C}}

\newcommand{\dydt}{\frac{dy}{dt}}
\newcommand{\dxdt}{\frac{dx}{dt}}
\def\set#1{\left\{ #1 \right\}}
\def\eval#1#2{\left\ #1\right|_{#2}}

\def\pp#1#2{\frac{\partial #1}{\partial #2}}
\def\dd#1#2{\frac{\,d#1}{\,d#2}}
\def\abs#1{\left|#1\right|}
\def\conj#1{\overline{#1}}

\usepackage{multicol}
\usepackage{tikz}
\usepackage{pgfplots}
\usetikzlibrary {graphs,graphdrawing} \usegdlibrary {force} 
\usetikzlibrary{graphs.standard}
\usetikzlibrary{positioning, 
                quotes}
\usegdlibrary {circular}

\usepackage{pst-platon}

\usepackage{comment}
\usepackage{cancel}

\begin{comment}
	Q1a: Good
	Q1b:  
	Q1c: Good
	Q2   Good
	Q3:  BADDDD
	Q4:  MEHH
	Q5   Good
	Q6:  Good 
\end{comment}

\begin{document}
% \thetop{Robert Christie}{MATHS 320}{S2 2024}{Assignment 1\\Due: 24-05-2024}{accentcolor}


\section*{Setup}
Prison contains some unknown number $N$ of prisoners. The prison consists of $N$ identical rooms arranged into a cycle.

Each day, all prisoners can send $1$ bit of information to the prisoner of the next cell in the cycle. Prisoners send their information before receiving the communication from the previous cell.

All prisoners can count the number of days and agree on the count. 


\section*{Upper bound}
\subsection*{Testing an upper bound}
Given some $n\in\NN$ we can test if $n$ is an upper bound on the number of prisoners $N$.

Initially, everyone agrees on the test value $n$. Let $X$ be a set of contacted prisoners. Initially, $X$ only contains yourself. 

For the first $n-1$ days of the test, prisoners send a signal if and only if there are in $X$.

\begin{itemize}
	\item Each day the size of $X$ can at most double, as each member can contact at most one prisoner. 

	\item Each day, the size of $X$ must increase by at least one as long as there exist prisoners not in $X$. 
	
	If $X$ did not increase, all signals were received by prisoners in $X$. Thus, $X$ forms a cycle, the only cycle contains all prisoners.
\end{itemize}

Starting from day $n$, each prisoner sends a signal if and only if they are not in $X$. If a prisoner now receives a signal, they are removed from $X$. 

The prisoners of $X^c$ cannot form a cycle (unless $X=\emptyset$), so at least one signal from $X^c$ must reach a prisoner in $X$, thus decreasing the size of $X$ by $1$. 

If initially $X^c=\emptyset$,  after at most $2^n$ days, all prisoners will be removed from $X$.

After the test completes, either: 
\begin{itemize}
	\item 
		All prisoners are in $X$, so $|X|\leq 2^n$ provides an upper bound and all prisoners know the test succeeded. 
	\item 
		Otherwise, all prisoners are in $X^c$ and so all know that the test failed. 
\end{itemize}

\subsection*{Finding an upper bound}
Starting with an initial guess of size $n_1$, the prisoners will test if this is an upper bound. 



\section*{Improving Communication}
Let $D=\NN$ be the set of all days. 

Partition $D$ into $d_1,d_2,d_{\dots}$ where: 

\begin{itemize}
	\item The partition $d_{i+1}$ contains every second day not contained in $\cup{j\leq i} d_j$.
	\item Partition $d_1$ contains every second day.
\end{itemize}

Each partition has contains countably infinite days. 

\section*{Partitioning}
To determine the prisoner count, we attempt to partition the prisoners according to whether or not they can be distinguished from each other. 

Initially we construct a partition $\Omega=\set{X_1,X_2}$ where $X_1$ contains you, and $X_2$ contains everyone else. Given a partition into $X_1,\dots,X_n$, denote $x_i=|X_i|$ and let $x$ be the $n$ dimensional vector: 
\[
	x=\left[ x_1,\dots,x_n \right]^T
\]


We construct the following test: 




% Vector $x$ of partition sizes. 





\section*{Connected Components / Pivots}
\end{document}

