\documentclass[a4paper]{article}
\usepackage[fontsize=9pt]{fontsize}

% Fonts 
\usepackage{opensans}
\usepackage{amsfonts}
\usepackage{montserrat}
\usepackage{amsmath}

\usepackage[mathrm=sym]{unicode-math}

\setmainfont{opensans}
\setmathfont{Fira Math}
\setmathfont{TeX Gyre DejaVu Math}[range={\vdots,\ddots}]
\setmathfont{Fira Math}[range=]

\newfontfamily{\montserrateb}{Montserrat SemiBold}
\newfontfamily{\montserratb}{Montserrat Bold}
\newfontfamily{\montserrat}{Montserrat Regular}
\newfontfamily{\montserratl}{Montserrat Light}
\DeclareMathAlphabet{\mathcal}{OMS}{cmbrs}{m}{n}

% \autoref
\usepackage{hyperref}

% Use for [H] option for figures to force in text placement
\usepackage{float}

% Captioning figures
\usepackage{caption}

% Subfigures
\usepackage{subcaption}

% For extending contents beyond margins
\usepackage{scrextend}

% For tables \midrule ect
\usepackage{booktabs}

% Colours
\usepackage[table,xcdraw]{xcolor}
\definecolor{accentcolor}{HTML}{6332a8}

% Change label in enumerate 
\usepackage{enumitem}

% Section settings
\usepackage{titlesec}
% \titleformat{\section}
% {\LARGE\montserrateb}
% {\thesection.}{0.5em}{}

% \titleformat{\subsection}
% {\large\montserratb}
% {\thesubsection.}{0.0em}{}

% Adjust document dimensions
\ExecuteOptions{a4paper}
\addtolength{\oddsidemargin}{-3cm}
\addtolength{\evensidemargin}{-3cm}
\addtolength{\topmargin}{-3cm}
\addtolength{\textwidth}{6cm}
\addtolength{\textheight}{4.5cm}
\addtolength{\textheight}{1.5cm}
\addtolength{\headsep}{-0.5cm}
% \addtolength{\footskip}{-1cm}
\parindent0pt
\parskip=2pt



\usepackage{matlab-prettifier}
\usepackage{graphicx}
\usepackage{mdframed}

% Creates coloured title box
\newcommand{\thetop}[5]{
	\begin{addmargin}[\oddsidemargin]{\oddsidemargin}
		\colorbox{#5}{\color{white}
			\hbox to \paperwidth{
				\vbox {
					\begin{center}
						{\large\montserratl #1}\\
						\vspace{4pt}
						{\huge\montserratb #2}\\
						{\montserratb #3}\\
						\vspace{-0.5em}
						\rule{20em}{1pt}

						{\large\montserratl
							#4
						}
					\end{center}
				}
			}
		}
	\end{addmargin}
}

\newcommand{\NN}{\mathbb{N}}
\newcommand{\ZZ}{\mathbb{Z}}
\newcommand{\RR}{\mathbb{R}}
\newcommand{\CC}{\mathbb{C}}
\newcommand{\dydt}{\frac{dy}{dt}}
\newcommand{\dxdt}{\frac{dx}{dt}}
\def\set#1{\left\{ #1 \right\}}
\def\eval#1#2{\left\ #1\right|_{#2}}

\def\pp#1#2{\frac{\partial #1}{\partial #2}}
\def\dd#1#2{\frac{\,d#1}{\,d#2}}

\def\norm#1{\left\|#1\right\|}

\usepackage{amsthm}
\newtheorem{theorem}{Theorem}
\newtheorem{corollary}[theorem]{Corollary}
\newtheorem{lemma}[theorem]{Lemma}
\newtheorem{proposition}[theorem]{Proposition}
\newtheorem{obs}[theorem]{Observation}
\newtheorem{fact}[theorem]{Fact}
\newtheorem{conjecture}[theorem]{Conjecture}
\newtheorem{question}[theorem]{Question}
\newtheorem{claim}[theorem]{Claim}

\newtheorem*{exercise}{Exercise}

\newtheorem*{claim*}{Claim}
\newtheorem*{proof*}{Proof}
\newtheorem*{example*}{Example}
\newtheorem*{lemma*}{Lemma}
\newtheorem*{theorem*}{Theorem}
\newtheorem*{proposition*}{Proposition}


\newenvironment{conj}[1][Conjecture.]{\begin{trivlist}
\item[\hskip \labelsep {\bfseries #1}]}{\end{trivlist}}
\newenvironment{solution}[1][Solution.]{\begin{trivlist}
\item[\hskip \labelsep {\bfseries #1}]}{\end{trivlist}}
\newenvironment{construct}[1][Construction.]{\begin{trivlist}
\item[\hskip \labelsep {\bfseries #1}]}{\end{trivlist}}
\newenvironment{definition}[1][Definition.]{\begin{trivlist}
\item[\hskip \labelsep {\bfseries #1}]}{\end{trivlist}}
\newenvironment{defn}[1][Definition.]{\begin{trivlist}
\item[\hskip \labelsep {\bfseries #1}]}{\end{trivlist}}
\newenvironment{thm}[1][Theorem.]{\begin{trivlist}
\item[\hskip \labelsep {\bfseries #1}]}{\end{trivlist}}
\newenvironment{exmps}[1][Examples.]{\begin{trivlist}
\item[\hskip \labelsep {\bfseries #1}]}{\end{trivlist}}
\newenvironment{example}[1][Example.]{\begin{trivlist}
\item[\hskip \labelsep {\bfseries #1}]}{\end{trivlist}}
\newenvironment{exmp}[1][Examples.]{\begin{trivlist}
\item[\hskip \labelsep {\bfseries #1}]}{\end{trivlist}}
\newenvironment{remark}[1][Remark.]{\begin{trivlist}
\item[\hskip \labelsep {\bfseries #1}]}{\end{trivlist}}
\newenvironment{problem}[1][Problem.]{\begin{trivlist}
\item[\hskip \labelsep {\bfseries #1}]}{\end{trivlist}}
\newenvironment{prop}[1][Proposition.]{\begin{trivlist}
\item[\hskip \labelsep {\bfseries #1}]}{\end{trivlist}}
\newenvironment{quest}[1][Question.]{\begin{trivlist}
\item[\hskip \labelsep {\bfseries #1}]}{\end{trivlist}}
\newenvironment{answer}[1][Answer.]{\begin{trivlist}
\item[\hskip \labelsep {\bfseries #1}]}{\end{trivlist}}
\newenvironment{algorithm}[1][Algorithm.]{\begin{trivlist}
\item[\hskip \labelsep {\bfseries #1}]}{\end{trivlist}}
\newenvironment{weakness}[1][Weaknesses.]{\begin{trivlist}
\item[\hskip \labelsep {\bfseries #1}]}{\end{trivlist}}
\newenvironment{puzzle}[1][Puzzle.]{\begin{trivlist}
\item[\hskip \labelsep {\bfseries #1}]}{\end{trivlist}}

\usepackage{titlesec}
\titleformat{\section}
{\centering\montserrateb\MakeUppercase}
{\thesection.}{0em}{}
\titlespacing{\section}{0em}{0em}{0em}

\titleformat{\subsection}
{\centering\montserratb}
{\thesection.}{0em}{}
\titlespacing{\subsection}{0em}{0em}{0em}

\usepackage{centernot}
\usepackage{mathtools}
\usepackage{stmaryrd}

% \makeatletter
% \newcommand{\xMapsto}[2][]{\ext@arrow 0599{\Mapstofill@}{#1}{#2}}
% \def\Mapstofill@{\arrowfill@{\Mapstochar\Relbar}\Relbar\Rightarrow}
% \makeatother


\def\curl{\operatorname{curl}}

\usepackage{multicol}

\begin{document}

\begin{multicols}{2}
	\section*{Taylor Polynomials}
	\begin{definition}
		For a sufficiently smooth function $f:\RR\to\RR$, the $K$-th degree Taylor Polynomial of $f(x)$ centred about $a$ is given by:
		$$P_k(x)=\sum_{j=0}^{k}\frac{f^{(j)}(a)}{j!}(x-a)$$
	\end{definition}

	\begin{example}
		Calculate the linear approximation to $f(x)=\sin(x)$ centred at $a=0$.
		\begin{align*}
			P_0(x) & =f(0)=0            \\
			P_1(x) & =f(0)+f'(0)(x-0)=x
		\end{align*}
	\end{example}

	\begin{definition}
		Let $S$ be a surface given by $f(x,y,z)=0$ where $f:\RR^3\to\RR$ is continuously differentiable. Then the degree one Taylor Polynomial at $(a,b,c)$ is:
		$$
			f(a,b,c)+\begin{bmatrix}
				\frac{\partial f}{ \partial x}(a,b,c) &
				\frac{\partial f}{ \partial y}(a,b,c) &
				\frac{\partial f}{ \partial z}(a,b,c)
			\end{bmatrix} \cdot \begin{bmatrix}
				x-a \\ y-b \\ z-c
			\end{bmatrix}
		$$

		When $(a,b,c)$ is on the surface $S$, then the tangent plane at $(a,b,c)$ is:
		$$
			\frac{\partial f}{ \partial x}(a,b,c)(x-a) +
			\frac{\partial f}{ \partial y}(a,b,c)(y-b) +
			\frac{\partial f}{ \partial z}(a,b,c)(z-c)
		$$

		Where $Df(\mathbf{a})$ is the \textbf{total derivative} (or jacobian) of $f$ at $\textbf{a}$.
		\[
			Df(a)=\begin{bmatrix}
				\pp f{x_1}(\textbf{a}),\dots,\pp f{x_m}(\textbf{a})
			\end{bmatrix}
		\]
		And $Hf(\textbf{a})$ is the \textbf{Hessian} of $f$ at \textbf{a}, a matrix with:
		\[
			[Hf(a)]_{i,j} = \frac{\partial^2f}{\partial x_i \partial x_j}(\textbf{a})
		\]


	\end{definition}

	\begin{definition}
		For $f:\RR^n\to\RR$, the degree $1,2$ Taylor polynomials centred at $a$ are:
		\begin{align*}
			P_1(\mathbf r) & = f(\mathbf a) Df(\mathbf a)(\mathbf r-\mathbf a)                                                                      \\
			P_2(\mathbf r) & = f(\mathbf a) Df(\mathbf a)(\mathbf r-\mathbf a) + \frac 12(\mathbf r-\mathbf a)^T Hf(\mathbf a)(\mathbf r-\mathbf a) \\
		\end{align*}
	\end{definition}


	\section*{Vector Derivatives}
	\begin{definition}
		The derivative of $r(t)$ where $r:\RR\to\RR^n$ with respect to $t$ is done component wise:
		$$r'(t)=\begin{bmatrix}
				r_1'(t) \\
				\vdots  \\
				r_n'(t)
			\end{bmatrix}$$
	\end{definition}


	% Exercise
	\begin{example}
		For functions $u,v,w: \RR\to\RR^n$ and $f:\RR\to\RR$ with $\alpha,\beta\in\RR$:
		\begin{align*}
			\alpha u + \beta v & \qquad \alpha u' + \beta v'                                       \\
			u\cdot v           & \qquad u'\cdot v + u\cdot v'                                      \\
			u\times v          & \qquad u'\times v + u\times v'                                    \\
			u\cdot(v\times w)  & \qquad u'\cdot v\times w   +u\cdot v'\times w + u\cdot v\times w' \\
			u(f(t))            & \qquad f'(t)\dd uf = f'(t)u'(f(t))
		\end{align*}
	\end{example}




	\section*{Curvilinear Coordinates}
	A coordinate parametrisation for a new system $\textbf{u}=[u_1,\dots,u_n]$ from the standard coordinates on $\RR^n$:
	\begin{align*}
		\xi : & \;\;\RR^n\longrightarrow \RR^n         \\
		      & \begin{bmatrix}
			        u_1    \\
			        \vdots \\
			        u_n
		        \end{bmatrix}\mapsto \begin{bmatrix}
			                             \xi_1(\textbf{u}) \\
			                             \vdots            \\
			                             \xi_n(\textbf{u})
		                             \end{bmatrix}
	\end{align*}
	We can find a basis for $\textbf{u}$ from the columns of $D\xi$,
	$$\pp\xi{u_1},\dots,\pp{\xi}{u_n}\qquad e_{u_j}= \pp{\xi}{u_j}\norm{\pp{\xi}{u_j}}^{-1}$$
	Note that when the basis are not linearly independent, our coordinates are not well-behaved. (Cylindrical/polar when $r=0$, Spherical when $x,y=0$.) These points are \textbf{coordinate singularities}.

	\subsection*{Polar}
	\[
		\xi: \begin{bmatrix}
			r \\
			\theta
		\end{bmatrix} \mapsto \begin{bmatrix}
			r\cos\theta \\
			r\sin\theta
		\end{bmatrix}
		\qquad
		\dd{e_r}{\theta}=e_\theta,\quad
		\dd{e_\theta}{\theta}=-e_r
	\]
	$$dA=r\,dr\,d\theta$$

	\subsection*{Cylindrical}
	\[
		\xi: \begin{bmatrix}
			r      \\
			\theta \\
			z
		\end{bmatrix} \mapsto \begin{bmatrix}
			r\cos\theta \\
			r\sin\theta \\
			z
		\end{bmatrix}
		\qquad
		\begin{array}{lll}
			\pp{e_r}{r} =0      & \pp{e_r}{\theta} =e_\theta  & \pp{e_r}{z} =0      \\
			\pp{e_\theta}{r} =0 & \pp{e_\theta}{\theta} =-e_r & \pp{e_\theta}{z} =0 \\
			\pp{e_z}{r} =0      & \pp{e_z}{\theta} =0         & \pp{e_z}{z} =0      \\
		\end{array}
	\]
	$$dA=r\,dr\,d\theta\,dz$$

	\subsection*{Spherical}
	\[
		\xi: \begin{bmatrix}
			r      \\
			\theta \\
			\phi
		\end{bmatrix}
		\mapsto
		\begin{bmatrix}
			r\cos\theta\sin\phi \\
			r\sin\theta\sin\phi \\
			r\cos\phi
		\end{bmatrix}
	\]
	\[
		\begin{array}{lll}
			\pp{e_r}{r}      =0 & \pp{e_r}{\theta} =\sin\phi e_\theta                   & \pp{e_r}{z} =e_\phi \\
			\pp{e_\theta}{r} =0 & \pp{e_\theta}{\theta} =-\sin\phi e_r -\cos\phi e_\phi & \pp{e_\theta}{z} =0 \\
			\pp{e_z}{r}      =0 & \pp{e_z}{\theta} =\cos\phi e_\theta                   & \pp{e_z}{z} =-e_r   \\
		\end{array}
	\]
	\[
		dV = r^2|\sin\phi|dr\,d\theta\,d\phi
	\]


	\section*{Parametrised Curves}
	Continuous map $c:I\to\RR^n$, $I\subseteq\RR$.
	\begin{itemize}
		\item \textbf{Closed curve/loop}, $I=[t_0,t_1]$ and $c(t_0)=c(t_1)$.
		\item \textbf{Simple}, $c$ is injective for non-endpoint $t\in I$.
		\item \textbf{Regular}, differentiable and $\forall_{t\in I}c'(t)\neq 0$.
		\item \textbf{Incremental Arc length}: $ds=\sqrt{c'\cdot c'}dt$
		\item \textbf{Arc length}: $s(t_0,t_1)=\int_{t_0}^{t_1}ds$
		\item \textbf{Path Integral}: $\int_s f \circ c(t)ds $

	\end{itemize}
	\section*{Manifolds}
	Multidimensional integrals are linear wrt integrands, and additive wrt. Regions of integration.
	\begin{theorem*}[Generalised Fubini] Generalises to higher dimensions.
		\begin{align*}
			\iint_Rf(x,y)dA & =\int_{y_1}^{y_2}\left[ \int_{x_1(y)}^{x_2(y)}f(x,y)dx \right]dy \\
			\iint_Rf(x,y)dA & =\int_{x_1}^{x_2}\left[ \int_{y_1(x)}^{y_2(x)}f(x,y)dy \right]dx
		\end{align*}
		Type 1: $y$ from $a$  to $b$, $x$ from $g_1(y)$ to $g_2(y)$.

		Type 2: $x$ from $a$  to $b$, $y$ from $h_1(x)$ to $h_2(x)$.
	\end{theorem*}


	\begin{definition}A \textbf{parametrised $n$-manifold} in $\RR^m$ is a continuous map:
		$$\mathbf{R}:S\subseteq\RR^n\to\RR^m$$
		\begin{itemize}
			\item  If $n=1$ curve, $n=2$ surface, $n=m-1$ hyper-surface.
			\item Input $u_1,\dots,u_n$ are curvilinear  coordinates on the manifold.
			\item Sometimes the image is referred to as the manifold.
			\item Tangent vectors are columns of $$D\mathbf{R}=\left[ \pp{\mathbf{R}}{u_1},\dots,\pp{\mathbf{R}}{u_n} \right]$$
			      If linearly independent gives us moving frame.
			\item If lin-indpt for all $\mathbf{u}$, $\mathbf{R}$ is \textbf{regular} or \textbf{immersed}.
			\item Regularity can be checked using $\operatorname{Null}(D\mathbf{R})=0$ or $n$-volume of paralleletope spanned by tangents.
			\item Calculated with \textbf{Gram determinant}:
			      $$\sqrt{\det ((D\mathbf{R})^T(D\mathbf{R}))}$$
			\item Or $|\det{D\mathbf{R}}|$ when $n=m$.
			\item Multiply this by $du_1,\dots,du_n$ to get $dM$ infinitesimal $n$-volume element of the manifold.
		\end{itemize}

	\end{definition}
	For a surface $R(u,v)$ in $\RR^3$, $dA=\left\|\pp{\mathbf{R}}{u}\times \pp{\mathbf{R}}{v}\right\|$\,du\,dv.
	Otherwise, for a surface in higher dimensions we can use $dA=\left\|\pp{\mathbf{R}}{u}\right\|\left\|\pp{\mathbf{R}}{v}\right\|\sin\theta\,du\,dv$.


	\section*{Vector Fields}

	\subsection*{Simply Connected Regions}
	A region $U\subseteq \RR^n$ is \textbf{simply connected} if:
	\begin{itemize}
		\item $U$ is path connected, any two points can be joined by a continuous path in $U$.
		\item Any loop in $U$ can be continuously contracted to a point while remaining in $U$.
	\end{itemize}



	\subsection*{Gradient}

	For $f:\RR^n\to\RR$, the total derivative matrix is a row vector and the gradient of $f$ is:
	$$\operatorname{grad} f= \nabla f= (Df)^T$$
	For $\mathbf{F}:U\subseteq\RR^n\to\RR^n$ a scalar potential  $f:U\to\RR$ has $\nabla f=\mathbf{F}$. Only \textbf{conservative} vector fields have potential functions.

	Necessary condition for conservative $(\forall_{i\neq j})$	$\pp{F_i}{x_j}=\pp{F_j}{x_i}$

	To find a potential, $f=\int F_1dx$, this has an arbitrary function $G(y)$ instead of a constant of integration. Try to solve $F_2=\pp{f}y$ for $G'(y)$, if possible find $\int G'(y)dy$.

	The \textbf{path integral} of a vector field $\mathbf{F}$ along and \textbf{oriented} curve $C$ parametrised by $\mathbf{r}:[t_0,t_1]\to\RR^n$ is:
	\[
		\int_C F_1dx_1+\dots+F_ndx_n=\int_CF\cdot d\mathbf{r}=\int_{t_0}^{t_1}\mathbf{F}(\mathbf{r}(t))\cdot \mathbf{r}'(t)dt
	\]
	This can be turned into a path integral of a scalar function with respect to arc length. (Note that curve orientation matters unlike with path integrals of scalar fields).

	The \textbf{FTC for line integrals}: Let $C$ be a (piecewise) smooth curve in $\RR^n$ with parametrisation $\mathbf{r}:[t_0,t_1]\to\RR^n$, and $f$ a continuously differentiable function.
	$$\int_C\nabla f \cdot d\mathbf{r}=f(\mathbf{r}(t_1))-f(\mathbf{r}(t_0))$$

	If $\textbf{F}$ is a vector field with continuous partial derivatives defined on an open simply conected region $U\subseteq\RR^n$. Then $\mathbf{F}$ is conservative if and only if:
	$$(\forall_{i\neq j})\pp{F_i}{x_j}=\pp{F_j}{x_i}$$


	\subsection*{Curl}
	For $\mathbf{F}(x,yz)$ a vector field on $\RR^3$. The \textbf{curl} of $\mathbf{F}$ is:
	$$
		\curl \mathbf{F}=\nabla\times \mathbf{F}=\begin{bmatrix}
			\pp{\mathbf{F}_3}{y}-\pp{\mathbf{F}_2}{z} \\
			\pp{\mathbf{F}_1}{z}-\pp{\mathbf{F}_3}{x} \\
			\pp{\mathbf{F}_2}{x}-\pp{\mathbf{F}_1}{y} \\
		\end{bmatrix}
	$$
	If $\curl\mathbf{F}=0$, then $\mathbf{F}$ is \textbf{curl-free} or \textbf{irrotational}. The curl of a gradient is zero $0=\curl(\nabla f)$ where $f$ is a scalar function.

	Give a vector field $\mathbf{v}(x,y,z)$ defined on a region $U\subseteq\RR^3$, a \textbf{vector potential} for $\mathbf{v}$ is a vector field $\mathbf{F}$ such that $\mathbf{v}=\curl \mathbf{F}$.

	A vector potential is determined up to an arbitrary conservative vector field since $\curl(\mathbf{F}+\nabla f)=\curl \mathbf{F} +0$

	\subsection*{Divergence}

	\subsection*{Green's Theorem}
	For $R$, a closed bounded region in $\RR^2$ with boundary $\partial R$ consisting of piecewise smooth curves.
	\begin{itemize}
		\item 	The \textbf{induced positive orientation} is given by orienting all boundary curves such that $R$ lies to the left of the curve direction.
		\item Green's theorem (Like FC for double integrals): Let $\mathbf{F}$ be a smooth vector field, then:
		      $$\iint_R\left( \pp{F_2}x - \pp{F_1}{y}dxdy \right)=\oint_{\partial R}F_1dx +F_2dy$$
	\end{itemize}
	We can use an $\mathbf{F}$ with $\pp{F_2}x - \pp{F_1}{y}=1$ to compute areas.

	\subsection*{Stokes' Theorem}

	A surface $S$ in $\RR^3$ is \textbf{orientable} if we can find a continuous unit normal vector. If we have a parametrisation $r(u,v)$, then normalising $\partial{r}u\times\partial{r}v$ gives us an orientation.

	If the surface has boundaries $\partial S$ consisting of piecewise smooth curves, then choosing a normal vector induces an orientation on the boundary. Walking along the boundary oriented head in the direction of $n$, the surface should be on your left.

	A surface is closed if it is compact (closed and bounded), and $\partial S=\emptyset$. Orientation is canonically given by orienting $n$ to point outwards.

	The surface integral of a vector field $\mathbf{F}$ on an oriented surface $S$ parametrised by $\mathbf{r}:U\to\RR^3$, with a unit normal vector field $\mathbf{n}=\frac{T_u\times T_v}{\|T_u\times T_v\|}$ is given by:
	\[
		\iint_S \mathbf{F}\cdot d\mathbf{S}=
		\iint_S \mathbf{F}\cdot \mathbf{n}dA=
		\iint_U \mathbf{F}(\mathbf{r})\cdot T_u\times T_v \,du\,dv
	\]
	The integrand $\mathbf{F}\cdot d\mathbf{S}$ is the \textbf{flux density}, the integral is the
	\textbf{total flux} of $\mathbf{F}$ across $\mathbf{S}$.

	\textbf{Stokes' theorem} for $S$ a compact oriented surface in $\RR^3$, where $\partial S$ consists of piecewise smooth curves with induced positive orientations. Then for any smooth vector field $\mathbf{F}$:
	\[
		\iint_S \curl \mathbf{F}\cdot d\mathbf{S}=\oint_{\partial S}F_1dx+F_2dy+F_3dz
	\]
	The theorem still applies when $\partial S=\emptyset$. Integral is $0$.

\end{multicols}
\end{document}
