
\documentclass[a4paper, 11pt]{article}

% Fonts 
\usepackage{opensans}
\usepackage{amsfonts}
\usepackage{montserrat}
\usepackage{amsmath}

\usepackage[mathrm=sym]{unicode-math}

\setmainfont{opensans}
\setmathfont{Fira Math}

\newfontfamily{\montserrateb}{Montserrat SemiBold}
\newfontfamily{\montserratb}{Montserrat Bold}
\newfontfamily{\montserrat}{Montserrat Regular}
\newfontfamily{\montserratl}{Montserrat Light}
\DeclareMathAlphabet{\mathcal}{OMS}{cmbrs}{m}{n}

% \autoref
\usepackage{hyperref}

% Use for [H] option for figures to force in text placement
\usepackage{float}

% Captioning figures
\usepackage{caption}

% Subfigures
\usepackage{subcaption}

% For extending contents beyond margins
\usepackage{scrextend}

% For tables \midrule ect
\usepackage{booktabs}

% Colours
\usepackage[table,xcdraw]{xcolor}
\definecolor{accentcolor}{HTML}{6332a8}

% Change label in enumerate 
\usepackage{enumitem}

% Section settings
\usepackage{titlesec}
\titleformat{\section}
{\LARGE\montserrateb}
{\thesection.}{0.5em}{}

\titleformat{\subsection}
{\large\montserratb}
{\thesubsection.}{0.5em}{}

% Adjust document dimensions
\ExecuteOptions{a4paper}
\addtolength{\oddsidemargin}{-3cm}
\addtolength{\evensidemargin}{-3cm}
\addtolength{\topmargin}{-3cm}
\addtolength{\textwidth}{6cm}
\addtolength{\textheight}{4.5cm}
\addtolength{\textheight}{1.5cm}
\addtolength{\headsep}{-0.5cm}
% \addtolength{\footskip}{-1cm}
\parindent0pt
\parskip=4pt



\usepackage{matlab-prettifier}
\usepackage{graphicx}
\usepackage{mdframed}

% Creates coloured title box
\newcommand{\thetop}[5]{
	\begin{addmargin}[\oddsidemargin]{\oddsidemargin}
		\colorbox{#5}{\color{white}
			\hbox to \paperwidth{
				\vbox {
					\begin{center}
						{\large\montserratl #1}\\
						\vspace{4pt}
						{\huge\montserratb #2}\\
						{\montserratb #3}\\
						\vspace{-0.5em}
						\rule{20em}{1pt}

						{\large\montserratl
							#4
						}
					\end{center}
				}
			}
		}
	\end{addmargin}
}

\newcommand{\NN}{\mathbb{N}}
\newcommand{\ZZ}{\mathbb{Z}}
\newcommand{\RR}{\mathbb{R}}
\newcommand{\CC}{\mathbb{C}}
\newcommand{\dydt}{\frac{dy}{dt}}
\newcommand{\dxdt}{\frac{dx}{dt}}
\def\set#1{\left\{ #1 \right\}}
\def\eval#1#2{\left\ #1\right|_{#2}}

\def\pp#1#2{\frac{\partial #1}{\partial #2}}
\def\dd#1#2{\frac{\,d#1}{\,d#2}}

\begin{document}
\thetop{Robert Christie}{MATHS 340}{S2 2023}{Assignment 2\\Due: 22-08-2023}{accentcolor}

\section*{Q1}
For each question, we apply Fubini's theorem to express to iterated integrals as triple integrals over region defined by their bounds. We then rewrite the bounds though a series of equivalent sets of inequalities. Lastly applying Fubini's theorem again to get iterated integrals in the correct order.

\begin{enumerate}[label=(\alph*)]
	\item
	      \begin{align*}
		      \begin{cases}
			      0 \leq z\leq 1     \\
			      0 \leq x\leq \ln 2 \\
			      e^x \leq y\leq 2   \\
		      \end{cases}
		      \iff
		      \begin{cases}
			      0 \leq z\leq 1          \\
			      0 \leq x\leq \ln 2      \\
			      x \leq \ln y \leq \ln 2 \\
			      y \leq 2                \\
		      \end{cases}
		      \iff
		      \begin{cases}
			      0 \leq z\leq 1            \\
			      0 \leq x\leq \ln 2        \\
			      x \leq \ln y \leq \ln 2   \\
			      e^0\leq e^x \leq y \leq 2 \\
		      \end{cases}
		      \iff
		      \begin{cases}
			      0 \leq z\leq 1     \\
			      1 \leq y \leq 2    \\
			      0\leq x \leq \ln y \\
		      \end{cases}
	      \end{align*}
	      \[
		      \therefore   \int_0^1\int_0^{\ln 2}\int_{e^x}^2f(x,y,z)\,dy\,dx\,dz    =\int_0^1\int_1^2\int_0^{\ln y} f(x,y,z) \,dx\,dy\,dz
	      \]
	\item
	      \begin{align*}
		       & \begin{cases}
			         0 \leq x\leq 2                      \\
			         0 \leq z\leq 3-\frac 32 x           \\
			         0 \leq y\leq 5-\frac 53z -\frac 52x \\
		         \end{cases}
		      \iff
		      \begin{cases}
			      0 \leq x                        \\
			      x\leq 2                         \\
			      0 \leq z                        \\
			      z\leq 3-\frac 32 x              \\
			      0 \leq y 5-\frac 53z -\frac 52x \\
		      \end{cases}
		      \iff
		      \begin{cases}
			      0 \leq z                             \\
			      z\leq 3                              \\
			      0 \leq x                             \\
			      x \leq 2 -\frac 23 z                 \\
			      0 \leq y \leq 5-\frac 53z -\frac 52x \\
		      \end{cases}   \\
		      \iff
		       & \begin{cases}
			         0 \leq z\leq 3                 \\
			         0 \leq x \leq 2 -\frac 23 z    \\
			         0 \leq y                       \\
			         y\leq 5-\frac 53z              \\
			         x\leq 2-\frac 23z  - \frac 25y \\
		         \end{cases}
		      \iff
		      \begin{cases}
			      0 \leq z \leq 3                       \\
			      0 \leq y \leq 5-\frac 53z             \\
			      0 \leq x \leq 2-\frac 23z  -\frac 25y \\
		      \end{cases}
	      \end{align*}
	      \[
		      \therefore   \int_0^2\int_0^{3-\frac 32x}\int_{0}^{5-\frac53z-\frac 52x}f(x,y,z)\,dy\,dz\,dx    =\int_0^3\int_0^{5-\frac 53z}\int_0^{2-\frac 23z-\frac 25y} f(x,y,z) \,dx\,dy\,dz
	      \]
	\item
	      \begin{align*}
		      \begin{cases}
			      0 \leq x\leq 1                                              \\
			      0 \leq y\leq 1                                              \\
			      x^2 \leq z\leq x\iff \sqrt{x^2} \leq \sqrt z \land z \leq x \\
		      \end{cases}
		      \iff
		      \begin{cases}
			      0 \leq z\leq 1       \\
			      0 \leq y\leq 1       \\
			      z \leq x\leq \sqrt z \\
		      \end{cases}
	      \end{align*}
	      \[
		      \therefore   \int_0^1\int_0^1\int_{x^2}^xf(x,y,z)\,dz\,dy\,dx    =\int_0^1\int_0^1\int_z^{\sqrt z} f(x,y,z) \,dx\,dy\,dz
	      \]

\end{enumerate}
\pagebreak
\section*{Q2}

\begin{enumerate}[label=(\alph*)]
	\item First we find parametric integrals for the surface area of each component body:
	      \begin{itemize}
		      \item \textbf{The square prism} $R_\text{Prism}$
		            \begin{mdframed}
			            6 square sides that can be parametrised in terms of $u\times v \in[0,4]\times [0,4]$ in the form:
			            $$
				            r(u,v)=ue_u+ve_v +r_0
			            $$
			            Where $\set{e_u,e_v}\subset\set{e_x,e_y,e_z}$ and $r_0\in\RR^3$, thus the area element:
			            \begin{align*}
				            dA & =\left\|\pp ru \times \pp rv\right\|\,du\,dv \\
				               & =\left\|e_u\times e_v \right\|      \,du\,dv \\
				               & =\,du\,dv
			            \end{align*}
			            Integrating over each parametrised surface gives:
			            $$A_\text{Face}=\iint_{R_\text{Face}}1\,dA =  \int_0^4\int_0^4 1\,du\,dv$$

			            To connect the prism and tube, the top face must have a cutout, the surface to be removed can be parametrised in terms of $u\times v\in[0,1]\times [0,2\pi]$:
			            $$r(u,v)=\begin{bmatrix}
					            u\cos v \\u\sin v\\0
				            \end{bmatrix}
			            $$
			            Giving an area element:
			            \begin{align*}
				            dA & = \left\|\pp ru \times \pp rv\right\|\,du\,dv
				            = \left\|
				            \begin{bmatrix}
					            \cos v \\
					            \sin v \\
					            0
				            \end{bmatrix}
				            \times
				            \begin{bmatrix}
					            -u\sin v \\
					            u\cos v  \\
					            0
				            \end{bmatrix}
				            \right\|
				            =
				            u \,du \,dv                                        \\
			            \end{align*}
			            Thus the surface integral over the removed section is:
			            \[
				            \int_0^{2\pi} \int_0^1u\,du\,dv
			            \]

			            Hence, the total area of the square prism is given by:
			            $$A_\text{Prism}=6\int_0^4\int_0^4 1\,du\,dv-\int_0^{2\pi} \int_0^1u\,du\,dv$$

		            \end{mdframed}

		            \pagebreak
		      \item \textbf{Tube} $R_\text{Tube}$
		            \begin{mdframed}
			            We parametrise the surface in terms of $u\times v\in[0,1]\times[0,2\pi]$:
			            $$r(u,v)=\begin{bmatrix}
					            e^u \cos v \\
					            e^u \sin v \\
					            u
				            \end{bmatrix}
			            $$
			            Thus the area element is given by:

			            \begin{align*}
				            dA & = \left\|\pp ru \times \pp rv\right\| \,du\,dv  = \left\|
				            \begin{bmatrix}
					            e^u\cos v \\ e^u\sin v \\ 1
				            \end{bmatrix}\times
				            \begin{bmatrix}
					            -e^u\sin v \\
					            e^u\cos v  \\
					            0
				            \end{bmatrix}
				            \right\| \,du\,dv                                                      \\
				            =  & \left\|
				            \begin{matrix}
					            -e^u\cos v \\ -e^u\sin v\\  e^{2u}\cos^2v +  e^{2u}\sin^2v
				            \end{matrix}
				            \right\| \,du \,dv
				            =\left\|
				            \begin{matrix}
					            -e^u\cos v \\ -e^u\sin v\\  e^{2u}
				            \end{matrix}
				            \right\| \,du \,dv                                                     \\
				               & =\sqrt{e^{2u} + e^4u}\,du\,dv = \sqrt{e^u\sqrt{1+e^{2u}}}\,du\,dv
			            \end{align*}
			            Integrating over the surface of the tube:
			            \begin{align*}
				            A_\text{Tube} & =  \iint_{R_\text{Tube}} 1 \,dA                          \\
				                          & =  \int_0^{2\pi}\int_0^1 e^{u}\sqrt{1+e^{2u}} \,du \, dv \\
			            \end{align*}

		            \end{mdframed}
		            %   \item Using cylindrical coordinates, where the area element is
		            %         $\text{dA}=r\text{d$r$}\text{d$z$}\text{d$\theta$}$ The tube wall has a surface area of:
		            %         \[
		            %             \int_0^{2\pi} \int_0^1 e^z \text{d$z$}\text{d$\theta$}
		            %         \]
		            %         \textcolor{red}{Do we need to account for $dr$ in the area
		            %             element, making the integrand $e^{2z}$}

		            \pagebreak
		      \item \textbf{Half sphere} $R_\text{H-sphere}$
		            \begin{mdframed}
			            We can parametrise the curved surface of the half sphere in terms of $u\times v\in [0,\frac\pi2]\times[0,2\pi]$:
			            $$r(u,v)=\begin{bmatrix}
					            4\sin u\cos v \\
					            4\sin u\sin v \\
					            1+4\cos u     \\
				            \end{bmatrix}$$

			            Note that:
			            $$\|a\times b\| = \|a\|\|b\||\sin\theta|=\sqrt{ \|a\|^2\|b\|^2\sin^2\theta}=\sqrt{ \|a\|^2\|b\|^2(1-\cos^2\theta)}=\sqrt{(a\cdot a)(b\cdot b)-(a\cdot b)^2}$$

			            The area element is:

			            \begin{align*}
				            dA & =\left\|\pp ru\times \pp rv\right\|
				            = 16\left\|
				            \begin{bmatrix}
					            \cos u \cos v \\
					            \cos u \sin v \\
					            -\sin u       \\
				            \end{bmatrix}
				            \times
				            \begin{bmatrix}
					            -\sin u\sin v \\
					            \sin u\cos v  \\
					            0             \\
				            \end{bmatrix}
				            \right\|                                                        \\
				               & =16\sqrt{(\sin^2 u + \cos^2u)(\sin^2)-(\sin^2u+\cos^2u)^2} \\
				               & =16\sqrt{\sin^2u-1}                                        \\
				               & =16\cos u
			            \end{align*}
			            %        & =\left\|
			            %     \begin{matrix}
			            %         \sin^2 u\cos v                                 \\
			            %         \sin^2 u\sin v                                 \\
			            %         \sin u \cos u \cos^2 v + \sin u\cos u \sin^2 v \\
			            %     \end{matrix}
			            %     \right\|
			            %     = \left\|
			            %     \begin{matrix}
			            %         \sin^2 u\cos v \\
			            %         \sin^2 u\sin v \\
			            %         2\sin u \cos u
			            %     \end{matrix}
			            %     \right\|                                                                                    \\
			            %        & =\sqrt{\sin^4 u \cos^2 v + \sin^4u\sin^2 v + 4 \cos^2u\sin^2u}                         \\
			            %        & =\sqrt{\sin^2 u\sin^2 u + 4 \cos^2u\sin^2u}=\sqrt{\sin^2 u\sin^2 u + 4 \cos^2u\sin^2u}
			            % \end{align*}


			            % cos(u)*cos(v)sin(u)*cos(v) - sin(v)*cos(u)sin(u)*sin(v)
			            % cos(u)sin(u)cos(v)cos(v) + cos(u)sin(u)sin(v)sin(v)
			            % cos(u)sin(u) + cos(u)sin(u)
			            % cos(u)sin(u) + cos(u)sin(u)
			            % 2cos(u)sin(u) 
			            % 4cos^2(u)sin^2(u) 

			            Integrating over the surface gives:
			            \[
				            \int_0^{2\pi}\int_0^{\frac{\pi}2} 16\cos u \,du\,dv
			            \]

			            The bottom of the sphere is a region of the $z=1$ plane between the tubes top at $(r,z)=(e,1)$, and the bottom of the sphere at $(r,z)=(4,1)$. Parametrising this surface in terms of $u\times v =[e,4]\times [0,2\pi]$:
			            $$r(u,v)=\begin{bmatrix}
					            u\cos v \\
					            u\sin v \\
					            1
				            \end{bmatrix}$$
			            Note that $\pp ru$ and $\pp rv$ are the same as for the parametrisation of the region removed from the prism so the area element is the same $dA=u\,du\,dv$. Thus, the area of the bottom of the half sphere is.
			            \[
				            \int_0^{2\pi}\int_e^4 u\,du,dv
			            \]
			            The union of these two non-overlapping parametrised surfaces gives the half sphere, thus we integrate over both surfaces:
			            \[
				            A_\text{H-sphere}=\iint_{R_\text{H-sphere}} =  \int_0^{2\pi}\int_0^{\frac{\pi}2} 16\cos u \,du\,dv +\int_0^{2\pi}\int_e^4 u\,du,dv
			            \]
		            \end{mdframed}
	      \end{itemize}
	      Thus the total area of $R$ is:
	      \begin{align*}
		      A & =A_\text{Prism}+A_\text{Tube}+A_\text{H-sphere}                                           \\
		        & = 6\int\limits_0^4\int\limits_0^4 1\,du\,dv-\int\limits_0^{2\pi} \int\limits_0^1u\,du\,dv
		      + \int\limits_0^{2\pi}\int\limits_0^1 e^{u}\sqrt{1+e^{2u}} \,du \, dv
		      + \int\limits_0^{2\pi}\int\limits_0^{\frac{\pi}2} 16\cos u \,du\,dv +\int\limits_0^{2\pi}\int\limits_e^4 u\,du\,dv
	      \end{align*}

	      %   \[
	      %       A =
	      %       6\int_0^4\int_0^{4} 1 \,du\,dv
	      %       +\int_0^{2\pi}\int_0^1 e^{u}\sqrt{1+e^{2u}} \,du \,dv
	      %       -\int_0^1\int_0^{2\pi} r \,d\theta\,dr
	      %       +\int_e^4\int_0^{2\pi} r\,d\theta\,dr
	      %       +\int_0^{\frac{\pi}2} \int_0^{2\pi}4^2\sin\phi \,d\theta\, d\phi
	      %   \]

	      \pagebreak
	\item Now consider the volume of the body $R$, we split $R$ into three sections and integrate over each of them. Note that for all three we use the parametrisation:
	      $$r(u,v,w)=\begin{bmatrix}
			      u \\v\\w
		      \end{bmatrix}$$
	      Thus the volume element is $dV=dx\,dy\,dz$ and the bounds for the parametrisation are simply the bounds of the body.
	      \begin{itemize}
		      \item Integrating $\rho$ over the prism:
		            \[
			            M_\text{Prism}=\int_{-4}^0\int_{-2}^2\int_{-2}^2\rho(u,v,w)\,d u \,d v \,d w
		            \]
		      \item For the bounds of the tube are given by:
		            $$\begin{cases}
				            0\leq w\leq 1                                                                  \\
				            -e^w\leq v \leq e^w                             & \quad\text{As $y\in[-r,r]$}  \\
				            -\sqrt{e^{2w}-v^2}\leq u \leq \sqrt{e^{2w}-v^2} & \quad\text{As $x^2+y^2=r^2$}
			            \end{cases}$$

		            Giving a parametric integral:
		            $$M_\text{Tube}=\int_0^1\int_{-e^w}^{e^w}\int_{-\sqrt{e^{2w}-v^2}}^{\sqrt{e^{2w}-v^2}}\rho(u,v,w)\,d u \,d v \,d w$$

		      \item For the sphere we have:
		            \[
			            \begin{cases}
				            1\leq w \leq 1+4                                          & \quad\text{Bounded below by $z=1$ and limited by $r=4$ } \\
				            -\sqrt{4^2-(w-1)^2}\leq v \leq \sqrt{4^2-(w-1)^2}         & \ \quad\text{Limited by $x^2+y^2+z^2\leq r$}             \\
				            -\sqrt{4^2-v^2-(w-1)^2}\leq u \leq \sqrt{4^2-v^2-(w-1)^2} & \quad\text{Limited by $x^2+y^2+z^2\leq r$}               \\
			            \end{cases}
		            \]
		            Integrating $\rho$ within these bounds gives the mass of the half sphere section:
		            \[
			            \int_{1}^5\int_{-4}^4\int_{-\sqrt{4^2-v^2-w^2}}^{\sqrt{4^2-v^2-w^2}}\rho(u,v,w) \,d u\,d v\,d w
		            \]
	      \end{itemize}
	      The union of these non-overlapping regions gives is body $R$, so the sum of the parametric integrals for each region gives the integral of $\rho$ over $R$:
	      \begin{align*}
		      M & = M_\text{Prism} + M_\text{Tube} + M_\text{H-sphere}                                                 \\
		        & =\int_{-4}^0\int_{-2}^2\int_{-2}^2\rho(u,v,w)\,d u \,d v \,d w                                       \\
		        & + \int_0^1\int_{-e^w}^{e^w}\int_{-\sqrt{e^{2w}-v^2}}^{\sqrt{e^{2w}-v^2}}\rho(u,v,w)\,d u \,d v \,d w \\
		        & + \int_{1}^5\int_{-4}^4\int_{-\sqrt{4^2-v^2-w^2}}^{\sqrt{4^2-v^2-w^2}}\rho(u,v,w) \,d u\,d v\,d w
	      \end{align*}
	      Where $\rho\left(x,y,z\right)=x^{2}\cos^{2}\left(y\right)\left(10-z\right)$ as given in the question.
\end{enumerate}

\pagebreak
\section*{Q3}
\begin{enumerate}[label=(\alph*)]

	\item
	      Partition $R$ into two disjoint regions ($R=R_1\cup R_2$ and $R_1\cap R_2=\emptyset$):
	      $$R_1=\set{(x,y) \;|\; y<\alpha x\;\alpha\in\RR}\qquad R_2=\set{(x,y) \;|\; y>\alpha x\;\alpha\in\RR}$$
	      Each of these regions is simply connected as they are convex subsets of $\RR^2$ --- any two paths between the same end points and be transformed between continuously. We can also see that the regions are simply connected as any loop can be contracted to a point. We see that:
	      \begin{align*}
		        & \pp {F_y}x - \pp {F_x}y                                                                \\
		      = & \pp{}x \left[ \frac {2y}{x^2+y^2} \right] -\pp{}y \left[ \frac {2x}{x^2+y^2} \right]   \\
		      = & \left\{ \frac{-4xy}{(x^2 + y^2)^2} \right\} - \left[ \frac{-4xy}{(x^2 + y^2)^2}\right] \\
		      = & 0
	      \end{align*}
	      Thus on each of $R_1$ and $R_2$ regions, $F$ satisfies the \textit{Criterion for Conservative Vector Fields in $\RR^2$}, therefore there exists potentials $\phi_1:R_1\to\RR$ and $\phi_2:R_2\to\RR$ which satisfy $\nabla \phi_{1/2}(x,y)=F(x,y)$ for all $(x,y)\in R_{1/2}$  respectively.

	      Now consider the potential $\phi:R\to\RR$ defined piecewise by:
	      $$\phi(x,y)=\begin{cases}
			      \phi_1(x,y) & (x,y)\in R_1 \\
			      \phi_2(x,y) & (x,y)\in R_2
		      \end{cases}$$
	      As $R_1$ and $R_2$ are disjoint and do not share a closed boundary, the constructed potential $\phi$ is still differentiable everywhere in $R$. See that $\phi$ satisfies $\nabla \phi(x,y) = F(x,y)$ for all $(x,y)\in R$. Hence, $\nabla \phi = F$.


	      %   Looking at the gradient field,  we see that a factor of $2x$ and $2y$ indicating the inside was differentiated using the chain rule, and a reciprocal indicating that the outer function could have been $\ln z$, with derivative $z^{-1}$.

	      %   Thus, as a guess we consider $\phi(x,y)=\ln(x^2+y^2)$ defined on $\RR^2\backslash\set{0}$, which is continuous and differentiable over its domain. Checking its gradient:
	      %   \[
	      %       \nabla\phi(x,y)
	      %       =\begin{bmatrix}
	      % 	      \frac{\partial \phi(x,y)}{\partial x} \\
	      % 	      \frac{\partial \phi(x,y)}{\partial y}
	      %       \end{bmatrix}
	      %       =\frac 2{x^2+y^2}
	      %       \begin{bmatrix}
	      % 	      x \\y
	      %       \end{bmatrix}
	      %   \]
	      %   As $\RR^2\backslash\set0\subset\set{(x,y)|y=\alpha x\;\alpha\in\RR}$, $\phi$ is defined here and, for any choice of $\alpha$, one has $\nabla\phi=F$. Hence, for any $R$, the potential $\phi(x,y)=\ln(x^2+y^2)$ exists and satisfies $\nabla \phi =F$.

	\item

	      Looking at the gradient field,  we see that a factor of $2x$ and $2y$ indicating the inside was differentiated using the chain rule, and a reciprocal indicating that the outer function could have been $\ln z$, with derivative $z^{-1}$.

	      Thus, as a guess we consider $\phi(x,y)=\ln(x^2+y^2)$ defined on $\RR^2\backslash\set{0}$, which is continuous and differentiable over its domain. Checking its gradient:
	      \[
		      \nabla\phi(x,y)
		      =\begin{bmatrix}
			      \frac{\partial \phi(x,y)}{\partial x} \\
			      \frac{\partial \phi(x,y)}{\partial y}
		      \end{bmatrix}
		      =\frac 2{x^2+y^2}
		      \begin{bmatrix}
			      x \\y
		      \end{bmatrix}
	      \]
	      Thus such a potential must exist as we have found it.

\end{enumerate}


\end{document}
