
\documentclass[a4paper, 11pt]{article}

% Fonts 
\usepackage{opensans}
\usepackage{amsfonts}
\usepackage{montserrat}
\usepackage{amsmath}

\usepackage[mathrm=sym]{unicode-math}

\setmainfont{opensans}
\setmathfont{Fira Math}

\newfontfamily{\montserrateb}{Montserrat SemiBold}
\newfontfamily{\montserratb}{Montserrat Bold}
\newfontfamily{\montserrat}{Montserrat Regular}
\newfontfamily{\montserratl}{Montserrat Light}
\DeclareMathAlphabet{\mathcal}{OMS}{cmbrs}{m}{n}

% \autoref
\usepackage{hyperref}

% Use for [H] option for figures to force in text placement
\usepackage{float}

% Captioning figures
\usepackage{caption}

% Subfigures
\usepackage{subcaption}

% For extending contents beyond margins
\usepackage{scrextend}

% For tables \midrule ect
\usepackage{booktabs}

% Colours
\usepackage[table,xcdraw]{xcolor}
\definecolor{accentcolor}{HTML}{6332a8}

% Change label in enumerate 
\usepackage{enumitem}

% Section settings
\usepackage{titlesec}
\titleformat{\section}
{\LARGE\montserrateb}
{\thesection.}{0.5em}{}

\titleformat{\subsection}
{\large\montserratb}
{\thesubsection.}{0.5em}{}

% Adjust document dimensions
\ExecuteOptions{a4paper}
\addtolength{\oddsidemargin}{-3cm}
\addtolength{\evensidemargin}{-3cm}
\addtolength{\topmargin}{-3cm}
\addtolength{\textwidth}{6cm}
\addtolength{\textheight}{4.5cm}
\addtolength{\textheight}{1.5cm}
\addtolength{\headsep}{-0.5cm}
% \addtolength{\footskip}{-1cm}
\parindent0pt
\parskip=4pt



\usepackage{matlab-prettifier}
\usepackage{graphicx}
\usepackage{mdframed}

% Creates coloured title box
\newcommand{\thetop}[5]{
	\begin{addmargin}[\oddsidemargin]{\oddsidemargin}
		\colorbox{#5}{\color{white}
			\hbox to \paperwidth{
				\vbox {
					\begin{center}
						{\large\montserratl #1}\\
						\vspace{4pt}
						{\huge\montserratb #2}\\
						{\montserratb #3}\\
						\vspace{-0.5em}
						\rule{20em}{1pt}

						{\large\montserratl
							#4
						}
					\end{center}
				}
			}
		}
	\end{addmargin}
}

\newcommand{\NN}{\mathbb{N}}
\newcommand{\ZZ}{\mathbb{Z}}
\newcommand{\RR}{\mathbb{R}}
\newcommand{\CC}{\mathbb{C}}
\newcommand{\dydt}{\frac{dy}{dt}}
\newcommand{\dxdt}{\frac{dx}{dt}}
\def\set#1{\left\{ #1 \right\}}
\def\eval#1#2{\left\ #1\right|_{#2}}

\def\pp#1#2{\frac{\partial #1}{\partial #2}}
\def\dd#1#2{\frac{\,d#1}{\,d#2}}

\begin{document}
\thetop{Robert Christie}{MATHS 340}{S1 2023}{Assignment 2\\Due: ??-??-2023}{accentcolor}

\section*{Q1}

\begin{enumerate}[label=(\alph*)]
	\item .
	\item .
	\item
	      \begin{align*}
		       & 0 \leq x\leq 1                                              \\
		       & 0 \leq y\leq 1                                              \\
		       & x^2 \leq z\leq x\iff \sqrt{x^2} \leq \sqrt z \land z \leq x \\
		      \intertext{Thus we can write the bounds as:}
		       & 0 \leq z\leq 1                                              \\
		       & 0 \leq y\leq 1                                              \\
		       & z \leq x\leq \sqrt z                                        \\
	      \end{align*}
	      As our triple integral is \textcolor{red}{``nice''}, we can rewrite it with these new bounds:
	      \[
		      \int_0^1\int_0^1\int_z^{\sqrt z} f(x,y,z) \,dx\,dy\,dz
	      \]

\end{enumerate}
\section*{Q2}

\begin{enumerate}[label=(\alph*)]
	\item First we find integrals for the surface area of each component body:
	      \begin{itemize}
		      \item The box has 6 sides, each with area:
		            \[
			            \int_0^4\int_0^4 1\,du\,dv
		            \]

		      \item For the tube, we parametrize the surface as:
		            $$r(u,v)=\begin{bmatrix}
				            e^u \cos v \\
				            e^u \sin v \\
				            u
			            \end{bmatrix} $$
		            Where $u\in[0,1]$ and $v\in[0,2\pi]$. Giving a surface integral:
		            \begin{align*}
			              & \int_0^{2\pi}\int_0^1 \left\|\pp {r(u,v)}{u} \times \pp{r(u,v)}{v} \right\| \,du\,dv    \\
			            = & \int_0^{2\pi}\int_0^1 \left\|
			            \begin{bmatrix}
				            e^u\cos v \\ e^u\sin v \\ 1
			            \end{bmatrix}\times
			            \begin{bmatrix}
				            -e^u\sin v \\
				            e^u\cos v  \\
				            0
			            \end{bmatrix}
			            \right\| \,du\,dv                                                                           \\
			            = & \int_0^{2\pi}\int_0^1 \left\| \begin{bmatrix}
				                                              -e^u\cos v \\ -e^u\sin v\\  e^{2u}\cos^2v +  e^{2u}\sin^2v
			                                              \end{bmatrix} \right\| \,du \,dv \\
			            = & \int_0^{2\pi}\int_0^1 \sqrt{e^{2u}+e^{4u}} \,du \, dv                                   \\
			            = & \int_0^{2\pi}\int_0^1 e^{u}\sqrt{1+e^{2u}} \,du \, dv                                   \\
		            \end{align*}

		            %   \item Using cylindrical coordinates, where the area element is
		            %         $\text{dA}=r\text{d$r$}\text{d$z$}\text{d$\theta$}$ The tube wall has a surface area of:
		            %         \[
		            %             \int_0^{2\pi} \int_0^1 e^z \text{d$z$}\text{d$\theta$}
		            %         \]
		            %         \textcolor{red}{Do we need to account for $dr$ in the area
		            %             element, making the integrand $e^{2z}$}
		      \item The top of the box must have a hole for the tube,
		            therefore we must remove an area correspond to the bottom of
		            the tube, a disk with radius $e^{0}=1$:
		            \[
			            - \int_0^1\int_0^{2\pi} r\,d\theta\,dr
		            \]
		      \item The bottom of the half sphere connects the top of the tube
		            to the sphere, thus it is a ring with inner radius $e^1$ and
		            outer radius $4$. Again using polar coordinates:
		            \[
			            \int_e^4\int_0^{2\pi} r\,d\theta\,dr
		            \]
		      \item We parametrize the surface of the sphere in spherical
		            coordinates, $[\rho,\theta,\phi]^T$ for $\rho=4$, $\theta\in[0,2\pi]$,
		            $\phi\in[0,\frac{\pi}2]$, the area element of the sphere is given by
		            $\rho^2\sin\phi\,d\theta\,d\phi$. Hence we have an integral
		            for the surface area of:
		            \[
			            \int_0^{\frac{\pi}2} \int_0^{2\pi}4^2\sin\phi \,d \theta \,d \phi
		            \]
	      \end{itemize}
	      Thus the surface area of $R$ is given by:
	      \[
		      A =
		      6\int_0^4\int_0^{4} 1 \,du\,dv
		      +\int_0^{2\pi}\int_0^1 e^{u}\sqrt{1+e^{2u}} \,du \,dv
		      -\int_0^1\int_0^{2\pi} r \,d\theta\,dr
		      +\int_e^4\int_0^{2\pi} r\,d\theta\,dr
		      +\int_0^{\frac{\pi}2} \int_0^{2\pi}4^2\sin\phi \,d\theta\, d\phi
	      \]
	\item We find three separate integrals for the mass of each region of
	      $R$:
	      \begin{itemize}
		      \item For the box we simply integrate $\rho$ over its bound in Cartesian
		            coordinates:
		            \[
			            \int_{-4}^0\int_{-2}^2\int_{-2}^2\rho(x,y,z)\,d x \,d y \,d z
		            \]
		      \item For the tube, we integrate in cylindrical coordinates where
		            the volume element is given by $\,dV = r\,d r\,d
			            \theta \,d z$, giving an integral of:
		            \[
			            \int_0^1\int_0^{2\pi}\int_0^{e^z} \rho(r\cos\theta,r\sin\theta,z) r\,d r\,d
			            \theta \,d z
		            \]
		      \item Lastly for the sphere, we integrate in Cartesian
		            coordinates:
		            \[
			            \int_{-4}^4\int_{-\sqrt{4^2-x^2}}^{\sqrt{4^2-x^2}}\int_1^{1+\sqrt{4^2-x^2-y^2}}\rho(x,y,z) \,d z\,d y\,d x
		            \]
	      \end{itemize}
	      Thus we have a total mass of:
	      \begin{align*}
		      m & =\int_{-4}^0\int_{-2}^2\int_{-2}^2\rho(x,y,z)\,d x \,d y \text
		      d z                                                                                                             \\
		        & +\int_0^1\int_0^{2\pi}\int_0^{e^z} \rho(r\cos\theta,r\sin\theta,z) r\,d r\,d
		      \theta \,d z                                                                                                    \\
		        & +\int_{-4}^4\int_{-\sqrt{4^2-x^2}}^{\sqrt{4^2-x^2}}\int_1^{1+\sqrt{4^2-x^2-y^2}}\rho(x,y,z) \,d z\,d y\,d x
	      \end{align*}
\end{enumerate}


\section*{Q3}
\begin{enumerate}[label=(\alph*)]

	\item Looking at the gradient field,  we see that a factor of $2x$ and $2y$ indicating the inside was differentiated using the chain rule, and a reciprocal indicating that the outer function could have been $\ln z$, with derivative $z^{-1}$. Thus, as a guess we consider $\phi(x,y)=\log(x^2+y^2)$ defined on $\RR^2\backslash\set{0}$, which is continuous and differentiable over its domain. Checking its gradient:
	      \[
		      \nabla\phi(x,y)
		      =\begin{bmatrix}
			      \frac{\partial \phi(x,y)}{\partial x} \\
			      \frac{\partial \phi(x,y)}{\partial y}
		      \end{bmatrix}
		      =\frac 2{x^2+y^2}
		      \begin{bmatrix}
			      x \\y
		      \end{bmatrix}
	      \]
	      As $\RR^2\backslash\set0\subset\set{(x,y)|y=\alpha x\;\alpha\in\RR}$, $\phi$ is defined here and one has $\nabla\phi=F$. Hence, $\phi$ is a potential which satisfies the question.

	\item In (a), we have already found a potential that satisfies the question, thus it exists.

\end{enumerate}


\end{document}
