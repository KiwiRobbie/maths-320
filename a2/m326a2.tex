
\documentclass[a4paper, 11pt]{article}

% Fonts 
\usepackage{opensans}
\usepackage{amsfonts}
\usepackage{montserrat}
\usepackage{amsmath}

\setmainfont{opensans}
\usepackage[mathrm=sym]{unicode-math}
% \setmathfont[Path=font/,mathrm=sym]{FiraMath-Regular}
\setmathfont[Path=font/,mathrm=sym]{LatoMath}

\newfontfamily{\montserrateb}{Montserrat SemiBold}
\newfontfamily{\montserratb}{Montserrat Bold}
\newfontfamily{\montserrat}{Montserrat Regular}
\newfontfamily{\montserratl}{Montserrat Light}
% \DeclareMathAlphabet{\mathcal}{OMS}{cmbrs}{m}{n}

% \usepackage[mathrm=sym]{unicode-math}
% \setmainfont{opensans}
% \setmathfont{Fira Math}

% \newfontfamily{\montserrateb}{Montserrat SemiBold}
% \newfontfamily{\montserratb}{Montserrat Bold}
% \newfontfamily{\montserrat}{Montserrat Regular}
% \newfontfamily{\montserratl}{Montserrat Light}
% \DeclareMathAlphabet{\mathcal}{OMS}{cmbrs}{m}{n}
% \setmathfont{Latin Modern Math}[range={\vdots}]

% \autoref
\usepackage{hyperref}

% Use for [H] option for figures to force in text placement
\usepackage{float}

% Captioning figures
\usepackage{caption}

% Subfigures
\usepackage{subcaption}

% For extending contents beyond margins
\usepackage{scrextend}

% For tables \midrule ect
\usepackage{booktabs}

% Colours
\usepackage[table,xcdraw]{xcolor}
\definecolor{accentcolor}{HTML}{a13640}

% Change label in enumerate 
\usepackage{enumitem}

% Section settings
\usepackage{titlesec}
\titleformat{\section}
{\LARGE\montserrateb}
{\thesection.}{0.5em}{}

\titleformat{\subsection}
{\large\montserratb}
{\thesubsection.}{0.5em}{}

% Adjust document dimensions
\ExecuteOptions{a4paper}
\addtolength{\oddsidemargin}{-3cm}
\addtolength{\evensidemargin}{-3cm}
\addtolength{\topmargin}{-3cm}
\addtolength{\textwidth}{6cm}
\addtolength{\textheight}{4.5cm}
\addtolength{\textheight}{1.5cm}
\addtolength{\headsep}{-0.5cm}
% \addtolength{\footskip}{-1cm}
\parindent0pt
\parskip=4pt



\usepackage{matlab-prettifier}
\usepackage{graphicx}
\usepackage{mdframed}

% Creates coloured title box
\newcommand{\thetop}[5]{
	\begin{addmargin}[\oddsidemargin]{\oddsidemargin}
		\colorbox{#5}{\color{white}
			\hbox to \paperwidth{
				\vbox {
					\begin{center}
						{\large\montserratl #1}\\
						\vspace{4pt}
						{\huge\montserratb #2}\\
						{\montserratb #3}\\
						\vspace{-0.5em}
						\rule{20em}{1pt}

						{\large\montserratl
							#4
						}
					\end{center}
				}
			}
		}
	\end{addmargin}
}

\newcommand{\NN}{\mathbb{N}}
\newcommand{\ZZ}{\mathbb{Z}}
\newcommand{\RR}{\mathbb{R}}
\newcommand{\CC}{\mathbb{C}}
\newcommand{\dydt}{\frac{dy}{dt}}
\newcommand{\dxdt}{\frac{dx}{dt}}
\def\set#1{\left\{ #1 \right\}}
\def\eval#1#2{\left\ #1\right|_{#2}}

\def\pp#1#2{\frac{\partial #1}{\partial #2}}
\def\dd#1#2{\frac{\,d#1}{\,d#2}}
\def\abs#1{\left|#1\right|}
\def\conj#1{\overline{#1}}

\usepackage{multicol}
\usepackage{tikz}
\usepackage{pgfplots}
\usetikzlibrary {graphs,graphdrawing} \usegdlibrary {force} 
\usetikzlibrary{graphs.standard}
\usetikzlibrary{positioning, 
                quotes}
\usegdlibrary {circular}

\usepackage{pst-platon}

\usepackage{comment}

\begin{comment}
	Q1: sol
	Q2: 
	Q3a: sol 
	Q4: sol
	Q5: sol
	Q6: 
	Q7:
\end{comment}

\begin{document}
\thetop{Robert Christie}{MATHS 326}{S1 2024}{Assignment 2\\Due: 20-04-2024}{accentcolor}

\begin{multicols*}{2}
	\section*{Q1}
	Consider the following weighted graph:
	\begin{center}
		\begin{tikzpicture}[
				node distance = 15mm and 15mm,
				% V/.style = {circle, draw, fill=gray!30},
				V/.style = {circle, draw},
				every edge quotes/.style = {auto, font=\footnotesize, sloped}
			]
			\begin{scope}[nodes=V]
				\node (A)   {A};
				\node (B) [above right=of A]    {B};
				\node (C) [below right=of B]    {C};
			\end{scope}

			\draw
			(A)  edge["$2$"] (B)
			(B)  edge["$-2$"] (C);

			\draw
			(A)  edge["$1$"] (C);
		\end{tikzpicture}
	\end{center}
	Applying Dijkstra's Algorithm starting from $v=A$:
	\begin{itemize}
		\item
		      Initial stage: We set:
		      \[
			      R=\emptyset,\quad
			      \delta(v)=\begin{cases}
				      0      & v=A \\
				      \infty & v=B \\
				      \infty & v=C
			      \end{cases}
		      \]
		      Also set $A$ to be the current vertex.

		\item Iteration 1: Add $A$ to $R$, and update $\delta$ giving:
		      \[
			      R=\set{A},\quad
			      \delta(v)=\begin{cases}
				      0 & v=A \\
				      2 & v=B \\
				      1 & v=C
			      \end{cases}
		      \]
		      Then we set $C$ as the current vertex as it has the smallest $\delta$ of the unvisited vertices.

		\item Iteration 2: We add $C$ to $R$, and update $\delta$ giving:
		      \[
			      R=\set{A,C},\quad
			      \delta(v)=\begin{cases}
				      \hphantom{-}0 & v=A \\
				      -1            & v=B \\
				      \hphantom{-}1 & v=C
			      \end{cases}
		      \]
		      Now that $C$ has been visited, the algorithm will not alter $\delta(C)$ any further.
	\end{itemize}
	Thus, the final value of $\delta(C)=1$. However, the actual shortest path from $A$ to $C$ is $A$---$B$---$C$ with length $0$. So Dijkstra's algorithm doesn't always find shortest paths if positive and negative edge weights are allowed.

	\section*{Q2}

	\begin{itemize}
		\item
		      Assume that $M$ is not unique, so there exists some other minimum spanning tree $M'$.

		\item Since both are MST, $M$ and $M'$ have the same weight.

		\item Since each edge weight is unique, $M$ and $M'$ must differ at at-least two points.

		\item
	\end{itemize}

	\section*{Q3}
	\begin{enumerate}[label=(\alph*)]
		\item Counter example: Consider the following graph:
		      \begin{center}
			      \begin{tikzpicture}[
					      node distance = 15mm and 15mm,
					      % V/.style = {circle, draw, fill=gray!30},
					      V/.style = {circle, draw},
					      every edge quotes/.style = {auto, font=\footnotesize, sloped}
				      ]
				      \begin{scope}[nodes=V]
					      \node (A)   {A};
					      \node (B) [above right=of A]    {B};
					      \node (C) [right=of B]          {C};
					      \node (D) [below right=of C]    {D};
				      \end{scope}

				      \draw[red!70!black]
				      (A)  edge["$\frac12$"] (B)
				      (B)  edge["$\frac12$"] (C)
				      (C)  edge["$\frac12$"] (D);

				      \draw[blue!70!black]
				      (A)  edge["$1$"] (D);

			      \end{tikzpicture}
		      \end{center}
		      In the original weighting, the shortest path from $A$ to $D$ is given by the blue path with weight $1$. Replacing each edge weight with the square would make the red path shortest with weight $\frac 34$, the blue path still has length $1$.

		\item
		      Proof: Let $G=(V,E,w)$ be a weight graph with original weighting function $w:E\to \RR^+$. Since a minimum spanning tree $M$ of $G$ is a forest with $p=\abs V$ vertices and $q=\abs E$ edges and $c=1$ connect components, By \textit{lemma 0.1}, $p-q=c$ so $q=p-c=p-1$.


		      1=>2:
		      Let $M$ be a MST in $w$. If $M$ is not an MST in


		      2=>1:

		      %   Assume that $M$ is not a MST in the new weighting function. Then some other $M'$ is a MST in the new weighting, but not in the old weighting.
		      %   \[
		      %       \sum_{e\in M}w(e) < \sum_{e\in M'}w(e)
		      %   \]





	\end{enumerate}

	\section*{Q4}
	1=>2:

	\begin{mdframed}
		Assume a tree $T=(V,E)$ has a perfect matching. For some $v\in V$, there is some $u\in V$ matched to $v$. This splits $T$ into $n\geq 1$ connected components.

		One of these components $C_1$ must contain $u$, there must be no other edge from $C_1$ to $V$ in $T$ otherwise $T$ would contain a cycle. As $T$ has a perfect matching, \textcolor{red}{$C_1-u$ must have a perfect matching} so $C_1$ has an odd number of vertices.

		For each remaining connected $C_i$ did contain a vertex matched to $v$. Thus, $C_i$ has a perfect matching and so $C_i$ has an even number of vertices.
	\end{mdframed}

	2=>1:
	\begin{mdframed}
		Claim: $T-v$ must contain exactly one odd connected component as $o(T-v)=1$ (by hypothesis) call this component $\Theta(v)$.

		Claim: Since $T$ was connected, each connected component must contain a vertex $u$ adjacent to $v$ in $T$.

		Claim: There is exactly one such vertex $u$ in each component. (Otherwise, if there were two vertices then there must be a cycle in $T$ which is a contradiction)

		Let $M$ be a set. For every $v\in V$, there is some $u$ from $\Theta(v)$ that it is adjacent to $v$ in $T$, add $e={v,u}$ to $M$.

		Claim: The edges in $M$ are mutually disjoint.

		Proof by contradiction:
		\begin{itemize}
			\item 	Assume that two edges in $M$ are not disjoint, so some $u,v\in V$ were both paired to some $w$.

			\item Let $G_u$ be the subgraph of $T$ induced by all the vertices of the even connected components of $T-u$ and $u$. Notice that $G_u$ is connected as $T$ was connected, and the only edges removed from $T-u$ contain $u$ so every connected component was connected to $u$ in $T$. Define $G_v$ in the same way.

			\item For any subgraph $G\in\set{G_u,G_v}$, only $u/v$ could be connected to a vertex outside of $G$,

			\item These subgraphs are disjoint since $\Theta(u)$ contains $w$ which is still connected to $v$ in $G_v$, thus $G_v$ is a subgraph of $\Theta(u)$ which is not contained in $G_u$.

			\item Notice that both $G_u$ and $G_v$ contain an odd number of vertices as they contain the even connected components in $T-u$ or $T-v$ as well as $u$ or $v$ respectfully.

			\item Now consider $T-w$, both $G_u$ and $G_v$ are connected components in $T-w$ as they are not contained in any larger connected as the only edge in $T$ leaving either subgraph contained $w$.


			      with odd numbers of vertices so $o(T-w)\geq 2$ which is a contradiction.
		\end{itemize}

		Therefore, $M$ is a matching by definition, however $M$ is also a perfect matching as it contains every $v\in V$.
	\end{mdframed}





	\section*{Q5}

	\begin{enumerate}[label=(\alph*)]
		\item
		      Since $G=(V,E)$ is regular, $G$ is $d$-regular for $d=\Delta(G)$. Since $G$ is class 1, there is a $d$-edge colouring $\lambda: E\to C$ with $|C|=d$. For any edge $e_0$ of $G$, consider the matching give by:
		      $$M= \set{e\in E : \lambda(e)=\lambda(e_0)}$$
		      By definition of edge-colouring, the edges of $M$ are mutually disjoint and therefore a valid matching. Since each degree has degree $d$, there are $d$ edges with distinct colours including an edge with colour $\lambda(e_0)$. Thus, every vertex must be contained in $M$ making $M$ a perfect matching so any $e_0$ is matchable.

		\item The Petersen graph is $3$-regular and was shown to have chromatic index $4$ in \textit{Assignment 1}, thus it is regular and class 2. Consider the following matchings on the Petersen graph where the red edges are the matched edges:
		      \begin{center}
			      \begin{tabular}{cc}
				      \begin{tikzpicture}[every node/.style={draw,circle}]
					      \begin{scope}[shift={(0,-0.75cm)}]
						      \graph[empty nodes, simple necklace layout, clockwise, radius=0.75cm] {
						      A, B, C, D, E;
						      A --[black] C --[red] E --[black] B --[red] D --[black] A;
						      };
					      \end{scope}
					      \graph[empty nodes, simple necklace layout, clockwise, radius=1.5cm] {
					      1--[black]2--[red]3--[black]4--[red]5--[black]1;
					      };
					      \draw[red] (1) -- (A);
					      \draw[black]  (2) -- (B);
					      \draw[black]  (3) -- (C);
					      \draw[black]  (4) -- (D);
					      \draw[black]  (5) -- (E);
				      \end{tikzpicture}
				       &
				      \begin{tikzpicture}[every node/.style={draw,circle}]
					      \begin{scope}[rotate=72]
						      \begin{scope}[shift={(0,-0.75cm)}]
							      \graph[empty nodes, simple necklace layout, clockwise, radius=0.75cm] {
							      A, B, C, D, E;
							      A --[black] C --[red] E --[black] B --[red] D --[black] A;
							      };
						      \end{scope}
						      \graph[empty nodes, simple necklace layout, clockwise, radius=1.5cm] {
						      1--[black]2--[red]3--[black]4--[red]5--[black]1;
						      };
						      \draw[red] (1) -- (A);
						      \draw[black]  (2) -- (B);
						      \draw[black]  (3) -- (C);
						      \draw[black]  (4) -- (D);
						      \draw[black]  (5) -- (E);
					      \end{scope}
				      \end{tikzpicture}
				      \\
				      \begin{tikzpicture}[every node/.style={draw,circle}]
					      \begin{scope}[rotate=144]
						      \begin{scope}[shift={(0,-0.75cm)}]
							      \graph[empty nodes, simple necklace layout, clockwise, radius=0.75cm] {
							      A, B, C, D, E;
							      A --[black] C --[red] E --[black] B --[red] D --[black] A;
							      };
						      \end{scope}
						      \graph[empty nodes, simple necklace layout, clockwise, radius=1.5cm] {
						      1--[black]2--[red]3--[black]4--[red]5--[black]1;
						      };
						      \draw[red] (1) -- (A);
						      \draw[black]  (2) -- (B);
						      \draw[black]  (3) -- (C);
						      \draw[black]  (4) -- (D);
						      \draw[black]  (5) -- (E);
					      \end{scope}
				      \end{tikzpicture}
				       &
				      \begin{tikzpicture}[every node/.style={draw,circle}]
					      \begin{scope}[rotate=216]
						      \begin{scope}[shift={(0,-0.75cm)}]
							      \graph[empty nodes, simple necklace layout, clockwise, radius=0.75cm] {
							      A, B, C, D, E;
							      A --[black] C --[red] E --[black] B --[red] D --[black] A;
							      };
						      \end{scope}
						      \graph[empty nodes, simple necklace layout, clockwise, radius=1.5cm] {
						      1--[black]2--[red]3--[black]4--[red]5--[black]1;
						      };
						      \draw[red] (1) -- (A);
						      \draw[black]  (2) -- (B);
						      \draw[black]  (3) -- (C);
						      \draw[black]  (4) -- (D);
						      \draw[black]  (5) -- (E);
					      \end{scope}
				      \end{tikzpicture}
				      \\
				      \multicolumn{2}{c}{
					      \begin{tikzpicture}[every node/.style={draw,circle}]
						      \begin{scope}[rotate=288]
							      \begin{scope}[shift={(0,-0.75cm)}]
								      \graph[empty nodes, simple necklace layout, clockwise, radius=0.75cm] {
								      A, B, C, D, E;
								      A --[black] C --[red] E --[black] B --[red] D --[black] A;
								      };
							      \end{scope}
							      \graph[empty nodes, simple necklace layout, clockwise, radius=1.5cm] {
							      1--[black]2--[red]3--[black]4--[red]5--[black]1;
							      };
							      \draw[red] (1) -- (A);
							      \draw[black]  (2) -- (B);
							      \draw[black]  (3) -- (C);
							      \draw[black]  (4) -- (D);
							      \draw[black]  (5) -- (E);
						      \end{scope}
					      \end{tikzpicture}
				      }
			      \end{tabular}
		      \end{center}
		      Since every edge is contained in at least one of the matchings above, no edge is unmatchable.
	\end{enumerate}

	\section*{Q6}
	Since $G=(V,E)$ is bipartite, then $V=V_1\uplus V_2$, WLOG assume $\abs{V_1}\geq\abs{V_2}$.



	\section*{Q7}

	\vfill
	\pagebreak
\end{multicols*}
\end{document}
