
\documentclass[a4paper, 11pt]{article}

% Fonts 
\usepackage{opensans}
\usepackage{amsfonts}
\usepackage{montserrat}
\usepackage{amsmath}

\setmainfont{opensans}
\usepackage[mathrm=sym]{unicode-math}
% \setmathfont[Path=font/,mathrm=sym]{FiraMath-Regular}
\setmathfont[Path=font/,mathrm=sym]{LatoMath}

\newfontfamily{\montserrateb}{Montserrat SemiBold}
\newfontfamily{\montserratb}{Montserrat Bold}
\newfontfamily{\montserrat}{Montserrat Regular}
\newfontfamily{\montserratl}{Montserrat Light}
% \DeclareMathAlphabet{\mathcal}{OMS}{cmbrs}{m}{n}

% \usepackage[mathrm=sym]{unicode-math}
% \setmainfont{opensans}
% \setmathfont{Fira Math}

% \newfontfamily{\montserrateb}{Montserrat SemiBold}
% \newfontfamily{\montserratb}{Montserrat Bold}
% \newfontfamily{\montserrat}{Montserrat Regular}
% \newfontfamily{\montserratl}{Montserrat Light}
% \DeclareMathAlphabet{\mathcal}{OMS}{cmbrs}{m}{n}
% \setmathfont{Latin Modern Math}[range={\vdots}]

% \autoref
\usepackage{hyperref}

% Use for [H] option for figures to force in text placement
\usepackage{float}

% Captioning figures
\usepackage{caption}

% Subfigures
\usepackage{subcaption}

% For extending contents beyond margins
\usepackage{scrextend}

% For tables \midrule ect
\usepackage{booktabs}

% Colours
\usepackage[table,xcdraw]{xcolor}
\definecolor{accentcolor}{HTML}{a13640}

% Change label in enumerate 
\usepackage{enumitem}

% Section settings
\usepackage{titlesec}
\titleformat{\section}
{\LARGE\montserrateb}
{\thesection.}{0.5em}{}

\titleformat{\subsection}
{\large\montserratb}
{\thesubsection.}{0.5em}{}

% Adjust document dimensions
\ExecuteOptions{a4paper}
\addtolength{\oddsidemargin}{-3cm}
\addtolength{\evensidemargin}{-3cm}
\addtolength{\topmargin}{-3cm}
\addtolength{\textwidth}{6cm}
\addtolength{\textheight}{4.5cm}
\addtolength{\textheight}{1.5cm}
\addtolength{\headsep}{-0.5cm}
% \addtolength{\footskip}{-1cm}
\parindent0pt
\parskip=4pt



\usepackage{matlab-prettifier}
\usepackage{graphicx}
\usepackage{mdframed}

% Creates coloured title box
\newcommand{\thetop}[5]{
	\begin{addmargin}[\oddsidemargin]{\oddsidemargin}
		\colorbox{#5}{\color{white}
			\hbox to \paperwidth{
				\vbox {
					\begin{center}
						{\large\montserratl #1}\\
						\vspace{4pt}
						{\huge\montserratb #2}\\
						{\montserratb #3}\\
						\vspace{-0.5em}
						\rule{20em}{1pt}

						{\large\montserratl
							#4
						}
					\end{center}
				}
			}
		}
	\end{addmargin}
}

\newcommand{\NN}{\mathbb{N}}
\newcommand{\ZZ}{\mathbb{Z}}
\newcommand{\RR}{\mathbb{R}}
\newcommand{\CC}{\mathbb{C}}
\newcommand{\dydt}{\frac{dy}{dt}}
\newcommand{\dxdt}{\frac{dx}{dt}}
\def\set#1{\left\{ #1 \right\}}
\def\eval#1#2{\left\ #1\right|_{#2}}

\def\pp#1#2{\frac{\partial #1}{\partial #2}}
\def\dd#1#2{\frac{\,d#1}{\,d#2}}
\def\abs#1{\left|#1\right|}
\def\conj#1{\overline{#1}}

\usepackage{multicol}
\usepackage{tikz}
\usepackage{pgfplots}
\usetikzlibrary {graphs,graphdrawing} \usegdlibrary {force} 
\usetikzlibrary{graphs.standard}
\usetikzlibrary{positioning, 
                quotes}
\usegdlibrary {circular}

\usepackage{pst-platon}

\usepackage{comment}

\begin{comment}
	Q1:  sol
	Q2:  sol 
	Q3a: sol
	Q3b:  
	Q4:  sol
	Q5a: sol
	Q5b: sol
	Q6:  sol
	Q7:  sol
\end{comment}

\begin{document}
\thetop{Robert Christie}{MATHS 326}{S1 2024}{Assignment 3\\Due: 10-05-2024}{accentcolor}

% \begin{multicols*}{2}
\section*{Q1}
\begin{enumerate}[label=(\alph*)]
	\item For any $x\in V\setminus\set{a,z}$, then we have:

	      \[
		      p_x = \sum_{y\in N(x)}\frac{c(\set{x,y})p_y}{\pi(x)}
	      \]

	      As the probability of reaching $a$ before $z$, is the sum of probabilities from each neighbour, weighted by the chance of reaching that neighbour.

	      This also means that as a function $x\mapsto p_x$, on $V\setminus\set{a,z}$, $p_x$ is harmonic. We also know that $p_a=1$ and $p_z=0$. Meaning that this is an instance of the discrete Dirchlet problem, hence $x\mapsto p_x$ must be the unique solution.

	      Now consider any $f:V\to\RR$ harmonic on $V\setminus\set{a,z}$. Consider a function $g:V\to\RR$ given by $g(x)=f(a)p_x+f(z)$. We know $g(x)$ is harmonic as it is a linear combination of harmonic functions (on $V\setminus\set{a,z}$).

	      As $p_a=1$ and $p_z=0$, $\alpha=g(a)=f(a)$ and $\beta=g(z)=f(z)$, we have that $f,g$ are solutions to the same discrete Dirchlet problem. Thus, $f=g$ as the solution is unique. Hence, we can write:
	      \[f(x)=f(a)p_x+f(b)=\alpha p_x +\beta\]

	\item We can rewrite:
	      \[f(x)=f(a)p_x+f(b)=\alpha p(x) + \beta q(x)\]
	      Where $p(x)=p_x$ and $q(x)=1$ are both harmonic functions on $V\setminus{a,z}$. By (a), $\set{p,q}$ spans the vector space of function harmonic on $V\setminus\set{a,z}$. We also see that:
	      \[
		      \alpha p +\beta q =0 \implies \begin{matrix}
			      \alpha p(a) + \beta q(a)=\alpha+\beta=0 \\
			      \alpha p(z) + \beta q(z)=\beta=0        \\
		      \end{matrix}
		      \implies \alpha=\beta=0
	      \]
	      Thus, the set $\set{p,q}$ is linearly independent and spanning, thus it is a basis with cardinality $2$ hence, the dimension of the vector space is also $2$.
\end{enumerate}

\section*{Q2}
Restating the Star-Triangle law for resistance:
\begin{mdframed}
	Consider a start with centre $x$ with edges to $y_0,y_1,y_2$. Then:
	\begin{align*}
		\gamma
		 & = \frac{c(x,y_0)c(x,y_1)c(x,y_2)}{c(x,y_0)+c(x,y_1)+c(x,y_2)}                       \\
		 & = \frac{1}{r(x,y_0)r(x,y_1)r(x,y_2)\left[ 1/r(x,y_0)+1/r(x,y_1)+1/r(x,y_2) \right]} \\
		 & = \frac1{r(x,y_0)r(x,y_1) + r(x,y_1)r(x,y_2) + r(x,y_2)r(x,y_0)}
	\end{align*}
	So $\set{y_i,y_{i+1}}$ where indices are taken mod $3$, has resistance:
	\begin{align*}
		r(y_i,y_{i+1}) & = \frac1{\gamma c(x,y_{i+2})}                                                                                                        \\
		               & = \frac{1}{r(x,y_{i+2})r(x,y_0)r(x,y_1)r(x,y_2)\left[ 1/r(x,y_0)+1/r(x,y_1)+1/r(x,y_2) \right]}                                      \\
		               & = r(x,y_{i+2}) \left[ \frac1{r(x,y_{i+0})r(x,y_{i+1})} + \frac1{r(x,y_{i+1})r(x,y_{i+2})} + \frac1{r(x,y_{i+2})r(x,y_{i+0})} \right] \\
		               & =  \frac{r(x,y_{i+2})}{r(x,y_{i+0})r(x,y_{i+1})} + \frac1{r(x,y_{i+1})} + \frac1{r(x,y_{i+0})}
	\end{align*}


\end{mdframed}

\begin{tabular}{ccccccc}
	 &
	\begin{tikzpicture}
		\node[circle,fill,scale=0.5] (a) at (0,1) {};
		\node[circle,fill,scale=0.5] (b) at (1,1) {};
		\node[circle,fill,scale=0.5] (c) at (2,1) {};
		\node[circle,fill,scale=0.5] (d) at (3,1) {};
		\node[circle,fill,scale=0.5] (e) at (0,0) {};
		\node[circle,fill,scale=0.5] (f) at (1,0) {};
		\node[circle,fill,scale=0.5] (g) at (2,0) {};
		\node[circle,fill,scale=0.5] (z) at (3,0) {};

		\draw (a) -- (b) -- (c) -- (d);
		\draw (e) -- (f) -- (g) -- (z);
		\draw (a) -- (e) ;
		\draw (b) -- (f);
		\draw (c) -- (g);
		\draw (d) -- (z);
	\end{tikzpicture}
	 &
	$
		\begin{array}{c}
			\longrightarrow         \\[-0.5em]
			\text{\tiny Series Law} \\[1em]
		\end{array}
	$
	 &
	\begin{tikzpicture}[
			every edge quotes/.style = {auto, font=\footnotesize, sloped}
		]
		\node[circle,fill,scale=0.5] (a) at (0,1) {};
		\node[circle,fill,scale=0.5] (b) at (1,1) {};
		\node[circle,fill,scale=0.5] (c) at (2,1) {};


		\node[circle,fill,scale=0.5] (f) at (1,0) {};
		\node[circle,fill,scale=0.5] (g) at (2,0) {};
		\node[circle,fill,scale=0.5] (z) at (3,0) {};

		\draw (a) -- (b);
		\draw (g) -- (z);
		\draw (b) -- (c) -- (g) -- (f) -- (b);

		\draw (a) edge["$2$"] (f) ;
		\draw (c) edge["$2$"] (z) ;
	\end{tikzpicture}
	 &
	$
		\begin{array}{c}
			\longrightarrow                \\[-0.5em]
			\text{\tiny Star Triangle Law} \\[1em]
		\end{array}
	$
	 &
	\begin{tikzpicture}[
			every edge quotes/.style = {auto, font=\footnotesize, sloped}
		]
		\node[circle,fill,scale=0.5] (a) at (0,1) {};
		% \node[circle,fill,scale=0.5] (b) at (1,1) {};
		\node[circle,fill,scale=0.5] (c) at (2,1) {};

		\node[circle,fill,scale=0.5] (f) at (1,0) {};
		\node[circle,fill,scale=0.5] (g) at (2,0) {};
		\node[circle,fill,scale=0.5] (z) at (3,0) {};


		\draw (a) -- (f) -- (g) -- (z);
		\draw (c) -- (g);

		\path (f) edge [bend left] (a);

		\draw (a) -- (c) -- (f);


		\draw (c) edge["$2$"] (z) ;


	\end{tikzpicture}    \\
	$
		\begin{array}{c}
			\longrightarrow                \\[-0.5em]
			\text{\tiny Star Triangle Law} \\[1em]
		\end{array}
	$
	 &
	\begin{tikzpicture}
		\node[circle,fill,scale=0.5] (a) at (0,1) {};
		\node[circle,fill,scale=0.5] (c) at (2,1) {};

		\node[circle,fill,scale=0.5] (f) at (1,0) {};
		\node[circle,fill,scale=0.5] (z) at (3,0) {};


		\draw (a) -- (c) -- (z);
		\draw (a) -- (f)  -- (z);

		\path (f) edge [bend left] (a);
		\path (c) edge [bend left] (z);

		\path (f) edge [bend left] (c);
		\path (c) edge [bend left] (f);
	\end{tikzpicture}
	 &
	$
		\begin{array}{c}
			\longrightarrow           \\[-0.5em]
			\text{\tiny Parallel Law} \\[1em]
		\end{array}
	$
	 & \begin{tikzpicture}
		   \node[circle,fill,scale=0.5] (a) at (0,1) {};
		   \node[circle,fill,scale=0.5] (c) at (2,1) {};

		   \node[circle,fill,scale=0.5] (f) at (1,0) {};
		   \node[circle,fill,scale=0.5] (z) at (3,0) {};


		   \draw (a) -- (c) -- (z) -- (f)--(a);
		   \draw (c) -- (f);
	   \end{tikzpicture}
	 &
	$
		\begin{array}{c}
			\longrightarrow                \\[-0.5em]
			\text{\tiny Star Triangle Law} \\[1em]
		\end{array}
	$
	 &
	\begin{tikzpicture}
		\node[circle,fill,scale=0.5] (a) at (0,1) {};
		\node[circle,fill,scale=0.5] (b) at (1,1) {};
		\node[circle,fill,scale=0.5] (c) at (2,1) {};

		\node[circle,fill,scale=0.5] (f) at (1,0) {};
		\node[circle,fill,scale=0.5] (z) at (3,0) {};


		\draw (a) -- (b) --(c) -- (z) -- (f);
		\draw (b) -- (f);
	\end{tikzpicture}
	\\
	$
		\begin{array}{c}
			\longrightarrow         \\[-0.5em]
			\text{\tiny Series Law} \\[1em]
		\end{array}
	$
	 &
	\begin{tikzpicture}
		\node[circle,fill,scale=0.5] (a) at (0,1) {};
		\node[circle,fill,scale=0.5] (b) at (1,1) {};
		\node[circle,fill,scale=0.5] (z) at (3,0) {};
		\draw (a) -- (b);
		\path (b) edge [bend left] (z);
		\path (z) edge [bend left] (b);


	\end{tikzpicture}
	 &
	$
		\begin{array}{c}
			\longrightarrow           \\[-0.5em]
			\text{\tiny Parallel Law} \\[1em]
		\end{array}
	$
	 &
	\begin{tikzpicture}
		\node[circle,fill,scale=0.5] (a) at (0,1) {};
		\node[circle,fill,scale=0.5] (b) at (1,1) {};
		\node[circle,fill,scale=0.5] (z) at (3,0) {};
		\draw (a) -- (b)  -- (z);
	\end{tikzpicture}
	 &
	$
		\begin{array}{c}
			\longrightarrow         \\[-0.5em]
			\text{\tiny Series Law} \\[1em]
		\end{array}
	$
	 &
	\begin{tikzpicture}
		\node[circle,fill,scale=0.5] (a) at (0,1) {};
		\node[circle,fill,scale=0.5] (z) at (3,0) {};
		\draw (a) -- (z);
	\end{tikzpicture}
\end{tabular}









\section*{Q3}

Assume that there is a shortest path from $a$ to $z$ where the voltage increases between some pair of vertices. Let $v_1,\dots,v_k,\dots,v_n$ with $v_1=a$ and $v_n=z$ be vertices of the path such that $v_k$ is the first vertex such that $v(v_k)>v(v_{k-1})$.




\section*{Q4}


\section*{Q5}
\begin{enumerate}[label=(\alph*)]
	\item
\end{enumerate}

\section*{Q6}



% \vfill
% \pagebreak
% \end{multicols*}
\end{document}
