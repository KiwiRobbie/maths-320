
\documentclass[a4paper, 11pt]{article}

% Fonts 
\usepackage{opensans}
\usepackage{amsfonts}
\usepackage{montserrat}
\usepackage{amsmath}

\setmainfont{opensans}
% \usepackage[mathrm=sym]{unicode-math}
\usepackage{lato-math}
% \setmathfont[Path=font/,mathrm=sym]{FiraMath-Regular}
% \setmathfont[Path=font/,mathrm=sym]{LatoMath}

\newfontfamily{\montserrateb}{Montserrat SemiBold}
\newfontfamily{\montserratb}{Montserrat Bold}
\newfontfamily{\montserrat}{Montserrat Regular}
\newfontfamily{\montserratl}{Montserrat Light}
% \DeclareMathAlphabet{\mathcal}{OMS}{cmbrs}{m}{n}

% \usepackage[mathrm=sym]{unicode-math}
% \setmainfont{opensans}
% \setmathfont{Fira Math}

% \newfontfamily{\montserrateb}{Montserrat SemiBold}
% \newfontfamily{\montserratb}{Montserrat Bold}
% \newfontfamily{\montserrat}{Montserrat Regular}
% \newfontfamily{\montserratl}{Montserrat Light}
% \DeclareMathAlphabet{\mathcal}{OMS}{cmbrs}{m}{n}
% \setmathfont{Latin Modern Math}[range={\vdots}]

% \autoref
\usepackage{hyperref}

% Use for [H] option for figures to force in text placement
\usepackage{float}

% Captioning figures
\usepackage{caption}

% Subfigures
\usepackage{subcaption}

% For extending contents beyond margins
\usepackage{scrextend}

% For tables \midrule ect
\usepackage{booktabs}

% Colours
\usepackage[table,xcdraw]{xcolor}
\definecolor{accentcolor}{HTML}{a13640}

% Change label in enumerate 
\usepackage{enumitem}

% Section settings
\usepackage{titlesec}
\titleformat{\section}
{\LARGE\montserrateb}
{\thesection.}{0.5em}{}

\titleformat{\subsection}
{\large\montserratb}
{\thesubsection.}{0.5em}{}

% Adjust document dimensions
\ExecuteOptions{a4paper}
\addtolength{\oddsidemargin}{-3cm}
\addtolength{\evensidemargin}{-3cm}
\addtolength{\topmargin}{-3cm}
\addtolength{\textwidth}{6cm}
\addtolength{\textheight}{4.5cm}
\addtolength{\textheight}{1.5cm}
\addtolength{\headsep}{-0.5cm}
% \addtolength{\footskip}{-1cm}
\parindent0pt
\parskip=4pt



\usepackage{matlab-prettifier}
\usepackage{graphicx}
\usepackage{mdframed}

% Creates coloured title box
\newcommand{\thetop}[5]{
	\begin{addmargin}[\oddsidemargin]{\oddsidemargin}
		\colorbox{#5}{\color{white}
			\hbox to \paperwidth{
				\vbox {
					\begin{center}
						{\large\montserratl #1}\\
						\vspace{4pt}
						{\huge\montserratb #2}\\
						{\montserratb #3}\\
						\vspace{-0.5em}
						\rule{20em}{1pt}

						{\large\montserratl
							#4
						}
					\end{center}
				}
			}
		}
	\end{addmargin}
}

\newcommand{\NN}{\mathbb{N}}
\newcommand{\ZZ}{\mathbb{Z}}
\newcommand{\RR}{\mathbb{R}}
\newcommand{\CC}{\mathbb{C}}
\newcommand{\dydt}{\frac{dy}{dt}}
\newcommand{\dxdt}{\frac{dx}{dt}}
\def\set#1{\left\{ #1 \right\}}
\def\eval#1#2{\left\ #1\right|_{#2}}

\def\pp#1#2{\frac{\partial #1}{\partial #2}}
\def\dd#1#2{\frac{\,d#1}{\,d#2}}
\def\abs#1{\left|#1\right|}
\def\conj#1{\overline{#1}}

\usepackage{multicol}
\usepackage{tikz}
\usepackage{pgfplots}
\usetikzlibrary {graphs,graphdrawing} \usegdlibrary {force} 
\usetikzlibrary{graphs.standard}
\usetikzlibrary{positioning, 
                quotes}
\usegdlibrary {circular}

\usepackage{pst-platon}

\usepackage{comment}

\begin{comment}
	Q1: Good
	Q2  needs time
	Q3: 
	Q4: Good 
	Q5
	Q6: Good 
\end{comment}

\begin{document}
\thetop{Robert Christie}{MATHS 326}{S1 2024}{Assignment 3\\Due: 10-05-2024}{accentcolor}

% \begin{multicols*}{2}
\section*{Q1}
\begin{enumerate}[label=(\alph*)]
	\item For any $x\in V\setminus\set{a,z}$, then we have:

	      \[
		      p_x = \sum_{y\in N(x)}\frac{c(\set{x,y})p_y}{\pi(x)}
	      \]

	      As the probability of reaching $a$ before $z$, is the sum of probabilities from each neighbour, weighted by the chance of reaching that neighbour.

	      This also means that as a function $x\mapsto p_x$, on $V\setminus\set{a,z}$, $p_x$ is harmonic. We also know that $p_a=1$ and $p_z=0$. Meaning that this is an instance of the discrete Dirchlet problem, hence $x\mapsto p_x$ must be the unique solution.

	      Now consider any $f:V\to\RR$ harmonic on $V\setminus\set{a,z}$. Consider a function $g:V\to\RR$ given by $g(x)=f(a)p_x+f(z)$. We know $g(x)$ is harmonic as it is a linear combination of harmonic functions (on $V\setminus\set{a,z}$).

	      As $p_a=1$ and $p_z=0$, $\alpha=g(a)=f(a)$ and $\beta=g(z)=f(z)$, we have that $f,g$ are solutions to the same discrete Dirchlet problem. Thus, $f=g$ as the solution is unique. Hence, we can write:
	      \[f(x)=f(a)p_x+f(b)=\alpha p_x +\beta\]

	\item We can rewrite:
	      \[f(x)=f(a)p_x+f(b)=\alpha p(x) + \beta q(x)\]
	      Where $p(x)=p_x$ and $q(x)=1$ are both harmonic functions on $V\setminus{a,z}$. By (a), $\set{p,q}$ spans the vector space of function harmonic on $V\setminus\set{a,z}$. We also see that:
	      \[
		      \alpha p +\beta q =0 \implies \begin{matrix}
			      \alpha p(a) + \beta q(a)=\alpha+\beta=0 \\
			      \alpha p(z) + \beta q(z)=\beta=0        \\
		      \end{matrix}
		      \implies \alpha=\beta=0
	      \]
	      Since the set $\set{p,q}$ is linearly independent and spanning it is a basis with cardinality $2$. Hence, the dimension of the vector space is also $2$.
\end{enumerate}

\pagebreak
\section*{Q2}
% Restating the Star-Triangle law for resistance:
% \begin{mdframed}
% 	Consider a start with centre $x$ with edges to $y_0,y_1,y_2$. Then:
% 	\begin{align*}
% 		\gamma
% 		 & = \frac{c(x,y_0)c(x,y_1)c(x,y_2)}{c(x,y_0)+c(x,y_1)+c(x,y_2)}                       \\
% 		 & = \frac{1}{r(x,y_0)r(x,y_1)r(x,y_2)\left[ 1/r(x,y_0)+1/r(x,y_1)+1/r(x,y_2) \right]} \\
% 		 & = \frac1{r(x,y_0)r(x,y_1) + r(x,y_1)r(x,y_2) + r(x,y_2)r(x,y_0)}
% 	\end{align*}
% 	So $\set{y_i,y_{i+1}}$ where indices are taken mod $3$, has resistance:
% 	\begin{align*}
% 		r(y_i,y_{i+1}) & = \frac1{\gamma c(x,y_{i+2})}                                                                                                        \\
% 		               & = \frac{1}{r(x,y_{i+2})r(x,y_0)r(x,y_1)r(x,y_2)\left[ 1/r(x,y_0)+1/r(x,y_1)+1/r(x,y_2) \right]}                                      \\
% 		               & = r(x,y_{i+2}) \left[ \frac1{r(x,y_{i+0})r(x,y_{i+1})} + \frac1{r(x,y_{i+1})r(x,y_{i+2})} + \frac1{r(x,y_{i+2})r(x,y_{i+0})} \right] \\
% 		               & =  \frac{r(x,y_{i+2})}{r(x,y_{i+0})r(x,y_{i+1})} + \frac1{r(x,y_{i+1})} + \frac1{r(x,y_{i+0})}
% 	\end{align*}
% \end{mdframed}
Applying the series, parallel, and star triangle laws to calculate $\mathcal{C}(a\leftrightarrow z)$ for the given graph $G$, non-unit conductances are labelled:

\begin{tabular}{ccccccc}
	 &
	\begin{tikzpicture}
		\node[circle,fill,scale=0.5] (a) at (0,1) {};
		\node[circle,fill,scale=0.5] (b) at (1,1) {};
		\node[circle,fill,scale=0.5] (c) at (2,1) {};
		\node[circle,fill,scale=0.5] (d) at (3,1) {};
		\node[circle,fill,scale=0.5] (e) at (0,0) {};
		\node[circle,fill,scale=0.5] (f) at (1,0) {};
		\node[circle,fill,scale=0.5] (g) at (2,0) {};
		\node[circle,fill,scale=0.5] (z) at (3,0) {};

		\draw (a) -- (b) -- (c) -- (d);
		\draw (e) -- (f) -- (g) -- (z);
		\draw (a) -- (e) ;
		\draw (b) -- (f);
		\draw (c) -- (g);
		\draw (d) -- (z);
	\end{tikzpicture}
	 &
	$
		\begin{array}{c}
			\longrightarrow         \\[-0.5em]
			\text{\tiny Series Law} \\[2em]
		\end{array}
	$
	 &
	\begin{tikzpicture}[
			every edge quotes/.style = {auto, font=\tiny, sloped}
		]
		\node[circle,fill,scale=0.5] (a) at (0,1) {};
		\node[circle,fill,scale=0.5] (b) at (1,1) {};
		\node[circle,fill,scale=0.5] (c) at (2,1) {};


		\node[circle,fill,scale=0.5] (f) at (1,0) {};
		\node[circle,fill,scale=0.5] (g) at (2,0) {};
		\node[circle,fill,scale=0.5] (z) at (3,0) {};

		\draw (a) -- (b);
		\draw (g) -- (z);
		\draw (b) -- (c) -- (g) -- (f) -- (b);

		\draw (a) edge["$1/2$"] (f) ;
		\draw (c) edge["$1/2$"] (z) ;
	\end{tikzpicture}
	 &
	$
		\begin{array}{c}
			\longrightarrow                \\[-0.5em]
			\text{\tiny Star Triangle Law} \\[-0.5em]
			\text{\tiny$\gamma$ = 1/3  }   \\[1em]
		\end{array}
	$
	 &
	\begin{tikzpicture}[
			every edge quotes/.style = {auto, font=\tiny, sloped}
		]
		\node[circle,fill,scale=0.5] (a) at (0,1) {};
		\node[circle,fill,scale=0.5] (c) at (2,1) {};

		\node[circle,fill,scale=0.5] (f) at (1,0) {};
		\node[circle,fill,scale=0.5] (g) at (2,0) {};
		\node[circle,fill,scale=0.5] (z) at (3,0) {};


		\draw (f) -- (g) -- (z);
		\draw (c) -- (g);

		\path (a) edge[bend right] node[below,rotate=-45]  {\tiny$1/2$}    (f);

		\path (a) edge node[above]{\tiny${1/3}$} (c)
		      (c) edge node[above,rotate=45]{\tiny${1/3}$} (f)
		      (f) edge node[above,rotate=-45]{\tiny${1/3}$} (a);


		\draw (c) edge["$1/2$"] (z) ;


	\end{tikzpicture}    \\
	$
		\begin{array}{c}
			\longrightarrow                \\[-0.5em]
			\text{\tiny Star Triangle Law} \\[-0.5em]
			\text{\tiny$\gamma$ = 1/3  }   \\[1em]
		\end{array}
	$
	 &
	\begin{tikzpicture}
		\node[circle,fill,scale=0.5] (a) at (0,1) {};
		\node[circle,fill,scale=0.5] (c) at (2,1) {};

		\node[circle,fill,scale=0.5] (f) at (1,0) {};
		\node[circle,fill,scale=0.5] (z) at (3,0) {};


		\path (a) edge node[above]{\tiny${1/3}$} (c) 
		      (c) edge node[below,rotate=-45]{\tiny${1/3}$} (z)
		      (z) edge node[below]{\tiny${1/3}$} (f)
			  (f) edge node[above,rotate=-45]{\tiny${1/3}$} (a);


		\path (a) edge[bend right]
		          node[below,rotate=-45]{\tiny$1/2$} (f);
		
		\path (z) edge[bend right]
		          node[above,rotate=-45]{\tiny$1/2$} (c);

		\path (f) edge[bend left] node[above,rotate=45,yshift=-0.2em]{\tiny${1/3}$} (c)
		      (c) edge[bend left] node[above,rotate=45,yshift=-0.2em]{\tiny${1/3}$} (f);
	\end{tikzpicture}
	 &
	$
		\begin{array}{c}
			\longrightarrow           \\[-0.5em]
			\text{\tiny Parallel Law} \\[2em]
		\end{array}
	$
	 & \begin{tikzpicture}
		   \node[circle,fill,scale=0.5] (a) at (0,1) {};
		   \node[circle,fill,scale=0.5] (c) at (2,1) {};

		   \node[circle,fill,scale=0.5] (f) at (1,0) {};
		   \node[circle,fill,scale=0.5] (z) at (3,0) {};


		   \path (a) edge node[above]{\tiny$1/3$} (c)
		         (c) edge node[above,rotate=-45]{\tiny$5/6$} (z)
				 (z) edge node[below]{\tiny$1/3$} (f)
				 (f) edge node[below,rotate=-45]{\tiny$5/6$} (a);

		   \path (f) edge node[above,rotate=45]{\tiny{$2/3$}} (c);

		%    \draw (a) -- (c) -- (z) -- (f)--(a);
		%    \draw (c) -- (f);
	   \end{tikzpicture}
	 &
	$
		\begin{array}{c}
			\longrightarrow                \\[-0.5em]
			\text{\tiny Star Triangle Law} \\[-0.5em]
			\text{\tiny $\gamma=19/18$}    \\[1em]
		\end{array}
	$
	 &
	\begin{tikzpicture}
		\node[circle,fill,scale=0.5] (a) at (0,1) {};
		\node[circle,fill,scale=0.5] (b) at (1,1) {};
		\node[circle,fill,scale=0.5] (c) at (2,1) {};

		\node[circle,fill,scale=0.5] (f) at (1,0) {};
		\node[circle,fill,scale=0.5] (z) at (3,0) {};


		\path (a) edge node[above]{\tiny{$19/12$}} (b)
		      (b) edge node[above]{\tiny{$19/15$}} (c)
		      (c) edge node[above,rotate=-45]{\tiny{$5/6$}}  (z)
		      (z) edge node[below]{\tiny{$1/3$}}  (f);
		\path (b) edge node[left]{\tiny{$19/6$}} (f);
	\end{tikzpicture}
	\\
	$
		\begin{array}{c}
			\longrightarrow         \\[-0.5em]
			\text{\tiny Series Law} \\[2em]
		\end{array}
	$
	 &
	\begin{tikzpicture}
		\node[circle,fill,scale=0.5] (a) at (0,1) {};
		\node[circle,fill,scale=0.5] (b) at (1,1) {};
		\node[circle,fill,scale=0.5] (z) at (3,0) {};
		\path (a) edge             node[above]{\tiny{$19/12$}}                 (b)
		      (b) edge [bend left] node[above,rotate=-26.565]{\tiny{$95/189$}} (z)
		      (z) edge [bend left] node[below,rotate=-26.565]{\tiny{$19/63$}}  (b);


	\end{tikzpicture}
	 &
	$
		\begin{array}{c}
			\longrightarrow           \\[-0.5em]
			\text{\tiny Parallel Law} \\[2em]
		\end{array}
	$
	 &
	\begin{tikzpicture}
		\node[circle,fill,scale=0.5] (a) at (0,1) {};
		\node[circle,fill,scale=0.5] (b) at (1,1) {};
		\node[circle,fill,scale=0.5] (z) at (3,0) {};
		\path (a) edge node[above]{\tiny{19/12}}     (b)  
		      (b) edge node[above,rotate=-26.565]{\tiny{$152/189$}} (z);
	\end{tikzpicture}
	 &
	$
		\begin{array}{c}
			\longrightarrow         \\[-0.5em]
			\text{\tiny Series Law} \\[2em]
		\end{array}
	$
	 &
	\begin{tikzpicture}
		\node[circle,fill,scale=0.5] (a) at (0,1) {};
		\node[circle,fill,scale=0.5] (z) at (3,0) {};
		\path (a) edge node[above,rotate=-18.43]{\tiny{$8/15$}} (z);
	\end{tikzpicture}
\end{tabular}

Thus, $\mathcal{C}(a\leftrightarrow z)=8/15$ and $\mathcal{R}(a\leftrightarrow z)=15/8=1.875$.

\pagebreak

\section*{Q3}

Counter Example: Consider the following graph and shortest path shown in red. Let $v$ be a voltage function with $v(a)=0$ and $v(z)=1$. This graph can be reduced to a weighted graph using the Series/Parallel Laws: 

\begin{center}
	\begin{tikzpicture}
		\node[label={$a$}] (x) at (0,0) {};
		\node[label={$z$}] (z) at (5,0) {};
	
		\node[label={$\vdots$}] (z) at (1,0.5) {};
		\node[label={$\vdots$}] (z) at (2,0.5) {};
		\node[label={$\times 12$}] (z) at (1.5,0.5) {};
	
		\node[label={$\vdots$}] (z) at (3,-1.25) {};
		\node[label={$\vdots$}] (z) at (4,-1.25) {};
		\node[label={$\times 12$}] (z) at (3.5,-1.25) {};
	
		\node[circle,fill,scale=0.5] (a) at (0,0) {};
		\node[circle,fill,scale=0.5] (b) at (1,0) {};
		\node[circle,fill,scale=0.5] (c) at (2,0) {};
		\node[circle,fill,scale=0.5] (d) at (3,0) {};
		\node[circle,fill,scale=0.5] (e) at (4,0) {};
		\node[circle,fill,scale=0.5] (f) at (5,0) {};
		\draw[red] (a) -- (b) -- (c) -- (d) -- (e) -- (f);
	
		\node[circle,fill,scale=0.5] (b1) at (1,0.5) {};
		\node[circle,fill,scale=0.5] (c1) at (2,0.5) {};
		\node[circle,fill,scale=0.5] (b2) at (1,1.25) {};
		\node[circle,fill,scale=0.5] (c2) at (2,1.25) {};
		\draw (a) -- (b1) -- (c1) -- (d);
		\draw (a) -- (b2) -- (c2) -- (d);
	
		\node[circle,fill,scale=0.5] (d1) at (3,-0.5) {};
		\node[circle,fill,scale=0.5] (e1) at (4,-0.5) {};
		\node[circle,fill,scale=0.5] (d2) at (3,-1.25) {};
		\node[circle,fill,scale=0.5] (e2) at (4,-1.25) {};
		\draw (c) -- (d1) -- (e1) -- (f);
		\draw (c) -- (d2) -- (e2) -- (f);

		\node[label={[xshift=-0.5em, yshift=-2em] $x$}] (x) at (2,0) {};
		\node[label={[xshift=0.5em, yshift=0em] $y$}] (y) at (3,0) {};
	\end{tikzpicture}
	\qquad
	\begin{tikzpicture}
		\node[label={$a$}] (x) at (0,0) {};
		\node[label={$z$}] (z) at (5,0) {};
	
		\node[label={[xshift=-0.5em, yshift=-2em] $x$}] (x) at (2,0) {};
		\node[label={[xshift=0.5em, yshift=-2em] $y$}] (y) at (3,0) {};

		\node[circle,fill,scale=0.5] (a) at (0,0) {};
		\node[circle,fill,scale=0.5] (c) at (2,0) {};
		\node[circle,fill,scale=0.5] (d) at (3,0) {};
		\node[circle,fill,scale=0.5] (f) at (5,0) {};
	
		\draw[red] (a) edge["$0.5$"] (c) ;
		\draw[red] (d) edge["$0.5$"] (f) ;
		
		\draw[red] (c) edge["$1$"] (d) ;
		
		\path (a) edge [bend left, in=120, out=60, "$4$"] (d);
		\path (f) edge [bend left, in=120, out=60, "$4$"] (c);
		
	\end{tikzpicture}
\end{center}

Where the weights correspond to conductance, this preserves the effective conductance between the any pair of $a,x,y,z$ are preserved, thus, $v(x),v(y)$ are also preserved. 

From Ohm's Law: 
\begin{alignat*}{3}
	i(a,x) &= c(\set{a,x})\left[v(a)-v(x)\right] &\;\;=\;\;& -\frac 12v(x)        \\
	i(x,z) &= c(\set{x,z})\left[v(x)-v(z)\right] &\;\;=\;\;& 4v(x)-4              \\
	i(a,y) &= c(\set{a,y})\left[v(a)-v(y)\right] &\;\;=\;\;& -4v(y)               \\
	i(y,z) &= c(\set{y,z})\left[v(y)-v(z)\right] &\;\;=\;\;& \frac 12v(y)-\frac12 \\
	i(x,y) &= c(\set{x,y})\left[v(x)-v(y)\right] &\;\;=\;\;& v(x)-v(y)            
\end{alignat*}

Applying Kirchhoff's Node Law at $x,y$: 
\begin{alignat*}{3}
	     0 &= \sum_{z\in N(x)}i(z,x) = i(a,x)+i(y,x)+i(z,x) \\
	       &= -\frac 12v(x) - [v(x)-v(y)]- [4v(x)-4]        \\
	  v(y) &= \frac 12v(x) + v(x) + 4v(x)-4                 \\
	       &= \frac {11}2v(x)-4                             \\
	     0 &= \sum_{z\in N(y)}i(z,y) = i(a,y)+i(x,y)+i(z,y) \\
		   &=  \left[ -4v(y) \right]
		      +\left[ v(x)-v(y) \right]
		      -\left[ \frac 12 v(y) - \frac 12 \right]      \\
	  v(x) &= 4.5v(y) - \frac 12 
\end{alignat*}
Solving this system of two linear equations gives: 
\[
	\frac{74}{95}=v(x) > v(y)=\frac{27}{95}
\]


\pagebreak
\section*{Q4}
Let $H=(V_H,E_H)$ be a connected subgraph of $G=(V_G,E_G)$. Let $T_H,T_G$ be random spanning trees of $H,G$ respectively. For $e\in E_H$:

\begin{align*}
	\mathbb{P}[e\in T_H] & = \mathcal{R}(x \leftrightarrow y) & \text{By Theorem 3.20} \\
	                     & = \epsilon(i)
\end{align*}
For some unit strength current flow $i$ from $x$ to $y$ on $H$. Extend $i$ to $i_G$ on $G$ by defining:
\[
	i_G(x,y) = \begin{cases}
		i(x,y) & \text{When $\set{x,y}\in E_H$} \\
		0      & \text{Otherwise}
	\end{cases}
\]
Clearly $i_G$ is a flow (not necessarily a current flow) as it still satisfies Kirchhoff's node law, moreover it also has unit strength. So by Thompson's principle, a unit current flow $i'$ from $x$ to $y$ has:
$$\epsilon(i)=\sum_{e=\set{x,y}\in E_H}i(x,y)^2r(e)
=\sum_{e=\set{x,y}\in E_G}i_G(x,y)^2r(e)
=\epsilon(i_G)\geq \epsilon(i')=\mathcal{R}_G(x\leftrightarrow y)$$
Thus, we have shown:
\[
	\mathbb{P}[e\in T_H]  = \mathcal{R}(x \leftrightarrow y) = \epsilon(i)=\epsilon(i_G)\geq \epsilon(i')=\mathcal{R}_G(x\leftrightarrow y) = \mathbb{P}[e\in T_G]
\]

\section*{Q5}
\begin{enumerate}[label=(\alph*)]
	\item
		Let $G=(V,E)$ be a connected graph, and $i'$ a current flow from $a$ to $z$ of unit strength on $G-e=(V,E')$. Extend $i'$ to a flow $\theta$ on $G$ with $\theta(x,y)=\theta(y,x)=0$ for $\set{x,y}=e$ and $\theta(u,v)=i(u,v)$ for $\set{u,v}\in E'$. So:
	      \[
		      \varepsilon(\theta)
		      =\frac12\hspace{-0.5em}\sum_{\substack{u\in V\\ v\in N_G(x)}}\theta(u,v)^2r(\set{u,v})
		      =\frac12\hspace{-0.5em}\sum_{\substack{u\in V\\ v\in N_{G-e}(x)}}i'(u,v)^2r(\set{u,v}) + \theta(x,y)^2r(\set{x,y})
		      =\frac12\hspace{-0.5em}\sum_{\substack{u\in V\\ v\in N_{G-e}(x)}}i'(u,v)^2r(\set{u,v})
			  =\varepsilon(i')
	      \]
		So for a unit strength current flow $i$ from $a$ to $z$ on $G$, $\varepsilon(i)\leq\varepsilon(i')$ by Thompson's Principle. Thus:
		$$\mathcal{R}_G(a\leftrightarrow b)=\varepsilon(i)\leq\varepsilon(i')= \mathcal{R}_{G-e}(a\leftrightarrow b)$$

	\item 
		Consider any current flow $i$ from $a$ to $z$ on $G=(V,E)$. We claim that $i(x,y)\neq0$ implies that we can construct a path from $a$ to $z$ that contains both of $x,y$. Proof:
		\begin{mdframed}
			Let $P$ be a path with vertices $x_1=x,x_2=y$. Consider the algorithm where we repeatedly apply the following step until both endpoints of $P$ are in $\set{a,z}$:
			
			Let $x_1,\dots,x_{k+1}$ be the vertices of $P$. If $x_{k+1}\notin\set{a,z}$, by Kirchhoff's node law, there is some $w\in N(x_{k+1})$ with $i(x_{k+1},w)>0$, attempt to append $w$ to $P$. Otherwise, if $x_{k+1}\in\set{a,z}$ and $x_1\notin\set{a,z}$, by Kirchhoff's node law there is some $w'\in N(x_1)$ such that $i(w',x_1)>0$. Attempt to extend $P$ by prepending $w'$.

			On a finite $G$, the algorithm always succeeds or fails. To fail, adding $w$ or $w'$ to $P$ did not produce a path. Meaning $P$ already contained $w$ or $w'$. Hence, we have a cycle consisting entirely of edges with $i(x_{i},x_{i+1})>0$ and Kirchhoff's cycle law is not satisfied. Since $i$ is a current flow, this is a contradiction and the algorithm must always construct a path between $a$ and $z$ containing the original edge. ($a$ to $a$ or $z$ to $z$ aren't paths). 
		\end{mdframed}
		\pagebreak
	   
		\item For any unweighed $G=(V,E)$, the conductance function $c:E\to\RR$ is given by $c(e)=1$. So by Theorem 3.18 of the course book: 
		\[
			\operatorname{Comm}(a\leftrightarrow z) = 2 \left( \sum_{e\in} c(e)\right)	\mathcal{R}(a\leftrightarrow z) = 2\abs{E}\mathcal{R}(a\leftrightarrow z)
		\]
		Now consider some fixed shortest path $P$ between $a$ and $z$ with length $d(a,z)$. Apply (a) to remove every edge of $G$ not in $P$, by (a) this does not decrease the effective resistance between $a$ and $z$. IE, if $G'$ is the graph obtained, then:
		\[
			\mathcal{R}_{G}(a\leftrightarrow z)	
			\leq \mathcal{R}_{G'}(a\leftrightarrow z)
		\]
		Since each non-endpoint vertex of $P$ is of degree $2$ in $G'$, applying the series law $\mathcal{R}_{G'}(a\leftrightarrow z)$ is the sum of the resistances in $G$. Thus, $\mathcal{R}_{G'}(a\leftrightarrow z)=d(a,z)$ as the edges are of unit resistance. Thus: 
	\[
		\operatorname{Comm}(a\leftrightarrow z) = 2 \left( \sum_{e\in} c(e)\right)	\mathcal{R}(a\leftrightarrow z) = 2\abs{E}\mathcal{R}(a\leftrightarrow z) \leq 2\abs{E}d(a,z)	
	\]


\end{enumerate}

\section*{Q6}

% Consider current flows $i,i'$ of unit strength on the weights $r$, $r'$ respectively. Hence:

Consider a unit current flow $i''$ on the resistances $\frac{r+r'}2$ from $a$ to $z$. Then:
\begin{align*}
	\mathcal{R}_{\frac{r+r'}2}(a\leftrightarrow z)
	 & = \varepsilon_{\frac{r+r'}2}(i'') &\text{By Lemma 3.29} \\
	 & = \frac 12 \sum_{\substack{x\in V\\ y\in N(x)}} i''(x,y)^2 \frac{r(\set{x,y})+r'(\set{x,y})}{2} \\
	 & = \frac 14 \sum_{\substack{x\in V\\ y\in N(x)}} i''(x,y)^2 r(\set{x,y})
	   + \frac 14 \sum_{\substack{x\in V\\ y\in N(x)}} i''(x,y)^2 r'(\set{x,y})\\
	 & = \frac12 \left[ \varepsilon_{r}(i'') + \varepsilon_{r'}(i'') \right]\\
	 & \geq \frac12 \left[ \varepsilon_{r}(i) + \varepsilon_{r'}(i') \right] &\text{By Thompson's Principle}\\
	 & = \frac12 \left[ 
	   \mathcal{R}_{r}(a\leftrightarrow z) 
	 + \mathcal{R}_{r'}(a\leftrightarrow z) 
	 \right]&\text{By Lemma 3.29}
\end{align*}
Where $i$ and $i'$ are unit strength current flows on $r$ and $r'$ respectively and Thompson's Principle is applicable as $i''$ also is a unit strength flow (but not necessarily a current flow) as the strength of a flow is independent of resistance. 
\end{document}
