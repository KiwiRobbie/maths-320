
\documentclass[a4paper, 11pt]{article}

% Fonts 
\usepackage{opensans}
\usepackage{amsfonts}
\usepackage{montserrat}
\usepackage{amsmath}

\usepackage[mathrm=sym]{unicode-math}

\setmainfont{opensans}
\setmathfont{Fira Math}

\newfontfamily{\montserrateb}{Montserrat SemiBold}
\newfontfamily{\montserratb}{Montserrat Bold}
\newfontfamily{\montserrat}{Montserrat Regular}
\newfontfamily{\montserratl}{Montserrat Light}
\DeclareMathAlphabet{\mathcal}{OMS}{cmbrs}{m}{n}

% \autoref
\usepackage{hyperref}

% Use for [H] option for figures to force in text placement
\usepackage{float}

% Captioning figures
\usepackage{caption}

% Subfigures
\usepackage{subcaption}

% For extending contents beyond margins
\usepackage{scrextend}

% For tables \midrule ect
\usepackage{booktabs}

% Colours
\usepackage[table,xcdraw]{xcolor}
\definecolor{accentcolor}{HTML}{6332a8}

% Change label in enumerate 
\usepackage{enumitem}

% Section settings
\usepackage{titlesec}
\titleformat{\section}
{\LARGE\montserrateb}
{\thesection.}{0.5em}{}

\titleformat{\subsection}
{\large\montserratb}
{\thesubsection.}{0.5em}{}

% Adjust document dimensions
\ExecuteOptions{a4paper}
\addtolength{\oddsidemargin}{-3cm}
\addtolength{\evensidemargin}{-3cm}
\addtolength{\topmargin}{-3cm}
\addtolength{\textwidth}{6cm}
\addtolength{\textheight}{4.5cm}
\addtolength{\textheight}{1.5cm}
\addtolength{\headsep}{-0.5cm}
% \addtolength{\footskip}{-1cm}
\parindent0pt
\parskip=4pt



\usepackage{matlab-prettifier}
\usepackage{graphicx}
\usepackage{mdframed}

% Creates coloured title box
\newcommand{\thetop}[5]{
	\begin{addmargin}[\oddsidemargin]{\oddsidemargin}
		\colorbox{#5}{\color{white}
			\hbox to \paperwidth{
				\vbox {
					\begin{center}
						{\large\montserratl #1}\\
						\vspace{4pt}
						{\huge\montserratb #2}\\
						{\montserratb #3}\\
						\vspace{-0.5em}
						\rule{20em}{1pt}

						{\large\montserratl
							#4
						}
					\end{center}
				}
			}
		}
	\end{addmargin}
}

\newcommand{\NN}{\mathbb{N}}
\newcommand{\ZZ}{\mathbb{Z}}
\newcommand{\RR}{\mathbb{R}}
\newcommand{\CC}{\mathbb{C}}
\newcommand{\dydt}{\frac{dy}{dt}}
\newcommand{\dxdt}{\frac{dx}{dt}}
\def\set#1{\left\{ #1 \right\}}
\def\eval#1#2{\left\ #1\right|_{#2}}

\def\pp#1#2{\frac{\partial #1}{\partial #2}}
\def\dd#1#2{\frac{\,d#1}{\,d#2}}
\def\abs#1{\left|#1\right|}
\def\conj#1{\overline{#1}}

\usepackage{multicol}
\usepackage{pgfplots}

\begin{document}
\thetop{Robert Christie}{MATHS 340}{S2 2023}{Assignment 3\\Due: 26-08-2023}{accentcolor}

\begin{multicols}{2}
	\section*{Q1}

	\begin{enumerate}[label=(\alph*)]
		\item Let $z=x+iy$, then:
		      $$\abs{z+\conj z} = \abs{2x}$$
		      Hence the region corresponds to $\abs{x}>5$ (Due to the strict inequality boundary points are not included.)


		      \begin{center}
			      \begin{tikzpicture}
				      \begin{axis}[
						      xmin=-11,
						      xmax=11,
						      ymin=-11,
						      ymax=11,
						      axis equal,
						      axis lines=middle,
						      xlabel=Re($z$),
						      ylabel=Im($z$),
						      disabledatascaling]

					      \fill [opacity=0.3] (-12, 5) rectangle (12,  11);
					      \fill [opacity=0.3] (-12,-5) rectangle (12, -11);

					      \draw [opacity=0.8,dashed] (-11, 5) -- (12,  5);
					      \draw [opacity=0.8,dashed] (-11,-5) -- (12, -5);
				      \end{axis}
			      \end{tikzpicture}
		      \end{center}
		\item Let $z=re^{i\theta}$. Note that at $z=0$, the region's existence is not well-defined. However, in the limit as $z\to 0$, $\frac 1{\abs z}\to +\infty$, while $\abs{\conj z}\to 0$, so we redefine the region $R$ as:
		      $$R=\set{z\in \CC \backslash \set{0} \,\Big|\, \frac{1}{\abs z} \leq \abs{\conj z}}\subset \CC$$
		      Since $\conj{z}=re^{-i\theta}$, $\abs z= \abs{\conj z}=r$, and: $$\frac 1r\leq r\iff 1\leq r^2 \iff 1\leq r$$
		      Thus the region is given by:
		      $$R=\set{re^{i\theta}\in \CC \backslash \set{0} \,\Big|\, 1\leq r}$$

		      \begin{center}
			      \begin{tikzpicture}
				      \begin{axis}[
						      xmin=-4.5,
						      xmax=4.5,
						      ymin=-4.5,
						      ymax=4.5,
						      axis equal,
						      axis lines=middle,
						      xlabel=Re($z$),
						      ylabel=Im($z$),
						      disabledatascaling]

					      \draw [opacity=0.8] (0,0) circle [radius=1];
					      \path [fill=black, even odd rule, fill opacity = 0.3] (-4.5,-4.5) rectangle (4.5,4.5) (0,0) circle (1);



					      %   \fill [opacity=0.3] (-12,-5) rectangle (12, -11);

					      %   \draw [opacity=0.3,dashed] (-11, 5) -- (12,  5);
					      %   \draw [opacity=0.3,dashed] (-11,-5) -- (12, -5);
				      \end{axis}
			      \end{tikzpicture}
		      \end{center}

		\item Let $z=x+yi$, then:
		      \begin{align*}
			      R_1 & =\set{x+yi\in\CC \,|\, \sqrt{(x-1)^2+y^2} \leq 1}  \\
			      R_2 & =\set{x+yi\in\CC \,|\, \sqrt{(x-1)^2+(y+1)^2} > 1}
		      \end{align*}
		      Thus we see that $R_1$, is a closed circle of radius $1$ about $z=1+0i$ and $R_2$ is all of $\CC$ excluding a circle of radius $1$ around $z=1-1i$, the boundary of $R_2$ is open.

		      Plotting $R_1$ (blue) and $R_2$ (red):

		      \begin{center}
			      \begin{tikzpicture}
				      \begin{axis}[
						      xmin=-4.5,
						      xmax=4.5,
						      ymin=-4.5,
						      ymax=4.5,
						      axis equal,
						      axis lines=middle,
						      xlabel=Re($z$),
						      ylabel=Im($z$),
						      disabledatascaling]

					      %   \draw [opacity=0.8] (0,0) circle [radius=1];
					      \draw [fill=blue, even odd rule, fill opacity = 0.3]  (1,0) circle (1);
					      \path [fill=red, even odd rule, fill opacity = 0.3] (-4.5,-4.5) rectangle (4.5,4.5) (1,-1) circle (1);
					      \draw [opacity=0.8,dashed]  (1,-1) circle (1);



					      %   \fill [opacity=0.3] (-12,-5) rectangle (12, -11);

					      %   \draw [opacity=0.3,dashed] (-11, 5) -- (12,  5);
					      %   \draw [opacity=0.3,dashed] (-11,-5) -- (12, -5);
				      \end{axis}
			      \end{tikzpicture}
		      \end{center}
		      Thus, plotting the intersection of the regions:

		      \begin{center}
			      \begin{tikzpicture}
				      \begin{axis}[
						      xmin=-4.5,
						      xmax=4.5,
						      ymin=-4.5,
						      ymax=4.5,
						      axis equal,
						      axis lines=middle,
						      xlabel=Re($z$),
						      ylabel=Im($z$),
						      disabledatascaling]

					      %   \draw [opacity=0.8] (0,0) circle [radius=1];
					      %   \draw [fill=blue, even odd rule, fill opacity = 0.3]  (1,0) circle (1);
					      %   \path [fill=red, even odd rule, fill opacity = 0.3] (-4.5,-4.5) rectangle (4.5,4.5) (1,-1) circle (1);
					      %   \draw [opacity=0.8,dashed]  (1,-1) circle (1);

					      \begin{scope}
						      \path [clip] (-4.5,-4.5) rectangle (4.5,4.5) (1,-1) circle (1);
						      \fill[fill=black, even odd rule, fill opacity = 0.3] (1,0) circle (1);
						      \draw [opacity=0.8]  (1,0) circle (1);
					      \end{scope}
					      \begin{scope}
						      \clip  (1,0) circle (1);
						      \draw [opacity=0.8,dashed]  (1,-1) circle (1);
					      \end{scope}
				      \end{axis}
			      \end{tikzpicture}
		      \end{center}
		      Note that the points where the two boundaries meet are not included in the region as they are closer that $1$ to $1-i$ and thus not part of $R_2$ and hence not present in $R_1\cap R_2$.

	\end{enumerate}
\end{multicols}
\end{document}
