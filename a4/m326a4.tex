
\documentclass[a4paper, 11pt]{article}

% Fonts 
\usepackage{opensans}
\usepackage{amsfonts}
\usepackage{montserrat}
\usepackage{amsmath}

\setmainfont{opensans}
% \usepackage[mathrm=sym]{unicode-math}
\usepackage{lato-math}
% \setmathfont[Path=font/,mathrm=sym]{FiraMath-Regular}
% \setmathfont[Path=font/,mathrm=sym]{LatoMath}

\newfontfamily{\montserrateb}{Montserrat SemiBold}
\newfontfamily{\montserratb}{Montserrat Bold}
\newfontfamily{\montserrat}{Montserrat Regular}
\newfontfamily{\montserratl}{Montserrat Light}
% \DeclareMathAlphabet{\mathcal}{OMS}{cmbrs}{m}{n}

% \usepackage[mathrm=sym]{unicode-math}
% \setmainfont{opensans}
% \setmathfont{Fira Math}

% \newfontfamily{\montserrateb}{Montserrat SemiBold}
% \newfontfamily{\montserratb}{Montserrat Bold}
% \newfontfamily{\montserrat}{Montserrat Regular}
% \newfontfamily{\montserratl}{Montserrat Light}
% \DeclareMathAlphabet{\mathcal}{OMS}{cmbrs}{m}{n}
% \setmathfont{Latin Modern Math}[range={\vdots}]

% \autoref
\usepackage{hyperref}

% Use for [H] option for figures to force in text placement
\usepackage{float}

% Captioning figures
\usepackage{caption}

% Subfigures
\usepackage{subcaption}

% For extending contents beyond margins
\usepackage{scrextend}

% For tables \midrule ect
\usepackage{booktabs}

% Colours
\usepackage[table,xcdraw]{xcolor}
\definecolor{accentcolor}{HTML}{a13640}

% Change label in enumerate 
\usepackage{enumitem}

% Section settings
\usepackage{titlesec}
\titleformat{\section}
{\LARGE\montserrateb}
{\thesection.}{0.5em}{}

\titleformat{\subsection}
{\large\montserratb}
{\thesubsection.}{0.5em}{}

% Adjust document dimensions
\ExecuteOptions{a4paper}
\addtolength{\oddsidemargin}{-3cm}
\addtolength{\evensidemargin}{-3cm}
\addtolength{\topmargin}{-3cm}
\addtolength{\textwidth}{6cm}
\addtolength{\textheight}{4.5cm}
\addtolength{\textheight}{1.5cm}
\addtolength{\headsep}{-0.5cm}
% \addtolength{\footskip}{-1cm}
\parindent0pt
\parskip=4pt



\usepackage{matlab-prettifier}
\usepackage{graphicx}
\usepackage{mdframed}

% Creates coloured title box
\newcommand{\thetop}[5]{
	\begin{addmargin}[\oddsidemargin]{\oddsidemargin}
		\colorbox{#5}{\color{white}
			\hbox to \paperwidth{
				\vbox {
					\begin{center}
						{\large\montserratl #1}\\
						\vspace{4pt}
						{\huge\montserratb #2}\\
						{\montserratb #3}\\
						\vspace{-0.5em}
						\rule{20em}{1pt}

						{\large\montserratl
							#4
						}
					\end{center}
				}
			}
		}
	\end{addmargin}
}

\newcommand{\NN}{\mathbb{N}}
\newcommand{\ZZ}{\mathbb{Z}}
\newcommand{\RR}{\mathbb{R}}
\newcommand{\CC}{\mathbb{C}}
\newcommand{\dydt}{\frac{dy}{dt}}
\newcommand{\dxdt}{\frac{dx}{dt}}
\def\set#1{\left\{ #1 \right\}}
\def\eval#1#2{\left\ #1\right|_{#2}}

\def\pp#1#2{\frac{\partial #1}{\partial #2}}
\def\dd#1#2{\frac{\,d#1}{\,d#2}}
\def\abs#1{\left|#1\right|}
\def\conj#1{\overline{#1}}

\usepackage{multicol}
\usepackage{tikz}
\usepackage{pgfplots}
\usetikzlibrary {graphs,graphdrawing} \usegdlibrary {force} 
\usetikzlibrary{graphs.standard}
\usetikzlibrary{positioning, 
                quotes}
\usegdlibrary {circular}

\usepackage{pst-platon}

\usepackage{comment}
\usepackage{cancel}

\begin{comment}
	Q1: Good
	Q2  needs time
	Q3: 
	Q4: Good 
	Q5
	Q6: Good 
\end{comment}

\begin{document}
\thetop{Robert Christie}{MATHS 326}{S1 2024}{Assignment 4\\Due: 24-05-2024}{accentcolor}

% \begin{multicols*}{2}
\section*{Q1}
\begin{enumerate}[label=(\alph*)]
	\item 
		We are given $r = 4, k = 11, \lambda = 2$, assume that these are parameters for a balanced design, by \textit{Theorem 4.10}:
		\[
			bk=vr \implies 11 b = 4 v
		\]
		\[
			\lambda (v-1) = r (k-1) \implies 2 (v- 1) = 4(11-1) \implies 2v - 2 = 40 \implies v = 21 
		\]
		Contradiction as this implies $b=\frac{4\cdot 21}{11}$ which is not an integer as neither $4$ nor $21$ have a prime factor of $11$. Thus, no balanced block design has these parameters.

	\item We are given $b=30,r=6,k=5$, assume that these are parameters for a balanced design, by \textit{Theorem 4.10}:
	\[
		bk=vr \implies  30\cdot 5 = 6v \implies v =25
	\]
	\[
		\lambda (v-1) = r (k-1) \implies  \lambda(24) = 6(4)\implies \lambda = 1
	\]

	Let $X=\ZZ_5^2$ 



	\item 
		We are given $v=46, b=10, \lambda = 2$, assume that these are parameters for a balanced design, by \textit{Theorem 4.10}:
		\[
			bk=vr \implies  10k = 46r \implies k=4.6r
		\]
		\[
			\lambda (v-1) = r (k-1) \implies  2(46-1) = r(k-1)\implies 0 = 4.6r^2 - r - 90
		\]
		Solving for possible values of $r$ using the quadratic equation: 
		\[
			r = \frac{1\pm\sqrt{1657}}{4.6}
		\]
		Which has no integer solutions as $40^2<1657<41^2$, hence $\sqrt{1657}$ is irrational.

		This is a contradiction so no balanced block design has these parameters.
\end{enumerate}

\pagebreak
\section*{Q2}
 Assume that there is a BIBD for $v=b=40$ with parameters $(v,b,r,k,\lambda)$. Then since $vb=rk$ we have $k=r$. Thus: 
\[
	\lambda(v-1) = r(k-1) \implies 39\lambda = r(r-1) = k(k-1)
\]

Since the design is incomplete, $r,k,\lambda\leq39$: 
% \[
	% \lambda(v-1)=r(k-1)\implies r(r-1)\leq39^2 \implies k,r\leq 39
% \]

Since we have $\lambda = k(k-1)/39$:
$$k-\lambda\in\set{1,2,4,9,16,25,36}$$
$$(39k- k(k-1))/39\in\set{1,2,4,9,16,25,36}$$




We can factorise $39=3\cdot 13$ and $\lambda=\lambda_1\lambda_2$. This gives the following cases: 

\begin{center}
	\begin{tabular}{c|cc|c}
		Case& $r$           & $r-1$         &                              \\\midrule
		A	& $39\lambda_1$ & $\lambda_2  $ & $39\lambda_1=\lambda_2  +1$  \\
		B	& $13\lambda_1$ & $3\lambda_2 $ & $13\lambda_1=3\lambda_2 +1$  \\
		C	& $3\lambda_1 $ & $13\lambda_2$ & $3\lambda_1 =13\lambda_2+1$  \\
		D	& $\lambda_1  $ & $39\lambda_2$ & $\lambda_1  =39\lambda_2+1$  \\
		\end{tabular}
\end{center}

Solving for $\lambda$ in each of these cases:

\begin{mdframed}
	\centering
	\begin{minipage}{0.4\textwidth}
		\centering
		Case A
		\begin{align*}
			\lambda_1=1    &\implies \lambda_2=38\\
			\lambda_1\geq 2&\implies \lambda>39
		\end{align*}
		\[\implies \lambda\in\set{38}\]
	\end{minipage}\hfill
	\begin{minipage}{0.4\textwidth}
		\centering
		Case B
		\begin{align*}
			\lambda_1=1    &\implies \lambda_2=4\\
			\lambda_1=2    &\implies \lambda_2=25/3\notin\ZZ\\
			\lambda_1\geq3 &\implies \lambda>39
		\end{align*}
		\[\implies \lambda\in\set{4}\]
	\end{minipage}
	
	\vspace{2em}
	
	\begin{minipage}{0.4\textwidth}
		\centering
		Case C
		\begin{align*}
			\lambda_2=1    &\implies \lambda_1=14/3\notin\ZZ\\
			\lambda_2=2    &\implies \lambda_1=9\\
			\lambda_2=3    &\implies \lambda_1=40/3\notin\ZZ\\
			\lambda_2\geq 3&\implies \lambda>39
		\end{align*}
		\[\implies \lambda\in\set{18}\]
	\end{minipage}\hfill
	\begin{minipage}{0.4\textwidth}
		\centering
		Case D
		\begin{align*}
			\lambda_2\geq1 &\implies \lambda>39
		\end{align*}
		\[\implies \lambda\in\emptyset\]
	\end{minipage}
\end{mdframed}

So we must have $\lambda\in\set{4,18,38}$.

% \begin{tabular}{cc|cc}
% 	$r$         & $r-1$       & $r=$    & $\lambda=$   \\\midrule
% 	$39$        & $\lambda$   & $r=39$  & $\lambda=38$ \\ 
% 	$\lambda$   & $39$        & $r=40$  & $\lambda=40$ \\ 
% 	$13$        & $3\lambda$  & $r=13$  & $\lambda=4$  \\ 
% 	$3\lambda$  & $13$        & $r=14$  & $\lambda\notin\ZZ$\\ 
% 	$3$         & $13\lambda$ & $r=3$   & $\lambda\notin\ZZ$\\ 
% 	$13\lambda$ & $3$         & $r=4$   & $\lambda\notin\ZZ$\\ 
% \end{tabular}

\pagebreak
\section*{Q3}
\begin{enumerate}[label=(\alph*)]
	\item CLAIM: Every pair of blocks has at most 1 common vertex. (FALSE!!!)



	\item 
\end{enumerate}

\section*{Q4}
First we verify that the construction can be performed. By P3, there are at least $4$ points, and by $P1$, any distinct pair of these points is on a unique line. Thus, there is a line $\ell$ to remove. 

We check that the axioms for an affine plane hold: 

\begin{itemize}
	\item
		 A1: Any two points in the construction already existed on some unique line $m$ in the projective plane, we have $m\neq\ell$ otherwise we would have removed the points. Hence, the line $m$ is present in the construction, lastly, no other lines have become incident with the points so $m$ is the unique line incident with both points.  



	\item 
		Consider any point $p$ and line $m'$ in the constructed plane such that $m'$ is not incident on $p$. 

		Clearly $m'$ is distinct from $\ell$, by P2, $m$ and $\ell$ have a unique common point $q$. Now by P1, $q$ and $p$ lie on a unique line $m$. Since $q$ is the only common point of $m$ and $m'$, and is removed in the constructed plane, $m\cap m'=\emptyset$. 


		\textcolor{red}{Show $m$ is unique.}

		\begin{center}
		\begin{tikzpicture}
			\draw[red] (-2,0) -- (2,0);
			\draw (0,-3) -- (0,1);
			\draw (1.5,-3) -- (-0.5,1);

			\node[label={[above,right,red]$q$}] (q) at (0,0) {};
			
			\node[label={[above,right]$p$}] (p) at (0,-2) {};
			\node[label={[above,right]$p'$}] (P) at (1,-2) {};

			\fill[red] (q) circle (2pt);
			\fill (p) circle (2pt);
			\fill (P) circle (2pt);

			\node[label={[below,right]$m$}] at (0,-3) {};
			\node[label={[below,right]$m'$}] at (1.5,-3) {};
			\node[label={[above,right,red]$\ell$}] at (2,0) {};
		\end{tikzpicture}
		\end{center}


	% CL: Given a point $p$ and a line $l$ not through $p$, there is exactly one line $l'$ through $p$ with $l\cap l'\neq \emptyset$.


		
	\item 
		A3: Direct proof, 
		% A3: Assume A3 does not old, then either: 
		% \begin{itemize}
		% 	\item There are less than $4$ points. 
		% 	\item $3$ points are collinear in any set of $4$ points. 
		% \end{itemize}
\end{itemize}

\pagebreak
\section*{Q5}
Consider the set $\set{L_1,\dots,L_6}$ of order $7$ Latin squares with entries $(L_k)_{ij}= i+kj$ mod $7$. Verifying that this is a set of Latin squares: 

\begin{mdframed}
	\begin{minipage}{0.5\textwidth}
		\begin{alignat*}{3}
			&& (L_k)_{ij} &= (L_k)_{ij'}	\\
	\implies&&i+kj        &= i+kj'          \\
	\implies&&kj          &= kj'            \\
	\implies&& j          &= j' &\text{Divide by $k$}            \\
	\end{alignat*}
	\end{minipage}
	\begin{minipage}{0.5\textwidth}
		\begin{alignat*}{2}
			&& (L_k)_{ij} &= (L_k)_{i'j}	\\
		\implies&&i+kj    &= i'+kj          \\
		\implies&&i       &= i'            \\
		\end{alignat*}
	\end{minipage}
	
	Note that we can divide by $k$ in mod $7$ as $1,\dots,6$ are not zero divisors.
\end{mdframed}

Assume that $k\neq k'$, and $(i,j)\neq (i',j')$, now for the sake of contradiction assume that: 

\begin{align*}
	\Big((L_k)_{ij}, (L_{k'})_{ij}\Big) &= \Big((L_k)_{i'j'}, (L_{k'})_{i'j'}\Big)\\
	\Big(i+kj, i+k'j\Big) &= \Big(i'+kj', i'+k'j'\Big)\\
	        \Big(0,0\Big) &= \Big((i-i')+k(j-j'), (i-i')+k'(j-j')\Big) \\
	     (i-i') + k(j-j') &=(i-i') + k'(j-j') \\
	              k(j-j') &= k'(j-j') \\
	                    0 &=(k-k')(j-j') \\
\end{align*}

However, the only zero divisor in mod $7$ is $0$, thus either $k-k'=0$, or $j-j'=0$. Since we assumed $k-k'=0$, we must have $j=j'$. However, we have that: 
\[
	\Big(0,0\Big) = \Big((i-i')+k(j-j'), (i-i')+k'(j-j')\Big) \implies 0 =i-i'\implies i=i'
\]
This is a contradiction. Thus, $\set{L_1,\dots,L_6}$ are a set of $6$ MOLS of order $7$. 

% To verify the set is mutually orthogonal,

\pagebreak
\section*{Q6}
\begin{enumerate}[label=(\alph*)]
	\item The square was completed in the following order:
	\begin{itemize}
		\item The \textcolor{gray}{gray} cells were given. 
		\item The \textcolor{blue}{blue} must be some permutation of $3,4,5$ and can be re-ordered by interchanging rows, order chosen WLOG. 
		\item The \textcolor{violet}{violet} cells must also be some permutation of $3,4,5$ distinct from the ordering of the blue cells. There are two options, the other failed to complete the square. 
		\item The \textcolor{cyan}{cyan} cell, had to be either $1$ or $2$, since no remaining cells are constrained by a $1$ or $2$, the choice is made WLOG.
		\item Each cell with a single remaining possibility was filled until the square was complete. 
	\end{itemize}



	\begin{center}
		\begin{tikzpicture}
			\node[gray] at (0, 0) {$1$};
			\node[gray] at (1, 0) {$2$};
			\node[gray] at (0,-1) {$2$};
			\node[gray] at (1,-1) {$1$};

			\node[gray] at (2, 0) {$3$};
			\node[gray] at (3, 0) {$4$};
			\node[gray] at (4, 0) {$5$};
			\node[gray] at (2,-1) {$5$};
			\node[gray] at (3,-1) {$3$};
			\node[gray] at (4,-1) {$4$};

			% Could re-order rows to reach same result
			\node[blue] at (0,-2) {$3$};
			\node[blue] at (0,-3) {$4$};
			\node[blue] at (0,-4) {$5$};

			% Seems symmetrical (offset 1 failed)
			\node[violet] at (1,-2) {$4$};
			\node[violet] at (1,-3) {$5$};
			\node[violet] at (1,-4) {$3$};


			% WLOG, must be 1 or 2
			\node[cyan] at (2,-2) {$1$};
			
			% Must be 
			\node at (3,-2) {$5$};
			\node at (4,-2) {$2$};
			\node at (2,-3) {$2$};
			\node at (2,-4) {$4$};
			\node at (3,-3) {$1$};
			\node at (4,-4) {$1$};
			\node at (4,-3) {$3$};
			\node at (3,-4) {$2$};

			% \node at (2,-3) {$1$};




			% % Only two options
			% \node at (3,-2) {$1$};
			
			% % Only options
			% \node at (2,-2) {$4$};
			% \node at (4,-2) {$2$};

			% \node at (4,-3) {$1$};
			% \node at (3,-3) {$2$};

	
	
	
	
			\coordinate (Q) at (-0.5, 0.5);
			
			\draw (Q)+(0, 0) -- +(5,0);
			\draw (Q)+(0,-1) -- +(5,-1);
			\draw (Q)+(0,-2) -- +(5,-2);
			\draw (Q)+(0,-3) -- +(5,-3);
			\draw (Q)+(0,-4) -- +(5,-4);
			\draw (Q)+(0,-5) -- +(5,-5);
	
			\draw (Q)+(0, 0) -- +(0,-5);
			\draw (Q)+(1, 0) -- +(1,-5);
			\draw (Q)+(2, 0) -- +(2,-5);
			\draw (Q)+(3, 0) -- +(3,-5);
			\draw (Q)+(4, 0) -- +(4,-5);
			\draw (Q)+(5, 0) -- +(5,-5);
		\end{tikzpicture}
	\end{center}

		
	
	\item A Latin square of order $5$ has an orthogonal mate if and only if it contains $5$ disjoint traversals. Assume that there exist $5$ disjoint traversals of some completion. 
	
	Consider the top left $2\times 2$ region, each cell must be in a different traversal: 
	\begin{mdframed}
		The region contains the following cells: 
		\begin{center}
		\begin{tikzpicture}
			\node at (0, 0) {$1$};
			\node at (1, 0) {$2$};
			\node at (0,-1) {$2$};
			\node at (1,-1) {$1$};
	
			\coordinate (Q) at (-0.5, 0.5);
			
			\draw (Q)+(0, 0) -- +(2, 0);
			\draw (Q)+(0,-1) -- +(2,-1);
			\draw (Q)+(0,-2) -- +(2,-2);

			\draw (Q)+(0,0) -- +(0,-2);
			\draw (Q)+(1,0) -- +(1,-2);
			\draw (Q)+(2,0) -- +(2,-2);
		\end{tikzpicture}
		\end{center}

		If two cells are in the same traversal, they cannot be in the same row/column, thus they must be diagonal (in the $2\times 2$ region). All diagonal entries are the same, so they cannot be part of the same traversal. 

		Thus, each cell in the $2\times 2$ region is part of a distinct traversal. 
	\end{mdframed}

	Label the traversals $A,B,C,D,E$, WLOG we can fix the traversal that each of the cells in the $2\times 2$ region are part of. Since each traversal appears once in each row/column, we can deduce which traversals the cells in each region must be assigned to: 

	\vspace{1em}

	\hfill\begin{tikzpicture}
		\node at (0,0) {$A$};
		\node at (1,0) {$B$};
		\node at (0,-1) {$C$};
		\node at (1,-1) {$D$};

		\node at (3,0) {$C,D,E$};
		\node at (3,-1) {$A,B,E$};

		\node[rotate=-90] at (0,-3) {$B,D,E$};
		\node[rotate=-90] at (1,-3) {$A,C,E$};

		\node[label={[align=center]$A,A,B,B$\\$C,C,D,D,E$}] at (3,-3.5) {};


		\coordinate (Q) at (-0.5, 0.5);
		
		\draw (Q)+(0, 0) -- +(5,0);
		\draw (Q)+(0,-1) -- +(5,-1);
		\draw (Q)+(0,-2) -- +(5,-2);
		\draw (Q)+(0,-5) -- +(5,-5);

		\draw (Q)+(0, 0) -- +(0,-5);
		\draw (Q)+(1, 0) -- +(1,-5);
		\draw (Q)+(2, 0) -- +(2,-5);
		\draw (Q)+(5, 0) -- +(5,-5);
	\end{tikzpicture}\hfill\vspace{0pt}\begin{minipage}[b]{0.5\textwidth}
		Notice that to be a traversal $E$ should contain both cells with values $1$ and $2$, however none of the $E$'s in first two rows/columns could contain a $1$ or $2$. This only leaves a single $E$ in the bottom right $3\times3$ region. So it is impossible for the $E$ traversal to contain both a $1$ and $2$. 

		\vspace{1em}
		
		Therefore, it is impossible for any completion to contain $5$ distinct traversals and thus no completion has an orthogonal mate. 

		\vspace{1em}
	\end{minipage}
	
	

% \pagebreak


% 	% By swapping rows and columns, we can fix an ordering of the top row and left column, this also forces the placement of the $E$, since traversals contains a cell of each value once and cannot be in the same column. 


% 	Now consider the relative positions of the $E$ traversals in top right $2\times 3$ region: 

% 	\begin{center}
% 		\begin{tikzpicture}
% 			\node at (0, 0) {$3$};
% 			\node at (1, 0) {$4$};
% 			\node at (2, 0) {$5$};
% 			\node at (0,-1) {$5$};
% 			\node at (1,-1) {$3$};
% 			\node at (2,-1) {$4$};
	
% 			\coordinate (Q) at (-0.5, 0.5);
			
% 			\draw (Q)+(0, 0) -- +(3, 0);
% 			\draw (Q)+(0,-1) -- +(3,-1);
% 			\draw (Q)+(0,-2) -- +(3,-2);

% 			\draw (Q)+(0,0) -- +(0,-2);
% 			\draw (Q)+(1,0) -- +(1,-2);
% 			\draw (Q)+(2,0) -- +(2,-2);
% 			\draw (Q)+(3,0) -- +(3,-2);
% 		\end{tikzpicture}
% 	\end{center}
% 	We can let the $5$ in the top row be a part of the $E$ traversal WLOG, if instead $3$ or $4$ was chosen to be a part of $E$, we could re-order the columns by rotating to the right, then re-labeling the cells, preserving the relationship between cells. 

% 	This also determines the middle bottom cell to be in the $E$ traversal, since this is the only cell with a different value and column to the first $E$ chosen. We can also re-order the rows so that the first column has increasing cell values: 


% \hfill\begin{tikzpicture}
% 			\node at (0, 0) {$1$};
% 			\node at (1, 0) {$2$};
% 			\node at (0,-1) {$2$};
% 			\node at (1,-1) {$1$};

% 			\node at (2, 0) {$3$};
% 			\node at (3, 0) {$4$};
% 			\node at (4, 0) {$5$};
% 			\node at (2,-1) {$5$};
% 			\node at (3,-1) {$3$};
% 			\node at (4,-1) {$4$};

% 			% Could re-order rows to reach same result
% 			\node at (0,-2) {$3$};
% 			\node at (0,-3) {$4$};
% 			\node at (0,-4) {$5$};
	

% 			\coordinate (Q) at (-0.5, 0.5);
			
% 			\draw (Q)+(0, 0) -- +(5,0);
% 			\draw (Q)+(0,-1) -- +(5,-1);
% 			\draw (Q)+(0,-2) -- +(5,-2);
% 			\draw (Q)+(0,-3) -- +(5,-3);
% 			\draw (Q)+(0,-4) -- +(5,-4);
% 			\draw (Q)+(0,-5) -- +(5,-5);
	
% 			\draw (Q)+(0, 0) -- +(0,-5);
% 			\draw (Q)+(1, 0) -- +(1,-5);
% 			\draw (Q)+(2, 0) -- +(2,-5);
% 			\draw (Q)+(3, 0) -- +(3,-5);
% 			\draw (Q)+(4, 0) -- +(4,-5);
% 			\draw (Q)+(5, 0) -- +(5,-5);
% 		\end{tikzpicture}\hfill
% 		\begin{tikzpicture}
% 			\node at (0,0) {$A$};
% 			\node at (1,0) {$B$};
% 			\node at (0,-1) {$C$};
% 			\node at (1,-1) {$D$};
	
% 			% \node at (2,0) {$C$};
% 			\node at (2.5,0) {$C,D$};
% 			% \node at (3,0) {$D$};
% 			\node at (4,0) {$E$};
	
	
% 			% \node at (3,0) {$C,D,E$};
% 			% \node at (3,-1) {$A,B,E$};
% 			\node at (3,-1) {$E$};
% 			\node at (2,-1) {$A/B$};
% 			\node at (4,-1) {$B/A$};
	
% 			\node[rotate=-90] at (0,-3) {$B,D,E$};
% 			\node[rotate=-90] at (1,-3) {$A,C,E$};
	
% 			\node[label={[align=center]$A,A,B,B$\\$C,C,D,D,E$}] at (3,-3.5) {};
	
	
% 			\coordinate (Q) at (-0.5, 0.5);
			
% 			\draw (Q)+(0, 0) -- +(5,0);
% 			\draw (Q)+(0,-1) -- +(5,-1);
% 			\draw (Q)+(0,-2) -- +(5,-2);
	
% 			% \draw (Q)+(0,-3) -- +(1,-3);
% 			% \draw (Q)+(0,-4) -- +(1,-4);
	
% 			\draw (Q)+(0,-5) -- +(5,-5);
	
% 			\draw (Q)+(0, 0) -- +(0,-5);
% 			\draw (Q)+(1, 0) -- +(1,-5);
% 			\draw (Q)+(2, 0) -- +(2,-5);
	
% 			\draw (Q)+(3, -1) -- +(3,-2);
% 			\draw (Q)+(4, 0) -- +(4,-2);
			
% 			\draw (Q)+(5, 0) -- +(5,-5);
% 		\end{tikzpicture}\hfill\phantom{.}

% 		Notice there is one possible cell for the $E$ traversal in the first column as $E$ already contains $3$ and $5$. We now consider the two cases for the first row:

% 	\begin{mdframed}
% 		If the $D$ traversal contains the $3$ from the first row, then this forces the $5$ of the first column to be part of $D$, this fixes the first column and second row by forcing $B$ to contain $3$. 

% 		\begin{tikzpicture}
% 			\node at (0,0) {$A$};
% 			\node at (1,0) {$B$};
% 			\node at (0,-1) {$C$};
% 			\node at (1,-1) {$D$};
	
% 			% \node at (2,0) {$C$};
% 			% \node at (3,0) {$D$};
% 			\node at (2,0) {$D$};
% 			\node at (3,0) {$C$};
% 			\node at (4,0) {$E$};
	
	
% 			% \node at (3,0) {$C,D,E$};
% 			% \node at (3,-1) {$A,B,E$};
% 			\node at (3,-1) {$E$};
% 			\node at (2,-1) {$A$};
% 			\node at (4,-1) {$B$};

% 			\node at (0,-2) {$B$};
% 			\node at (0,-3) {$E$};
% 			\node at (0,-4) {$D$};
	


% 			% \node[rotate=-90] at (0,-3) {$B,D,E$};
% 			\node[rotate=-90] at (1,-3) {$A,C,E$};
	
% 			\node[label={[align=center]$A,A,B,B$\\$C,C,D,D,E$}] at (3,-3.5) {};
	
	
% 			\coordinate (Q) at (-0.5, 0.5);
			
% 			\draw (Q)+(0, 0) -- +(5,0);
% 			\draw (Q)+(0,-1) -- +(5,-1);
% 			\draw (Q)+(0,-2) -- +(5,-2);
	
% 			\draw (Q)+(0,-3) -- +(1,-3);
% 			\draw (Q)+(0,-4) -- +(1,-4);
	
% 			\draw (Q)+(0,-5) -- +(5,-5);
	
% 			\draw (Q)+(0, 0) -- +(0,-5);
% 			\draw (Q)+(1, 0) -- +(1,-5);
% 			\draw (Q)+(2, 0) -- +(2,-5);
	
% 			\draw (Q)+(3, 0) -- +(3,-2);
% 			\draw (Q)+(4, 0) -- +(4,-2);
			
% 			\draw (Q)+(5, 0) -- +(5,-5);
% 		\end{tikzpicture}

% 		We have the following traversals so far: 
% 		\[
% 			\set{1,5}\subset A,	\quad\set{2,3,4}\subset B, \quad\set{2,4}\subset C,\quad\set{1,3,5}\subset D,\quad\set{3,4,5}\subset E
% 		\]
% 		Notice that $E$ is missing a $1$ and $2$, however

% 	\end{mdframed}
\end{enumerate}

\end{document}
